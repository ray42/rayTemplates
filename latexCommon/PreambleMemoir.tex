%%%% Preamble common to Memoir only



%%%%%%%%%%%%%%%%%%%%%%%%%%%%%%%%%%%%%%%%%%%%%%%%%%%%%%%%%%%%%%%%%%%%%%%%%%%%
%%%%%%%%%%%%%%%%%%%%%%%%%%%%%%%%%%%%%%%%%%%%%%%%%%%%%%%%%%%%%%%%%%%%%%%%%%%%



%---------------------------------------------------------------------------
% enumerate number per section
%---------------------------------------------------------------------------

\usepackage{enumitem}
% Note: This also allows us to do something like this:
%\begin{enumerate}[label=\FOO*]
% \alph* = a, \Alph* = A, \roman* = i, ii, etc...
% Put in parenthesis for parenthesis. I.e.
% (\alph*) = (a)
% Label: \alph, \Alph, \arabic, \roman and \Roman


\setenumerate[1]{label=\thesection.\arabic*.}
\setenumerate[2]{label*=\arabic*.}


%%%%%%%%%%%%%%%%%%%%%%%%%%%%%%%%%%%%%%%%%%%%%%%%%%%%%%%%%%%%%%%%%%%%%%%%%%%%
%%%%%%%%%%%%%%%%%%%%%%%%%%%%%%%%%%%%%%%%%%%%%%%%%%%%%%%%%%%%%%%%%%%%%%%%%%%%

%\usepackage{mathpazo} % idk why I even needed this.
%\usepackage{epigraph} % already loaded by memoir class.
\setlength{\epigraphwidth}{.75\linewidth}



%---------------------------------------------------------------------------
% Other packages, maybe not used.
%---------------------------------------------------------------------------

%this is obsolete, do not use it if someone suggests it.
%\usepackage{subfigure}

% sidecap  --  Typeset captions sideways
%\usepackage{sidecap}






%%%%%%%%%%%%%%%%%%%%%%%%%%%%%%%%%%%%%%%%%%%%%%%%%%%%%%%%%%%%%%%%%%%%%%%%%%%%
%%%%%%%%%%%%%%%%%%%%%%%%%%%%%%%%%%%%%%%%%%%%%%%%%%%%%%%%%%%%%%%%%%%%%%%%%%%%
%
% Mathematic symbols
%
%%%%%%%%%%%%%%%%%%%%%%%%%%%%%%%%%%%%%%%%%%%%%%%%%%%%%%%%%%%%%%%%%%%%%%%%%%%%
%%%%%%%%%%%%%%%%%%%%%%%%%%%%%%%%%%%%%%%%%%%%%%%%%%%%%%%%%%%%%%%%%%%%%%%%%%%%

% from mathtools, 
% floor and ceiling, usage:
% \floor*{ foo }
% and the unstarred version takes an optional argument that can be 
% \big, \Big, etc
% \abs[\Bigg]{\frac{a}{b}}
%% NOTE: for \brace - use \set
\DeclarePairedDelimiter\ceil{\lceil}{\rceil}
\DeclarePairedDelimiter\floor{\lfloor}{\rfloor}
\DeclarePairedDelimiter\braket{\langle}{\rangle} % < >
\DeclarePairedDelimiter\abs{\lvert}{\rvert}      % | |
\DeclarePairedDelimiter\norm{\lVert}{\rVert}     %|| ||
\DeclarePairedDelimiter\paren{\lparen}{\rparen}  % ( )
\DeclarePairedDelimiter\brce{\lbrace}{\rbrace}   % { }
\DeclarePairedDelimiter\brck{\lbrack}{\rbrack}   % [ ]
\DeclarePairedDelimiter\Brck{\lBrack}{\rBrack}   % [[ ]]

%%%%%%%%%%%%%%%%%%%%%%%%%%%%%%%%%%%%%%%%%%%%%%%%%%%%%%%%
%% Auxiliary symbols, like conjuate

% The below monstrocity is for a hacked widebar, used for \conj
% See https://tex.stackexchange.com/questions/16337/can-i-get-a-widebar-without-using-the-mathabx-package/60253#60253
%\makeatletter
%\let\save@mathaccent\mathaccent
%\newcommand*\if@single[3]{%
%  \setbox0\hbox{${\mathaccent"0362{#1}}^H$}%
%  \setbox2\hbox{${\mathaccent"0362{\kern0pt#1}}^H$}%
%  \ifdim\ht0=\ht2 #3\else #2\fi
%  }
%%The bar will be moved to the right by a half of \macc@kerna, which is computed by amsmath:
%\newcommand*\rel@kern[1]{\kern#1\dimexpr\macc@kerna}
%%If there's a superscript following the bar, then no negative kern may follow the bar;
%%an additional {} makes sure that the superscript is high enough in this case:
%\newcommand*\widebar[1]{\@ifnextchar^{{\wide@bar{#1}{0}}}{\wide@bar{#1}{1}}}
%%Use a separate algorithm for single symbols:
%\newcommand*\wide@bar[2]{\if@single{#1}{\wide@bar@{#1}{#2}{1}}{\wide@bar@{#1}{#2}{2}}}
%\newcommand*\wide@bar@[3]{%
%  \begingroup
%  \def\mathaccent##1##2{%
%%Enable nesting of accents:
%    \let\mathaccent\save@mathaccent
%%If there's more than a single symbol, use the first character instead (see below):
%    \if#32 \let\macc@nucleus\first@char \fi
%%Determine the italic correction:
%    \setbox\z@\hbox{$\macc@style{\macc@nucleus}_{}$}%
%    \setbox\tw@\hbox{$\macc@style{\macc@nucleus}{}_{}$}%
%    \dimen@\wd\tw@
%    \advance\dimen@-\wd\z@
%%Now \dimen@ is the italic correction of the symbol.
%    \divide\dimen@ 3
%    \@tempdima\wd\tw@
%    \advance\@tempdima-\scriptspace
%%Now \@tempdima is the width of the symbol.
%    \divide\@tempdima 10
%    \advance\dimen@-\@tempdima
%%Now \dimen@ = (italic correction / 3) - (Breite / 10)
%    \ifdim\dimen@>\z@ \dimen@0pt\fi
%%The bar will be shortened in the case \dimen@<0 !
%    \rel@kern{0.6}\kern-\dimen@
%    \if#31
%      \overline{\rel@kern{-0.6}\kern\dimen@\macc@nucleus\rel@kern{0.4}\kern\dimen@}%
%      \advance\dimen@0.4\dimexpr\macc@kerna
%%Place the combined final kern (-\dimen@) if it is >0 or if a superscript follows:
%      \let\final@kern#2%
%      \ifdim\dimen@<\z@ \let\final@kern1\fi
%      \if\final@kern1 \kern-\dimen@\fi
%    \else
%      \overline{\rel@kern{-0.6}\kern\dimen@#1}%
%    \fi
%  }%
%  \macc@depth\@ne
%  \let\math@bgroup\@empty \let\math@egroup\macc@set@skewchar
%  \mathsurround\z@ \frozen@everymath{\mathgroup\macc@group\relax}%
%  \macc@set@skewchar\relax
%  \let\mathaccentV\macc@nested@a
%%The following initialises \macc@kerna and calls \mathaccent:
%  \if#31
%    \macc@nested@a\relax111{#1}%
%  \else
%%If the argument consists of more than one symbol, and if the first token is
%%a letter, use that letter for the computations:
%    \def\gobble@till@marker##1\endmarker{}%
%    \futurelet\first@char\gobble@till@marker#1\endmarker
%    \ifcat\noexpand\first@char A\else
%      \def\first@char{}%
%    \fi
%    \macc@nested@a\relax111{\first@char}%
%  \fi
%  \endgroup
%}
%\makeatother
%\newcommand{\conj}[1]{\widebar{#1}}
\newcommand{\conj}[1]{\overline{#1}} %conjugate
\newcommand{\rrT}{\intercal} %transpose
\newcommand{\rrInv}{{-1}} %inverse
\newcommand{\rrxor}{\textasciicircum} % XOR, ^

\DeclareMathOperator{\diag}{diag} % diag
\DeclareMathOperator{\lcm}{lcm} % Least common multiple

%%%%%%%%%% This is for variables
% https://tex.stackexchange.com/questions/98149/to-imply-a-dash-and-not-minus-sign-in-math-mode
% This is so that hyphen will get interpreted as hyphen in mathmode, not as
% minus sign.
% Non-italic version
\newcommand{\varA}[1]{{\operatorname{#1}}}
% Italic version
\newcommand{\varB}[1]{{\ensuremath{\operatorname{\mathit{#1}}}}}

%%%%%%%%%%%%%%%%%%%%%%%%%%%%%%%%%%%%%%%%%%%%%%%%%%%%%%%%

% Use the set with \given, e.g.
%$\set{ x \given x > 0 }$
%$\set[\big]{ x \given x > 0 }$
%$\set[\Big]{ x \given x > 0 }$
%$\set[\bigg]{ x \given x > 0 }$
%$\set[\Bigg]{ x \given x > 0 }$
%$\set*{ x \given x > 0 \rule{0cm}{2cm} }$ % \rule is just here so you can see that it autostretches
\DeclarePairedDelimiterX\set[1]\lbrace\rbrace{\def\given{\;\delimsize\vert\;}#1}

% Probability, we use \Pr\set*{} to get Pr{...}
% Usage: \Prr[optional: power]{mandatory X}
% o = optional, m = mandatory
\NewDocumentCommand{\Prr}{o m}{%
  \IfValueTF{#1}{%
    \Pr^{#1}\set*{#2}%
  }{%
    \Pr\set*{#2}%
  }%
}

% Usage: \Err[optional: power]{mandatory X}
\DeclareMathOperator{\Expct}{E}
% o = optional, m = mandatory
\NewDocumentCommand{\Err}{o m}{%
  \IfValueTF{#1}{%
    \Expct^{#1}\brck*{#2}%
  }{%
    \Expct\brck*{#2}%
  }%
}

% Usage: \Vrr[optional: power]{mandatory X}
\DeclareMathOperator{\Var}{Var}
% o = optional, m = mandatory
\NewDocumentCommand{\Vrr}{o m}{%
  \IfValueTF{#1}{%
    \Var^{#1}\brck*{#2}%
  }{%
    \Var\brck*{#2}%
  }%
}

% Indicator vairable, used in ch5 CLRS
% Usage: \Irr[optional: power]{mandatory X}
\DeclareMathOperator{\Indct}{I}
% o = optional, m = mandatory
\NewDocumentCommand{\Irr}{o m}{%
  \IfValueTF{#1}{%
    \Indct^{#1}\set*{#2}%
  }{%
    \Indct\set*{#2}%
  }%
}

% Black-Height, used in ch13 CLRS.
% Usage: \Bh[optional: power]{mandatory X}
\DeclareMathOperator{\RRWBh}{bh}
% o = optional, m = mandatory
\NewDocumentCommand{\Bh}{o m}{%
  \IfValueTF{#1}{%
    \RRWBh^{#1}\paren*{#2}%
  }{%
    \RRWBh\paren*{#2}%
  }%
}




%%%%%%%%%%%%%%%%%%%%%%%%%%%%%%%%%%%%%%%%%%%%%%%%%%%%%%%%%%%%%%%%%%%%%%%%%%%%
% Usage: \Reach[optional: via node]
% o = optional, m = mandatory
\NewDocumentCommand{\Reach}{o}{
  \IfValueTF{#1}{
    \overset{#1}{\leadsto}
  }{
    \leadsto
  }
}


% https://tex.stackexchange.com/questions/5502/how-to-get-a-mid-binary-relation-that-grows
% usage: \remiddle|
% NOTE: use \given instead
%\newcommand{\relmiddle}[1]{\mathrel{}\middle#1\mathrel{}}

% big-o notation for complexity
% Also tests for empty, so I can use \bigo{} and it doesn't provide
% brackets.
\DeclareRobustCommand{\comBigOh}[1]{
\ifthenelse{\equal{#1}{}}{\mathcal{O}}
                         {\mathcal{O}\left(#1\right)}}

\DeclareRobustCommand{\comLittleOh}[1]{
\ifthenelse{\equal{#1}{}}{o}
                         {o\left(#1\right)}}
% theta- for worse case complexity.
% Okay, this is not worse case. I should call these comTheta and such.
\DeclareRobustCommand{\comTheta}[1]{
\ifthenelse{\equal{#1}{}}{\Theta}
                         {\Theta\left(#1\right)}}

\DeclareRobustCommand{\comBigOmega}[1]{
\ifthenelse{\equal{#1}{}}{\Omega}
                         {\Omega\left(#1\right)}}
\DeclareRobustCommand{\comLittleOmega}[1]{
\ifthenelse{\equal{#1}{}}{\omega}
                         {\omega\left(#1\right)}}


%% Set notation %%%%%%%%%%%%%%%%%%%%%%%%%%%%%%%%%%%%%%%%%%%%%%%%%%%%%%%%%%%%
% empty set:
% Try \varnothing from the amssymb package. It is perfectly round, and the 
% comprehensive LaTeX symbol list states that it is preferred by many to 
% \emptyset.
% https://tex.stackexchange.com/questions/22798/nice-looking-empty-set
\let\oldemptyset\emptyset
\let\emptyset\varnothing

\newcommand{\setEmpty}{\emptyset}%empty set
\newcommand{\setPrime}{\mathbb{P}} %prime numbers
\newcommand{\setWhole}{\mathbb{W}} %whole numbers
\newcommand{\setNat}{\mathbb{N}} %natural numbers
\newcommand{\setInt}{\mathbb{Z}} %integers,
\newcommand{\setIrrat}{\mathbb{I}} %irrational numbers
\newcommand{\setRat}{\mathbb{Q}} %rational numbers
\newcommand{\setReal}{\mathbb{R}} %real numbers
\newcommand{\setCom}{\mathbb{C}} %complex numbers


%% Constants %%%%%%%%%%%%%%%%%%%%%%%%%%%%%%%%%%%%%%%%%%%%%%%%%%%%%%%%%%%%
\newcommand\ct[1]{\text{\rmfamily\upshape #1}}
\newcommand{\rNatNum}{\ct{e}}% natural number
\newcommand{\rGoldRat}{\upphi} % Golden Ratio
\newcommand{\rPi}{\uppi} % pi
\newcommand{\rIm}{\ct{i}}% Imaginary
\newcommand{\zmat}{0} % zero matrix
\newcommand{\eyemat}{I} % identity matrix


%% Integrals and differentials %%%%%%%%%%%%%%%%%%%%%%%%%%%%%%%%%%%%%%%%%%
% https://tex.stackexchange.com/questions/60545/should-i-mathrm-the-d-in-my-integrals
\newcommand*{\diff}[1]{
\ifthenelse{\isempty{#1}}{\mathop{}\!\mathrm{d}}
                         {\mathop{}\!\mathrm{d^{#1}}}}


%%%%%%%%%%%%%%%%%%%%%%%%%%%%%%%%%%%%%%%%%%%%%%%%%%%%%%%%%%%%%%%%%%%%%%%%%%%%
%%%%%%%%%%%%%%%%%%%%%%%%%%%%%%%%%%%%%%%%%%%%%%%%%%%%%%%%%%%%%%%%%%%%%%%%%%%%
%
% END OF Mathematic symbols
%
%%%%%%%%%%%%%%%%%%%%%%%%%%%%%%%%%%%%%%%%%%%%%%%%%%%%%%%%%%%%%%%%%%%%%%%%%%%%
%%%%%%%%%%%%%%%%%%%%%%%%%%%%%%%%%%%%%%%%%%%%%%%%%%%%%%%%%%%%%%%%%%%%%%%%%%%%



%---------------------------------------------------------------------------
% Page layout
%---------------------------------------------------------------------------
\settrimmedsize{297mm}{210mm}{*}
\setlength{\trimtop}{0pt}
\setlength{\trimedge}{\stockwidth}
\addtolength{\trimedge}{-\paperwidth}
\settypeblocksize{634pt}{448.13pt}{*}
\setulmargins{4cm}{*}{*}
\setlrmargins{*}{*}{1.5}
\setmarginnotes{17pt}{51pt}{\onelineskip}
\setheadfoot{\onelineskip}{2\onelineskip}
\setheaderspaces{*}{2\onelineskip}{*}
\checkandfixthelayout

%\setlrmarginsandblock{0.15\paperwidth}{*}{1} % Left and right margin
%\setulmarginsandblock{0.2\paperwidth}{*}{1}  % Upper and lower margin
%\checkandfixthelayout

%%% SECTIONAL DIVISIONS
%%%-------------------------------------------------------------------------

\maxsecnumdepth{subsubsection} % Subsubsections (and higher) are numbered
\setsecnumdepth{subsubsection}

\makeatletter %
\makechapterstyle{standard}{
  \setlength{\beforechapskip}{0\baselineskip}
  \setlength{\midchapskip}{1\baselineskip}
  \setlength{\afterchapskip}{8\baselineskip}
  \renewcommand{\chapterheadstart}{\vspace*{\beforechapskip}}
  \renewcommand{\chapnamefont}{\centering\normalfont\Large}
  \renewcommand{\printchaptername}{\chapnamefont \@chapapp}
  \renewcommand{\chapternamenum}{\space}
  \renewcommand{\chapnumfont}{\normalfont\Large}
  \renewcommand{\printchapternum}{\chapnumfont \thechapter}
  \renewcommand{\afterchapternum}{\par\nobreak\vskip \midchapskip}
  \renewcommand{\printchapternonum}{\vspace*{\midchapskip}\vspace*{5mm}}
  \renewcommand{\chaptitlefont}{\centering\bfseries\LARGE}
  \renewcommand{\printchaptertitle}[1]{\chaptitlefont ##1}
  \renewcommand{\afterchaptertitle}{\par\nobreak\vskip \afterchapskip}
}
\makeatother

\chapterstyle{standard}

\setsecheadstyle{\normalfont\large\bfseries}
\setsubsecheadstyle{\normalfont\normalsize\bfseries}
\setparaheadstyle{\normalfont\normalsize\bfseries}
\setparaindent{0pt}\setafterparaskip{0pt}

%%% FLOATS AND CAPTIONS
%%%-------------------------------------------------------------------------

% Used for figure environment, can now use [H] to put the
% figure right here and force text to go below it.
\usepackage{float}

\makeatletter                  % You do not need to write [htpb] all the time
\renewcommand\fps@figure{htbp} %
\renewcommand\fps@table{htbp}  %
\makeatother                   %

\captiondelim{\space } % A space between caption name and text
\captionnamefont{\small\bfseries} % Font of the caption name
\captiontitlefont{\small\it} % Font of the caption text

\changecaptionwidth          % Change the width of the caption
\captionwidth{1\textwidth} %

%%%%%%%%%%%%%%%%%%%%%%%%%%%%%%%%%%%%%%%%%%%%%%%%%%%%%%%%%%%%%%%%%%%%%%%%%%%%
% diagbox to diagonally split a cell in a table
% https://ctan.org/pkg/diagbox
% Usage: \diagbox{Col name}{Row name}
% https://tex.stackexchange.com/questions/110018/professional-slashbox-alternative
\usepackage{diagbox}

%%%%%%%%%%%%%%%%%%%%%%%%%%%%%%%%%%%%%%%%%%%%%%%%%%%%%%%%%%%%%%%%%%%%%%%%%%%%
% Long table, see: http://tex.stackexchange.com/questions/301227/how-to-make-table-split-in-two-pages
\usepackage{longtable}
% Usage:
%To switch from a table/tabular combination to a longtable setup, only the
%following changes are required:
%
%1)Terminate the \caption directive with a double backslash;
%2)organize the header and footer material with \endhead, \endfoot, and 
%  possibly \endfirsthead and \endlastfoot directives; and
%3)omit the \centering directive that's usually provided for tabular 
%  material.
%
%\begin{longtable}{ccc}
%\caption{With \texttt{longtable} environment}\label{tab:b}\\
%% header and footer information
%\hline
%H1 & H2 & H3 \\
%\hline
%\endhead
%\hline
%\endfoot
%% body of table
%123 & 456 & 789 \\
%\end{longtable}

% Now making it smaller with footnotesize
%\begin{footnotesize}
%\begin{longtable}{|l p{10cm}|}
%\caption[Operations to Find Elements in an Associative Container]
%{Operations to Find Elements in an Associative Container}
%\label{tabC11N7OperationsToFindElementsInAnAssociativeContainer}\\
%% header and footer information
%\hline
%\endfirsthead
%\hline
%\endlastfoot
%% body of table
%\multicolumn{2}{|l|}{\textbf{Type Aliases}} \\
%\ctt{iterator}&Type of the iterator for this container type \\
%\end{longtable}
%\end{footnotesize}

%%%%%%%%%%%%%%%%%%%%%%%%%%%%%%%%%%%%%%%%%%%%%%%%%%%%%%%%%%%%%%%%%%%%%%%%%%%%

% How to use itemize inside table? Use fake items:
% http://tex.stackexchange.com/questions/150492/how-to-use-itemize-in-table-environment
\usepackage{booktabs}% http://ctan.org/pkg/booktabs
\newcommand{\tabitem}{~~\llap{\textbullet}~~}


%\usepackage[textfont={small,it}, labelfont={small,bf}]{caption}
\usepackage[textfont={small}, labelfont={small,bf}]{caption}
\usepackage[font={small,it},labelformat=simple]{subcaption}
\renewcommand\thesubfigure{(\alph{subfigure})}

%\usepackage{tabularx}


%%% ABSTRACT
%%%-------------------------------------------------------------------------
% Font of abstract title
\renewcommand{\abstractnamefont}{\normalfont\small\bfseries} 
% Width of abstract
\setlength{\absleftindent}{0.1\textwidth} 
\setlength{\absrightindent}{\absleftindent}

%%% HEADER AND FOOTER 
%%%-------------------------------------------------------------------------

\makepagestyle{standard} % Make standard pagestyle

\makeatletter                 % Define standard pagestyle
\makeevenfoot{standard}{}{}{} %
\makeoddfoot{standard}{}{}{}  %
\makeevenhead{standard}{\bfseries\thepage\normalfont\qquad\small\leftmark}{}{}
\makeoddhead{standard}{}{}{\small\rightmark\qquad\bfseries\thepage}
% \makeheadrule{standard}{\textwidth}{\normalrulethickness}
\makeatother                  %

\makeatletter
\makepsmarks{standard}{
\createmark{chapter}{both}{shownumber}{\@chapapp\ }{ \quad }
\createmark{section}{right}{shownumber}{}{ \quad }
\createplainmark{toc}{both}{\contentsname}
\createplainmark{lof}{both}{\listfigurename}
\createplainmark{lot}{both}{\listtablename}
\createplainmark{bib}{both}{\bibname}
\createplainmark{index}{both}{\indexname}
\createplainmark{glossary}{both}{\glossaryname}
}
\makeatother                               %

\makepagestyle{chap} % Make new chapter pagestyle

\makeatletter
\makeevenfoot{chap}{}{\small\bfseries\thepage}{} % Define new chapter pagestyle
\makeoddfoot{chap}{}{\small\bfseries\thepage}{}  %
\makeevenhead{chap}{}{}{}   %
\makeoddhead{chap}{}{}{}    %
% \makeheadrule{chap}{\textwidth}{\normalrulethickness}
\makeatother

\nouppercaseheads
\pagestyle{standard}           % Choosing pagestyle and chapter pagestyle
\aliaspagestyle{chapter}{chap} %


%%%%%%%%%%%%%%%%%%%%%%%%%%%%%%%%%%%%%%%%%%%%%%%%%%%%%%%%%%%%%%%%%%%%%%%%%%%%
%%%%%%%%%%%%%%%%%%%%%%%%%%%%%%%%%%%%%%%%%%%%%%%%%%%%%%%%%%%%%%%%%%%%%%%%%%%%
%%% NEW COMMANDS
%%%%%%%%%%%%%%%%%%%%%%%%%%%%%%%%%%%%%%%%%%%%%%%%%%%%%%%%%%%%%%%%%%%%%%%%%%%%
%%%%%%%%%%%%%%%%%%%%%%%%%%%%%%%%%%%%%%%%%%%%%%%%%%%%%%%%%%%%%%%%%%%%%%%%%%%%


% Multiline cell: (rayray multi cell)
\newcommand{\rrmcell}[2][c]{%
  \begin{tabular}[#1]{@{}c@{}}#2\end{tabular}}
% From here: http://tex.stackexchange.com/questions/2441/how-to-add-a-forced-line-break-inside-a-table-cell
% Usage:
%Foo bar & \specialcell{Foo\\bar} & Foo bar \\    % vertically centered
%Foo bar & \specialcell[t]{Foo\\bar} & Foo bar \\ % aligned with top rule
%Foo bar & \specialcell[b]{Foo\\bar} & Foo bar \\ % aligned with bottom rule
%For those wanting to control the horizontal alignment, 
% change c@ to l@ or r@ (or make it another parameter like the vertical 
% alignment?)
% Note: This doesn't seem to work with \ctt, so I'll use parbox:
% \parbox[t]{5cm}{foo\\bar}


%---------------------------------------------------------------------------
% Custom higlighting.
%---------------------------------------------------------------------------
% use \ctt{} for code
%     \rnhl{} for highlighting importanting keywords.
%     \rnblue{} for random notes
%     \rngreen{} for random notes



%\definecolor{darkred}{rgb}{0.8,0.1,0.1}


% We use the soul package to do high-lighting (change background colour)
% of code words. We are not using \colorbox since text inside \colorbox can 
% not be split into several lines if it is too long. So the macro is most 
% suited for coloring a few words. If you want to color several sentences 
% which spread across several lines, it is better to use soul.
\usepackage{soul} % \sethlcolor{}, \hl
                  % \setulcolor{}, \ul


% Now use hl to change the background.
% Note: Putting \hl INSIDE \textcolor is important, if we put it outside,
% it would not work.

% The command \sethlcolor is fragile and will break in moving arguments. 
% You can either \protect it in moving arguments or, better, declare it 
% robust from the beginning:
\DeclareRobustCommand{\ctt}[1]{\texttt{\sethlcolor{light-gray}\hl{#1}}}
\DeclareRobustCommand{\cit}[1]{\textit{\sethlcolor{light-gray}\hl{#1}}}
\DeclareRobustCommand{\cbt}[1]{\textbf{\texttt{\sethlcolor{light-gray}\hl{#1}}}}

% To be used in section titles, since we need texorpdfstring.
% Reason: The PDF bookmarks are a different thing than the table of 
% contents. The bookmarks are not typeset by TeX: they simply are strings 
% of characters, so no math or general formatting instructions are allowed.
%\DeclareRobustCommand{\sctt}[1]{\texorpdfstring{\ctt{#1}}\texorpdfstring{\,}}

%$ ray ray text less than, greater than
%\newcommand*{\cttltgt}[1]{\textless{} #1 \textgreater{}}
% to work.

\definecolor{mynotescolourblue}{rgb}{0.0,0.0,0.3}
\definecolor{mynotescolourgreen}{rgb}{0, 0.3, 0}
\colorlet{mynotescolourred}{red!50!black!90!}

%% Now we do a few more of these, I like these...
%\DeclareRobustCommand{\hlcyan}[1]{{\sethlcolor{cyan}\hl{#1}}}
% rnhl = ray notes high light
% for highlighting keywords.
\DeclareRobustCommand{\rrhl}[1]{{\textcolor{mynotescolourred}{\sethlcolor{GreenYellow}\hl{\textbf{#1}}}}}

% ray attention: for things I should pay more attention to, for example, if 
% the book says "we read from a file", I may not pay attention to the word
% "file" much, although it is vital to the design of my program.
\DeclareRobustCommand{\rra}[1]{\setulcolor{mynotescolourred}\ul{#1}}

% rrgreen for green text for things I understand, but wanted to re-word.
\DeclareRobustCommand{\rrgreen}[1]{{\textcolor{mynotescolourgreen}{#1}}}
% Italic version
\DeclareRobustCommand{\rrgreenit}[1]{\rrgreen{\textit{#1}}}
% Typewriter version
\DeclareRobustCommand{\rrgreentt}[1]{\texttt[mynotescolourgreen]{#1}}


% rrred for red text for things I do not understand.
\DeclareRobustCommand{\rrred}[1]{{\textcolor{mynotescolourred}{#1}}}
% Italic version
\DeclareRobustCommand{\rrredit}[1]{\rrred{\textit{#1}}}
% Typewriter version
\DeclareRobustCommand{\rrredtt}[1]{\texttt[mynotescolourred]{#1}}


% rrblue blue text. There is not yet a specific use for this.
\DeclareRobustCommand{\rrblue}[1]{{\textcolor{mynotescolourblue}{#1}}}
% Italic version
\DeclareRobustCommand{\rrblueit}[1]{\rrblue{\textit{#1}}}
% Typewriter version
\DeclareRobustCommand{\rrbluett}[1]{\texttt[mynotescolourblue]{#1}}


% Usage: \source{Title}{Url}
% Title will be bold and underline. Url will be red{textbf{Source}}: Url
\DeclareRobustCommand{\sourceurl}[2]
{\noindent{\textbf{\underline{#1}}}\newline
 \textcolor{mynotescolourred}{\textbf{Source: }}\url{#2}}

\DeclareRobustCommand{\bkpgrf}[1]{\textbf{\rrred{BookPage: #1}}}

% Usage: \mdframedrule = a horizontal line for division.
\DeclareRobustCommand{\mdframedrule}{\noindent\rule{\textwidth}{1pt}}

%\newcommand{\pagerule}{\noindent\hrulefill}
%\newcommand{\dpagerule}{\noindent\hrulefill\newline\noindent\hrulefill}




\newlength{\qqseplinewidth}
\newlength{\qqseplinesep}
\setlength{\qqseplinewidth}{1mm}
\setlength{\qqseplinesep}{2mm}
\colorlet{qqsepline}{PaleVioletRed3}

\newcommand*{\qqsepline}{%
  \par
  \vspace{\dimexpr\qqseplinesep+.5\parskip}%
  \cleaders\vbox{%
    \begingroup % because of color
      \color{qqsepline}%
      \hrule width\linewidth height\qqseplinewidth
    \endgroup
  }\vskip\qqseplinewidth
  \vspace{\dimexpr\qqseplinesep-.5\parskip}%
}

\newcommand*{\qasepline}{\noindent\hfil\rule{0.5\textwidth}{.4pt}\hfil}

%https://tex.stackexchange.com/questions/19902/drawing-horizontal-line-same-width-as-the-page-width
\newcommand*{\linemarginlength}{\noindent\makebox[\linewidth]{\rule{\textwidth}{1pt}}}
\newcommand*{\linepagelength}{\noindent\makebox[\linewidth]{\rule{\paperwidth}{1pt}}}




\newcommand*{\exersepline}{\noindent\hfil\colorbox{qqsepline}{\rule{0.5\textwidth}{.4pt}}\hfil}


\newlength{\rrseplinewidth}
\newlength{\rrseplinesep}
\setlength{\rrseplinewidth}{1mm}
\setlength{\rrseplinesep}{2mm}
\colorlet{rrsepline}{BlueGreen}

\newcommand*{\rrsepline}{%
  \par
  \vspace{\dimexpr\rrseplinesep+.5\parskip}%
  \cleaders\vbox{%
    \begingroup % because of color
      \color{rrsepline}%
      \hrule width\linewidth height\rrseplinewidth
    \endgroup
  }\vskip\rrseplinewidth
  \vspace{\dimexpr\rrseplinesep-.5\parskip}%
}

\newcommand*{\rrrsepline}{\noindent\hfil\colorbox{rrsepline}{\rule{0.5\textwidth}{.4pt}}\hfil}


% Underlined and bold and newlined
% the \phantomsection is for when we use label{rh:foo}
\DeclareRobustCommand{\rrheaderunderline}[1]{\par\medskip\phantomsection\noindent\textbf{\underline{#1}}\medskip}
\DeclareRobustCommand{\rrheader}[1]{\par\medskip\phantomsection\noindent{\normalsize\textbf{#1}}\medskip}
\DeclareRobustCommand{\rrheaderlarge}[1]{\par\medskip\phantomsection\noindent{\large\textbf{#1}}\medskip}
\DeclareRobustCommand{\rrheaderLarge}[1]{\par\medskip\phantomsection\noindent{\Large\textbf{#1}}\medskip}


%\setlength{\parindent}{0pt}
\nonzeroparskip


% For my own notes.
\newcommand{\RayNotesBegin}{\rrsepline{}\begingroup\color{mynotescolourblue}}
\newcommand{\RayNotesEnd}{\endgroup\rrsepline{}}





%%%%%%%%%%%%%%%%%%%%%%%%%%%%%%%%%%%%%%%%%%%%%%%%%%%%%%%%%%%%%%%%%%%%%%%%%%%%
%%%%%%%%%%%%%%%%%%%%%%%%%%%%%%%%%%%%%%%%%%%%%%%%%%%%%%%%%%%%%%%%%%%%%%%%%%%%
%%%%%%%%%%%%%%%%%%%%%%%%%%%%%%%%%%%%%%%%%%%%%%%%%%%%%%%%%%%%%%%%%%%%%%%%%%%%
% https://tex.stackexchange.com/questions/133113/continued-figures
% continued figure coutners:

%Before the continued figure add these lines:
%
%\renewcommand{\thefigure}{\arabic{figure} (Cont.)} or
%\renewcommand{\thefigure}{\arabic{section}.\arabic{figure} (Cont.)}
%\renewcommand{\thefigure}{\arabic{chapter}.\arabic{figure} (Cont.)}
%\addtocounter{figure}{-1}
%and after it these line:
%
%\renewcommand{\thefigure}{\arabic{figure}}
%

\usepackage{chngcntr}


\usepackage{lipsum} % Just to put in some text


%%%%%%%%%%%%%%%%%%%%%%%%%%%%%%%%%%%%%%%%%%%%%%%%%%%%%%%%%%%%%%%%%%%%%%%%%%%%
% Cleveref, allows using cref
% NOTE: Cleverref has to be loaded LAST
% https://tex.stackexchange.com/questions/148699/equation-reference-undefined-when-using-cref-and-packageamsmath
% Otherwise it will not work:
% However, care must be taken when using cleveref in conjunction with other 
% packages that modify LaTeX's referencing system (see Section 11). 
% Basically, cleveref must be loaded last.
%%%%%%%%%%%%%%%%%%%%%%%%%%%%%%%%%%%%%%%%%%%%%%%%%%%%%%%%%%%%%%%%%%%%%%%%%%%%
\usepackage[sort&compress, % on multiple refs sort them and write as range
            capitalise, % Use Section not section etc.
            noabbrev, % Use Table not Tab. etc.
            nameinlink % Make the name (eg Section) part of the hyperlink
            ]{cleveref}




%%%%%%%%%%%%%%%%%%%%%%%%%%%%%%%%%%%%%%%%%%%%%%%%%%%%%%%%%%%%%%%%%%%%%%%%%%%%
%% New Theorems and environments, they have to come after loading cleverref
%% and clever ref must be the last package loaded.
%https://tex.stackexchange.com/questions/19104/cleveref-with-counters-with-same-name

\newcounter{exer}[section]
\newenvironment{exer}[1][]{\refstepcounter{exer}\par\medskip
  \noindent\textbf{Exercise~\theexer:#1} \rmfamily}{\medskip}

%\newtheoremstyle{exercise}
%  {\topsep}   % above space
%  {\topsep}   % below space
%  {\itshape}  % body font
%  {0pt}       % indent
%  {\bfseries} % head font
%  {}         % head punctuation
%  {5pt plus 1pt minus 1pt} % HEADSPACE
%  {}          % CUSTOM-HEAD-SPEC
%
%\theoremstyle{exercise}
%\newcounter{exercises}[chapter]
%\newtheorem{exer}[exercises]{Exercise}
%%\newtheorem{exers}{Exer}
%
\counterwithin{exer}{section}

%%%%%%%%%%%%%%%%%%%%%%%%%%%%%%%%%%%%%%%%%%%%%%%%%%%%%%%

\newcounter{prob}[chapter]
\newenvironment{prob}[1][]{\refstepcounter{prob}\par\medskip
  \noindent\textbf{Problem~\theprob:#1} \rmfamily}{\medskip}

%\newtheoremstyle{exercise}
%  {\topsep}   % above space
%  {\topsep}   % below space
%  {\itshape}  % body font
%  {0pt}       % indent
%  {\bfseries} % head font
%  {}         % head punctuation
%  {5pt plus 1pt minus 1pt} % HEADSPACE
%  {}          % CUSTOM-HEAD-SPEC
%
%\theoremstyle{exercise}
%\newcounter{exercises}[chapter]
%\newtheorem{exer}[exercises]{Exercise}
%%\newtheorem{exers}{Exer}
%
\counterwithin{prob}{chapter}


%%%%%%%%%%%%%%%%%%%%%%%%%%%%%%%%%%%%%%%%%%%%%%%%%%%%%%

%https://tex.stackexchange.com/questions/64931/using-newtheorem
%The \newtheorem command has two mutually exlusive optional arguments:
%
%Using
%\newtheorem{<name>}{<heading>}[<counter>]
%will create an environment <name> for a theorem-like structure; the counter
%for this structure will be subordinated to <counter>. 
%
%On the other hand, using
%\newtheorem{<name>}[<counter>]{<heading>}
%will create an environment <name> for a theorem-like structure; the counter
%for this structure will share the previously defined <counter> counter.
%
%In the definition of defn you need to use the first optional argument of
%\newtheorem to indicate that this environment shares the counter of the
%previously defined thm environment.
%
%If the counters need to be subordinate to the section counter, use section
%for the second optional argument of \newtheorem in the definition of thm.

\newtheorem{theorem}{Theorem}[chapter]

% A environment called corollary is created, the counter of this new
% environment will be reset every time a new theorem environment is used.
%\newtheorem{corollary}{Corollary}[theorem]
% changed so it shares the counter with theorem:
\newtheorem{corollary}[theorem]{Corollary}

% In this case, the even though a new environment called lemma is created,
% it will use the same counter as the theorem environment.
\newtheorem{lemma}[theorem]{Lemma}

\theoremstyle{remark}
\newtheorem*{remark}{Remark}

\theoremstyle{definition}
\newtheorem{definition}{Definition}[chapter]

% change the end of proof symbol
\renewcommand\qedsymbol{$\blacksquare$}


%%%%%%%%%%%%%%%%%%%%%%%%%%%%%%%%%%%%%%%%%%%%%%%%%%%%%%%%%%%%%%%%%%%%%%%%%%%%
%%%%%%%%%%%%%%%%%%%%%%%%%%%%%%%%%%%%%%%%%%%%%%%%%%%%%%%%%%%%%%%%%%%%%%%%%%%%
%% Continue with cleveref setup.




% Call subsections sections
%\crefname{subsection}{Section}{Sections}
%\Crefname{subsection}{Section}{Sections}

% Use the section symbol §
\crefname{subsection}{\S}{\S\S}
\Crefname{subsection}{\S}{\S\S}
\crefname{section}{\S}{\S\S}
\Crefname{section}{\S}{\S\S}


% just use (...) for equations
\crefformat{equation}{#2(#1)#3}
\crefrangeformat{equation}{#3(#1)#4--#5(#2)#6}
\crefmultiformat{equation}{(#2#1#3)}{ and~(#2#1#3)}{, (#2#1#3)}{ and~(#2#1#3)}

% Except for start of sentences where we need to say "Equations"
\Crefformat{equation}{Equation~#2(#1)#3}
\Crefrangeformat{equation}{Equations~#3(#1)#4--#5(#2)#6}
\Crefmultiformat{equation}{Equations~(#2#1#3)}{ and~(#2#1#3)}{, (#2#1#3)}{ and~(#2#1#3)}

%\newcommand\crefpairconjunction{--}
%\crefname{equation}{}{} % no "equation[s]" label mid-sentence
%\Crefname{equation}{Equation}{Equations}

% a reference for "this X"
%\newcommand{\thisref}[1]{this \lcnamecref{#1}}
%\newcommand{\Thisref}[1]{This \lcnamecref{#1}}

% Need this for listings
\crefname{lstlisting}{listing}{listings}
\Crefname{lstlisting}{Listing}{Listings}

% Need this for listings
\crefname{exer}{exercise}{exercises}
\Crefname{exer}{Exercise}{Exercises}

% Need this for listings
\crefname{prob}{problem}{problems}
\Crefname{prob}{Problem}{Problems}

%\crefname{lemma}{lemma}{lemmas}
%\Crefname{lemma}{Lemma}{Lemmas}
%\crefname{theorem}{theorem}{theorems}
%\Crefname{theorem}{Theorem}{Theorems}



% Teach hyperref how to handle cref inside sections, note, this is not used.
% But I'm keeping it here so I know how to do this in the future.
\pdfstringdefDisableCommands{\def\secexref#1{Example \ref{#1}}}
\DeclareRobustCommand*{\refcfirst}{\ref*{ex:sqlgr}}

% wrapper around cref to get the page number as well.
\DeclareRobustCommand{\pagecref}[1]{\cref{#1} p.\pageref{#1}}
\DeclareRobustCommand{\Pagecref}[1]{\Cref{#1} p.\pageref{#1}}

% Used for referencing exercises for the answer sections
%\DeclareRobustCommand{\Exerref}[1]{\noindent\textbf{\Pagecref{#1}:}}
\newcommand*{\Exerref}[1]{\noindent\textbf{\Pagecref{#1}:}}


%https://tex.stackexchange.com/questions/10102/multiple-references-to-the-same-footnote-with-hyperref-support-is-there-a-bett/10116#10116
% How to use:
% text...\footnote{\label{first}First footnote}
% text...\cref{first}
\crefformat{footnote}{#2\footnotemark[#1]#3}
%%%%%%%%%%%%%%%%%%%%%%%%%%%%%%%%%%%%%%%%%%%%%%%%%%%%%%%%%%%%%%%%%%%%%%%%%%%%
%%%%%%%%%%%%%%%%%%%%%%%%%%%%%%%%%%%%%%%%%%%%%%%%%%%%%%%%%%%%%%%%%%%%%%%%%%%%




%%%%%%%%%%%%%%%%%%%%%%%%%%%%%%%%%%%%%%%%%%%%%%%%%%%%%%%%%%%%%%%%%%%%%%%%%%%%
% https://tex.stackexchange.com/questions/234729/overfull-boxes-in-memoir-table-of-contents
% This is to fix the column width of the contents page, since if the numbers
% are too long, e.g. 11.20.21 (for the subsection), it causes an overflow
% with the text. Change the cft-bit as you see fit.
\setpnumwidth{3em}
\setrmarg{4em}
%\setlength{\cftpartnumwidth}{3em}
\setlength{\cftsectionnumwidth}{3em} % change here.
\setlength{\cftsubsectionnumwidth}{4em} % change here.


%%%%%%%%%%%%%%%%%%%%%%%%%%%%%%%%%%%%%%%%%%%%%%%%%%%%%%%%%%%%%%%%%%%%%%%%%%%%
%%% TABLE OF CONTENTS
%%%-------------------------------------------------------------------------
% Only parts, chapters and sections in the table of contents
\maxtocdepth{subsection} 
\settocdepth{subsection}
% Add a \par to the end of the TOC
\AtEndDocument{\addtocontents{toc}{\par}} 





