%%%% Stuff common to all latex documents.

% New versions of latex has deprecated \it and \tt with \itshape and
% \ttfamily. But some packages (such as lstlisting)  still use it. So we
% alias for compatibility
\renewcommand{\it}{\itshape}
\renewcommand{\tt}{\ttfamily}

%%%%%%%%%%%%%%%%%%%%%%%%%%%%%%%%%%%%%%%%%%%%%%%%%%%%%%%%%%%%%%%%%%%%%%%%%%%%
%%%%%%%%%%%%%%%%%%%%%%%%%%%%%%%%%%%%%%%%%%%%%%%%%%%%%%%%%%%%%%%%%%%%%%%%%%%%
%%%%%%%%%%%%%%%%%%%%%%%%%%%%%%%%%%%%%%%%%%%%%%%%%%%%%%%%%%%%%%%%%%%%%%%%%%%%
%%%%%%%%%%%%%%%%%%%%%%%%%%%%%%%%%%%%%%%%%%%%%%%%%%%%%%%%%%%%%%%%%%%%%%%%%%%%

%%% How to use scripts:
%%% Update mendeley references
%\immediate\write18{./Bin/update_mendeley_ref.sh}

%%% Generate the word count.
%\immediate\write18{./Bin/runtexcount.sh > ./Temp/texcountoutput.tex}

%%%%%%%%%%%%%%%%%%%%%%%%%%%%%%%%%%%%%%%%%%%%%%%%%%%%%%%%%%%%%%%%%%%%%%%%%%%%
%%%%%%%%%%%%%%%%%%%%%%%%%%%%%%%%%%%%%%%%%%%%%%%%%%%%%%%%%%%%%%%%%%%%%%%%%%%%
%%% PACKAGES %%%%%%%%%%%%%%%%%%%%%%%%%%%%%%%%%%%%%%%%%%%%%%%%%%%%%%%%%%%%%%%
%%%%%%%%%%%%%%%%%%%%%%%%%%%%%%%%%%%%%%%%%%%%%%%%%%%%%%%%%%%%%%%%%%%%%%%%%%%%
%%%%%%%%%%%%%%%%%%%%%%%%%%%%%%%%%%%%%%%%%%%%%%%%%%%%%%%%%%%%%%%%%%%%%%%%%%%%

%---------------------------------------------------------------------------
% Apply patches to other packages. Currently over required for mdframed
%---------------------------------------------------------------------------
\usepackage{xpatch}

%---------------------------------------------------------------------------
% Silence warnings
%---------------------------------------------------------------------------

% Silence warnings by hyper link, and mdframed
\usepackage{silence}

%\WarningFilter{latexfont}{Some font shapes were not available, defaults substituted}
\WarningFilter{latex}{Empty bibliography}

% The below is only for use when includeonly is used.
% latex gets confused and thinks there are multiple labels, but when
% we comment out includeonly{}, everything is fine.
% NOTE: Make sure to commen this out when we are defining new labels
% since we DO want to know if we have multiply-defined labels.
% UNCOMMENT THIS THIS THIS THIS THIS THIS
% UNCOMMENT THIS THIS THIS THIS THIS THIS
% UNCOMMENT THIS THIS THIS THIS THIS THIS
%\WarningFilter{latex}{Label}
%\WarningFilter{latex}{There were multiply-defined labels}

\WarningFilter{memoir}{You are using the caption package}
% We filter out this message since we use the caption package.
% the caption package is required to make the longtable captions work
% properly:
%Class memoir Warning: You are using the caption package with the memoir
%class. To prepare we will now reset all captioning macros and configurations
%to kernel defaults, and then let the caption package take over. Please
%remember to use the caption package interfaces in order to configure your
%captions. .

% Note: use this variable for array stretch
\newcommand{\arraystretchsize}{1.3}


%---------------------------------------------------------------------------
% Colours
%---------------------------------------------------------------------------
%\usepackage{color} % loaded by xcolor

% If you are using tikz or pstricks package you must declare the xcolor 
% package before that, otherwise it will not work.
% http://ctan.org/pkg/xcolor
\usepackage[usenames,dvipsnames,svgnames,x11names]{xcolor}

%---------------------------------------------------------------------------
% Float placements
%---------------------------------------------------------------------------
\usepackage{placeins}
% This is to use \FloatBarrier, to make latex flush all floats.

%---------------------------------------------------------------------------
% Tikz, pgfplots etc...
%---------------------------------------------------------------------------

%Tools for drawing Euclidean geometry
\usepackage{tkz-euclide}

\usepackage{pgfplots}
% Compatibility mode, because I have missing tikz labels or something.
\pgfplotsset{compat=1.9}

\usepackage{tikz}
\usetikzlibrary{calc}
\usetikzlibrary{matrix,arrows}
\usetikzlibrary{decorations.markings}
\tikzset{->-/.style={decoration={
  markings,
  mark=at position #1 with {\arrow{>}}},postaction={decorate}}}

%\tikzset{->-/.style={decoration={
%  markings,
%  mark=at position 0.25 and 0.9 step 0.25 with {\arrow{>}}},postaction={decorate}}}

%\tikzset{->>>>-/.style={decoration={
%  markings,
%  mark=at position 0.25 with {\arrow{>}}},postaction={decorate}},
%  mark=at position 0.5 with {\arrow{>}}},postaction={decorate}};}

% This is for the braces in the Knudsen diagram in Chapter 1
\usetikzlibrary{decorations.pathreplacing}

% This is for the flowchart in development of the BPF
\usetikzlibrary{shapes.geometric, arrows,chains}
\usetikzlibrary{positioning}


%---------------------------------------------------------------------------
% Frame around text.
%---------------------------------------------------------------------------
\usepackage[framemethod=TikZ]{mdframed}

% Apply the patch, this uses package xpatch: 
% source: http://tex.stackexchange.com/questions/162640/how-to-get-more-than-3-levels-of-nesting-in-mdframed-environments
\makeatletter
\xpatchcmd{\mdf@preenvsetting}
  {\mdf@envdepth >\tw@}
  {\mdf@envdepth >20}
  {}
  {}
\makeatother

% Ignore bad break warning
\WarningFilter{mdframed}{You got a bad break}

\makeatletter

\mdf@PackageWarning{You got a bad break\MessageBreak
  because the last split box is empty\MessageBreak
  You have to change the settings}

\makeatother


%\mdfdefinestyle{mdframed1}{
%  frametitle={}, % next we do the placement of the frames
%  skipabove=0pt,
%  skipbelow=0pt,
%  leftmargin=0pt,
%  rightmargin=0pt,
%  innerleftmargin=10pt,
%  innerrightmargin=10pt,
%  innertopmargin=5pt,
%  innerbottommargin=5pt, % next we control the line
%  linewidth=0.4pt,
%  innerlinewidth=0pt,
%  middlelinewidth=0.4pt,
%  outerlinewidth=0pt,
%  roundcorner=0pt,% Next we do the colours
%  linecolor=black,
%  innerlinecolor=black,
%  middlelinecolor=black,
%  outerlinecolor=black,}

% Use the styles with begin{mdframed}[style=mdframed1]
% Here as use the standard latex colors: black, red, green, blue, cyan,
% magenta, yellow
\newcommand{\mdfinnerrightmargin}{2pt}
\newcommand{\mdflinewidth}{1pt}

\mdfdefinestyle{mdframed1}{
  innerrightmargin=\mdfinnerrightmargin,
  linewidth=\mdflinewidth,
  linecolor=black}
\mdfdefinestyle{mdframed2}{
  innerrightmargin=\mdfinnerrightmargin,
  linewidth=\mdflinewidth,
  linecolor=red}
\mdfdefinestyle{mdframed3}{
  innerrightmargin=\mdfinnerrightmargin,
  linewidth=\mdflinewidth,
  linecolor=green}
\mdfdefinestyle{mdframed4}{
  innerrightmargin=\mdfinnerrightmargin,
  linewidth=\mdflinewidth,
  linecolor=blue}
\mdfdefinestyle{mdframed5}{
  innerrightmargin=\mdfinnerrightmargin,
  linewidth=\mdflinewidth,
  linecolor=cyan}
\mdfdefinestyle{mdframed6}{
  innerrightmargin=\mdfinnerrightmargin,
  linewidth=\mdflinewidth,
  linecolor=magenta}
\mdfdefinestyle{mdframed7}{
  innerrightmargin=\mdfinnerrightmargin,
  linewidth=\mdflinewidth,
  linecolor=yellow}

\mdfdefinestyle{mdfexercise}{
  roundcorner=8pt}

%%%%%%%%%%%%%%%%%%%%%%%%%%%%%%%%%%%%%%%%%%%%%%%%%%%
%% curly boxes with background colours
%%%%%%%%%%%%%%%%%%%%%%%%%%%%%%%%%%%%%%%%%%%%%%%%%%%
%\mdfdefinestyle{mdfcurlyblue}{
%  backgroundcolor=blue!20,
%%  shadow=true,
%  roundcorner=8pt}
%
%\mdfdefinestyle{mdfcurlyyellow}{
%  backgroundcolor=yellow!20,
%%  shadow=true,
%  roundcorner=8pt}

\mdfdefinestyle{mdfNOTE}{
  backgroundcolor=blue!30,
%  shadow=true,
  roundcorner=8pt}

\mdfdefinestyle{mdfKEYCONCEPT}{
  backgroundcolor=green!30,
%  shadow=true,
  roundcorner=8pt}

\mdfdefinestyle{mdfWARNING}{
  backgroundcolor=yellow!30,
%  shadow=true,
  roundcorner=8pt}

\mdfdefinestyle{mdfREMEMBER}{
  backgroundcolor=red!25,
  frametitlefont={\bfseries\color{red!65}},
  frametitlebackgroundcolor=DarkBlue,
  frametitlealignment=\center,
%  shadow=true,
  roundcorner=8pt}

\mdfdefinestyle{mdfRAYSNOTES}{
  backgroundcolor=blue!12,
  frametitlealignment=\center,
%  shadow=true,
  roundcorner=8pt}

\mdfdefinestyle{mdfOUTANS}{
  backgroundcolor=green!20,
%  shadow=true,
  roundcorner=8pt}

\mdfdefinestyle{mdfINANS}{
  backgroundcolor=yellow!20,
%  shadow=true,
  roundcorner=8pt}

%---------------------------------------------------------------------------
% url, hyperref, bookmark, etc...
%---------------------------------------------------------------------------

\WarningFilter{hyperref}{You have enabled option `breaklinks'.}

% url is loaded by hyperref, which is loaded by bookmark.

% The command \PassOptionsToPackage (used below) tells LaTeX to behave as 
% if its options were passed, when it finally loads a package. As you would 
% expect from its name, \PassOptionsToPackage can deal with a list of 
% options, just as you would have in the the options brackets of 
% \usepackage.

% Ordinarily, breaks are not allowed after “-” characters because this leads
% to confusion. (Is the “-” part of the address or just a hyphen?) The 
% package option “[hyphens]” allows breaks after explicit hyphen characters.
% The \url command will never ever hyphenate words.
% Also can use \path{}
\PassOptionsToPackage{hyphens,obeyspaces,spaces}{url}

% Allows link text to break across lines; since this cannot be accommodated 
% in PDF, it is only set true by default if the pdftex driver is used. This
% makes links on multiple lines into different PDF links to the same target.
\PassOptionsToPackage{breaklinks=true}{hyperref}

% http://tex.stackexchange.com/questions/167948/package-rerunfilecheck-warning-file-out-has-changed
% Or add \usepackage{bookmark}. Then a more modern implementation of the 
% bookmarks managing is used without .out file. The bookmarks are updated 
% earlier, thus in most cases only one LaTeX run is needed.
\usepackage{bookmark}

%% breakurl needs hyperref, but hyperref is loaded by bookmark!
\usepackage[anythingbreaks]{breakurl}

% Moved this into individual main.tex, since I need to set the pdf title,
% which is unique per project
%%% Internal links
%\hypersetup{
%colorlinks=true,
%linkcolor=blue,
%filecolor=magenta,      
%urlcolor=cyan,
%citecolor=cyan,
%%pdfborder={0 0 0},      % No borders around internal hyperlinks
%%pdftitle={RNO LeetCode},
%pdfauthor={Ray White} % author
%}
%\urlstyle{same}

%memhfixc  --  Adjustment for using hyperref in memoir documents
%Any recent version of hyperref will automatically load this package if it
%finds itself running under the memoir class.
%Otherwise, the package should simply be loaded (without options) after 
% hyperref has been loaded.
%\usepackage{memhfixc}


%---------------------------------------------------------------------------
% graphic, graphicx
%---------------------------------------------------------------------------

%% This is for includegraphics{}
% \usepackage{graphics} % Already loaded by graphicsx
% pdftex (default if compiling with pdflatex), if you are compiling with 
% pdftex to get a PDF that you will see with any PDF viewer.

% This gets an error, reason is below:
%\usepackage[pdftex]{graphicx}

% PDFLaTeX does not support EPS files. However, some modern LaTeX 
% distributions will try to automatically convert it to a PDF image, which 
% maybe this doesn't work in your case. It would be also possible that the 
% size information, namely the bounding box, in that EPS file is missing.
%
%If you are using latex, i.e. the DVI mode of LaTeX, which supports EPS then
%you should remove the incorrect pdftex option from the graphicx package. 
%The packages should be able to figure out the used driver by themselves, so
%you should avoid such options anyway. If you are using pdflatex then it 
%would be better to convert the EPS to a PDF file manually, e.g. using the 
%epstopdf tool (I'm not sure if it comes with the Windows version, but I 
%think so). You need to change the .eps extension to .pdf then, of course. 
%You can also drop the extension and LaTeX will look for files with the 
%given base name with all supported extensions, i.e. \includegraphics{file} 
%will use file.eps for latex and file.pdf (or file.png, file.jpg) for 
%pdflatex.

% TL;DR: Take out pdftex, and convert eps to pdf manually (although the
% engine should do it for us, if required)
\usepackage{graphicx}


%---------------------------------------------------------------------------
% fonts, symbols and encodings
%---------------------------------------------------------------------------

% I want to use sans serif
\usepackage{helvet}
\renewcommand{\familydefault}{\sfdefault}

%https://tex.stackexchange.com/questions/174021/typeset-whole-document-in-sans-serif-including-math-mode
%\usepackage{sansmath} % Enables turning on sans-serif math mode, and using other environments
%\sansmath % Enable sans-serif math for rest of document


%\usepackage{textcomp}
\usepackage[official]{eurosym}

\usepackage[utf8]{inputenc} % If utf8 encoding
%If not utf8 encoding, then this is probably the way to go
%\usepackage[lantin1]{inputenc} 
\usepackage[T1]{fontenc}    %

% Multilingual support for Plain TEX or LATEX
\usepackage[english]{babel} % English please

\usepackage[final]{microtype} % Less badboxes

% For bold texttt, usage: \texttt{\textbf{foo}}
\usepackage[scaled=0.85]{beramono}

%\usepackage{lmodern}% http://ctan.org/pkg/lm

%\usepackage{courier}

% \usepackage{kpfonts} %Font

% Note: difference between newcommand and newcommand*:

%\newcommand{\examplea}[1]

%\newcommand*{\exampleb}[1]

% Most of the time, \newcommand* is the best choice as you want the 
% error-checking that it provides. That is why examples given by experienced
% LaTeX users normally use this form, rather than just \newcommand
% So I will use newcommand*

% rr stands for rayray
% so rrbf is rayray's bold font, etc...

%\texttt[<color>]{<stuff>} has been redefined to take an optional <color> 
% (default is black), source: 
% http://tex.stackexchange.com/questions/36455/how-can-i-ensure-all-teletype-monospaced-text-is-in-a-certain-color
\let\oldtexttt\texttt% Store \texttt
% \texttt[<color>]{<stuff>}
\renewcommand{\texttt}[2][black]{\textcolor{#1}{\ttfamily #2}}



\newcommand*{\rrbf}[1]{\textbf{#1}}
\newcommand*{\rrbftt}[1]{\texttt{\textbf{#1}}}
\newcommand*{\rrit}[1]{\textit{#1}}
\newcommand*{\rrtt}[1]{\texttt{#1}}
\newcommand*{\rrem}[1]{\emph{#1}}

\newcommand*{\rrbfb}[1]{\textcolor{blue}{\textbf{#1}}}
\newcommand*{\rrbfbtt}[1]{\texttt[blue]{\textbf{#1}}}

%% Other shorthands

%% Here, again rr stands for rayray, but the t stands for text.
%% This means these should be used in text mode, rrt = rayray text mode.

%% Text backslash
\newcommand*{\rrtbslash}{\textbackslash{}}
%% Text less than
\newcommand*{\rrtlt}{\textless{}}
\newcommand*{\rrtltlt}{\textless{}\textless{}}
%% Text greather than
\newcommand*{\rrtgt}{\textgreater{}}
\newcommand*{\rrtgtgt}{\textgreater{}\textgreater{}}



\newcommand{\rcindent}[1]{\noindent\hspace*{2em}}

\newcommand{\definedtermsskip}{\bigskip}



%---------------------------------------------------------------------------
% Maths
%---------------------------------------------------------------------------

%http://tex.stackexchange.com/questions/32100/what-does-each-ams-package-do
%Most of the answer was extracted from the Introduction sections of the 
%documentation of amsmath and amsthm:
%\usepackage{amsmath}%AMS mathematical facilities for LATEX
\usepackage{mathtools} % mathtools loads/extends amsmath.
\usepackage{amsthm} % ams theorem
\usepackage{amssymb} % loads amsfonts

%https://tex.stackexchange.com/questions/232922/a-hack-for-getting-a-capital-weierstrass-p-in-order-to-represent-the-power-set
\usepackage{mathrsfs}
\usepackage[mathscr]{euscript}
\let\euscr\mathscr \let\mathscr\relax% just so we can load this and rsfs
\usepackage[scr]{rsfso}
\newcommand{\powerset}{\raisebox{.15\baselineskip}{\Large\ensuremath{\wp}}}
%Usage:
%$\mathcal{P}(X)$ 
%$\euscr{P}(X)$
%$\mathscr{P}(X)$
%$\powerset(X)$

% For boxes around equations
% https://tex.stackexchange.com/questions/20575/attractive-boxed-equations
% https://tex.stackexchange.com/questions/193242/how-to-make-a-box-around-an-equation-in-align-environment
\usepackage{empheq}
\usepackage[most]{tcolorbox}
%\begin{empheq}[box=\tcbhighmath]{equation*}
%    c_i = \langle\psi|\phi\rangle
%\end{empheq}



%1)amsmath provides miscellaneous enhancements for improving the information 
%structure and printed output of documents containing mathematical formulas.
%Some of the features provided by this package are:
%  The \DeclareMathOperator command (through the auxiliary package amsopn) 
%  to define new "operator name" commands analogous to \sin and \lim, 
%  including proper side spacing and automatic selection of the correct font
%  style and size (even when used in sub- or superscripts).
%
%  Multiple substitutes for the eqnarray environment to make various kinds 
%  of equation arrangements easier to write.
%
%  Equation numbers automatically adjust up or down to avoid overprinting on
%  the equation contents (unlike eqnarray).
%
%  Spacing around equals signs matches the normal spacing in the equation 
%  environment (unlike eqnarray).
%
%  A way to produce multiline subscripts as are often used with summation or
%  product symbols.
%
%  An easy way to substitute a variant equation number for a given equation
%  instead of the automatically supplied number.
%
%  An easy way to produce subordinate equation numbers of the form (1.3a) 
%  (1.3b) (1.3c) for selected groups of equations.
%
%  The \text command (through the auxiliary package amstext) for typesetting
%  a fragment of text inside a display.

%2)amsthm helps to define theorem-like structures; the introduction to the 
%documentation gives a nice concise description of the package:
%  The amsthm package provides an enhanced version of LaTeX's  \newtheorem 
%  command for defining theorem-like environments. The enhanced \newtheorem
%  recognizes a \theoremstyle specification (as in Mittelbach's theorem 
%  package) and has a * form for defining unnumbered environments. The
%  amsthm package also defines a proof environment that automatically adds a
%  QED symbol at the end. AMS document classes incorporate the amsthm 
%  package, so everything described here applies to them as well.
%
%  If the amsthm package is used with a non-AMS document class and with the
%  amsmath package, amsthm must be loaded after amsmath, not before.

%3)amssymb provides an extended symbol collection. For example, after 
%loading amssymb you have the following additional binary relation symbols:
%\barwedge, \boxdot, \boxminus, \boxplus, \boxtimes, \Cap, \Cup (and many 
%more), the arrow \leadsto, and some other symbols such as \Box and 
%\Diamond. Another useful feature is the \mathbb command to produce 
%blackboard bold characters.

% Since amssymb internally loads amsfonts, it's enough to load the former.


%---------------------------------------------------------------------------
% New column type, uses array package
% http://tex.stackexchange.com/questions/154958/vertically-centered-and-right-left-center-horizontal-alignment-in-tabular
%---------------------------------------------------------------------------
\usepackage{array}
\newcolumntype{C}[1]{>{\centering\arraybackslash}m{#1}}   %% centered
\newcolumntype{R}[1]{>{\raggedleft\arraybackslash}m{#1}}  %% right aligned
% Use:
%\begin{tabular}{|m{5em}| C{5em}| R{5em}|}
%left & center & right
%\end{tabular}
% produces:
% | left    |  center  |    right |


%%%%%%%%%%%%%%%%%%%%%%%%%%%%%%%%%%%%%%%%%%%%%%%%%%%%%%%%%%%%%%%%%%%%%%%%%%%%
%%%%%%%%%%%%%%%%%%%%%%%%%%%%%%%%%%%%%%%%%%%%%%%%%%%%%%%%%%%%%%%%%%%%%%%%%%%%

%---------------------------------------------------------------------------
% Other packages
%---------------------------------------------------------------------------

\usepackage{verbatim} % extension of verbatim

\usepackage{fancyvrb}

% Stop footnote from running across pages:
% http://tex.stackexchange.com/questions/32208/footnote-runs-onto-second-page
%\interfootnotelinepenalty=\@M
% http://www.tex.ac.uk/FAQ-splitfoot.html
\interfootnotelinepenalty=10000

% Use verbatim in footnotes.
\VerbatimFootnotes
\fvset{fontfamily=courier,
%       fontseries=b,
       fontsize=\small,
       xleftmargin=0.05cm,
       frame=leftline,}
%       showspaces=true}
%% fancyvrb allows \begin{Verbatim}[samepage=true],
%% fancyvrb allows \begin{Verbatim}[xleftmargin=.5in],

\usepackage{upgreek} % For uptau, used for tangent vectors
\usepackage{bm} % Bold maths symbols \bm{}
\usepackage{xifthen}% provides \isempty test, this is used for optional arg.

%https://tex.stackexchange.com/questions/287657/learning-to-use-xparse
%https://tex.stackexchange.com/questions/33749/xparse-define-new-command-with-multiple-optional-parameters
%https://tex.stackexchange.com/questions/60893/new-command-with-optional-argument-being-first-argument
% For using \NewDocumentCommand
\usepackage{xparse}

%---------------------------------------------------------------------------
% Code listing
%---------------------------------------------------------------------------

\usepackage{listings}

%\lstMakeShortInline[columns=fixed]|

\lstdefinelanguage{newcpp}{
language=C++,
morekeywords={auto,concept,constexpr,decltype,final,override,requires,
              concept},
}

% Define a new colour
\definecolor{light-gray}{gray}{0.92}

\lstdefinelanguage{pseudocode}{
%language=C++,
comment=[l]{//},
morecomment=[s]{/*}{*/},
morecomment=[s][\color{blue}]{/*+}{*/},
morecomment=[s][\color{red}]{/*-}{*/},
morekeywords={for, For, to, downto, input, output, return, datatype,
  function, in, if, If, else, Else, foreach, while, While, begin, end, do,
  procedure},
}

\lstdefinestyle{pseudostyle}{
  captionpos=b,
  numbers=left, 
  stepnumber=1, 
  numberstyle=\tiny,
  frame=leftline,
%  framexleftmargin=5mm,
  xleftmargin=5mm,
  showstringspaces=false, 
  language=pseudocode,
  mathescape=true,
  backgroundcolor=\color{light-gray},
  basicstyle=\linespread{0.8}\ttfamily\small,
%  keywordstyle=\color{blue}\ttfamily,
  keywordstyle=\color{black}\bfseries\em,
  stringstyle=\color{magenta}\ttfamily,
  commentstyle=\color{darkgray}\ttfamily,
  morecomment=[l][\color{magenta}]{\#},
  moredelim=**[is][\itshape]{@it}{@}, % this keeps keyword syntax
  moredelim=**[is][\bfseries]{@bf}{@}, % this keeps keyword syntax
  moredelim=**[is][\bfseries\itshape]{@ibf}{@}, % this keeps keyword syntax
  moredelim=**[is][\color{red!70!black}]{@R}{@},
%  moredelim=[is][\color{red!70!black}]{@@}{@@},
}

\lstdefinestyle{raygeneric}{
  captionpos=b,
  numbers=none, 
  stepnumber=1, 
  numberstyle=\tiny,
  frame=leftline,
%  framexleftmargin=5mm,
  xleftmargin=5mm,
  showstringspaces=false, 
  mathescape=true,
  backgroundcolor=\color{light-gray},
  basicstyle=\linespread{0.8}\ttfamily\small,
%  keywordstyle=\color{blue}\ttfamily,
  keywordstyle=\color{black}\bfseries\em,
  stringstyle=\color{magenta}\ttfamily,
  commentstyle=\color{darkgray}\ttfamily,
  morecomment=[l][\color{magenta}]{\#},
  moredelim=**[is][\itshape]{@}{@}, % this keeps keyword syntax
  moredelim=**[is][\bfseries]{`}{`}, % this keeps keyword syntax
  moredelim=**[is][\bfseries\itshape]{¬}{¬}, % this keeps keyword syntax
  moredelim=**[is][\color{red!70!black}]{@@}{@@},
%  moredelim=[is][\color{red!70!black}]{@@}{@@},
}


% Used for C++
\lstdefinestyle{raycpp}{
  captionpos=b,
%  numbers=left, 
  numbers=none, 
  stepnumber=2, 
  frame=single,
  showstringspaces=false, 
  language=newcpp,
  backgroundcolor=\color{light-gray},
  basicstyle=\linespread{0.8}\ttfamily\footnotesize,
  keywordstyle=\color{blue}\ttfamily,
  stringstyle=\color{red}\ttfamily,
  commentstyle=\color{darkgray}\ttfamily,
  morecomment=[l][\color{magenta}]{\#},
}

% Python
\lstdefinestyle{raypython}{
  captionpos=b,
  %  numbers=left, 
  numbers=none, 
  stepnumber=2, 
  frame=leftline,
%  framexleftmargin=5mm,
  xleftmargin=5mm,
  showstringspaces=false, 
  language=Python,
  mathescape=true,
  backgroundcolor=\color{light-gray},
  basicstyle=\linespread{0.8}\ttfamily\small,
  keywordstyle=\color{blue}\ttfamily,
  stringstyle=\color{magenta}\ttfamily,
  commentstyle=\color{darkgray}\ttfamily,
  morecomment=[l][\color{magenta}]{\#},
%  moredelim=**[is][\color{red}]{@}{@}, % this keeps keyword syntax
  moredelim=[is][\color{red!70!black}]{@}{@},
}



%% Used for bash scripts
\lstdefinestyle{raybash}{
        captionpos=b,
%        numbers=left, 
        numbers=none, 
        stepnumber=2, 
        frame=single,
        showstringspaces=false, 
        language=bash,
        basicstyle=\linespread{0.8}\ttfamily\footnotesize,
        keywordstyle=\color{blue}\ttfamily,
        stringstyle=\color{red}\ttfamily,
        commentstyle=\color{gray}\ttfamily,
        morecomment=[l][\color{magenta}]{\#},}

\lstdefinestyle{raybashsmall}{
        captionpos=b,
%        numbers=left, 
        numbers=none, 
        stepnumber=2, 
        frame=single,
        showstringspaces=false, 
        language=bash,
        basicstyle=\linespread{0.8}\ttfamily\scriptsize,
        keywordstyle=\color{blue}\ttfamily,
        stringstyle=\color{red}\ttfamily,
        commentstyle=\color{gray}\ttfamily,
        morecomment=[l][\color{magenta}]{\#},}

% Note: Use this for captions:
%\begin{lstlisting}[style=raycppsmall,
%  caption={Grade Clustering Program},captionpos=b,
%  label={lstC3N1Name}]

\newcommand{\lred}[1]{\textcolor{red}{\mathtt{#1}}}
\newcommand{\lredt}[1]{\textcolor{red}{\mathtt{\text{#1}}}}
\newcommand{\lr}[1]{\textcolor{red}{\mathtt{#1}}}
\newcommand{\lrt}[1]{\textcolor{red}{\mathtt{\text{#1}}}}

% Used for C++ snippets of code, unimportant snippets.
% For color text: https://tex.stackexchange.com/questions/115547/textcolor-within-lstlisting
%\definecolor{darkred}{red}{0.5}
\lstdefinestyle{raycppsnippet}{
  captionpos=b,
  %  numbers=left, 
  numbers=none, 
  stepnumber=2, 
  frame=leftline,
%  framexleftmargin=5mm,
  xleftmargin=5mm,
  showstringspaces=false, 
  language=newcpp,
  mathescape=true,
  backgroundcolor=\color{light-gray},
  basicstyle=\linespread{0.8}\color{black}\ttfamily\small,
  keywordstyle=\color{blue}\ttfamily,
  stringstyle=\color{magenta}\ttfamily,
  commentstyle=\color{darkgray}\ttfamily,
  morecomment=[l][\color{magenta}]{\#},
%  moredelim=**[is][\color{red}]{@}{@}, % this keeps keyword syntax
  moredelim=[is][\color{red!70!black}]{@}{@},
}

\lstdefinestyle{raycppcheight}{
  captionpos=b,
  %  numbers=left, 
  numbers=none, 
  stepnumber=2, 
  frame=leftline,
%  framexleftmargin=5mm,
  xleftmargin=5mm,
  showstringspaces=false, 
  language=newcpp,
  mathescape=true,
  backgroundcolor=\color{light-gray},
  basicstyle=\linespread{0.8}\ttfamily\small,
  keywordstyle=\color{blue}\ttfamily,
  stringstyle=\color{magenta}\ttfamily,
  commentstyle=\color{darkgray}\ttfamily,
  morecomment=[l][\color{magenta}]{\#},
  moredelim=**[is][\itshape]{@}{@}, % this keeps keyword syntax
  moredelim=**[is][\bfseries]{`}{`}, % this keeps keyword syntax
  moredelim=**[is][\bfseries\itshape]{¬}{¬}, % this keeps keyword syntax
%  moredelim=[is][\color{red!70!black}]{@}{@},
}

% Use this, this is what I use. Shortcut is:
% lstpp{TAB}
\lstdefinestyle{raycppnewsnippet}{
  captionpos=b,
  %  numbers=left, 
  numbers=none, 
  stepnumber=2, 
  frame=leftline,
%  framexleftmargin=5mm,
  xleftmargin=5mm,
  showstringspaces=false, 
  language=newcpp,
  mathescape=true,
  backgroundcolor=\color{light-gray},
  basicstyle=\linespread{0.8}\ttfamily\small,
  keywordstyle=\color{blue}\ttfamily,
  stringstyle=\color{magenta}\ttfamily,
  commentstyle=\color{darkgray}\ttfamily,
  morecomment=[l][\color{magenta}]{\#},
  moredelim=**[is][\itshape]{@it}{@}, % this keeps keyword syntax
  moredelim=**[is][\bfseries]{@bf}{@}, % this keeps keyword syntax
  moredelim=**[is][\bfseries\itshape]{@ibf}{@}, % this keeps keyword syntax
  moredelim=**[is][\color{red!70!black}]{@R}{@},
%  moredelim=[is][\color{red!70!black}]{@@}{@@},
}

% Used for C++ code displaying, displaying important code 
% i.e. for the first time.
\lstdefinestyle{raycppdisplay}{
  captionpos=b,
  %  numbers=left, 
  numbers=none, 
  stepnumber=2, 
  frame=leftline,
%  framexleftmargin=5mm,
  xleftmargin=5mm,
  showstringspaces=false, 
  language=newcpp,
  mathescape=true,
  backgroundcolor=\color{light-gray},
  basicstyle=\linespread{1}\normalsize\bfseries\sffamily,
  keywordstyle=\color{blue}\normalsize\bfseries\sffamily,
  stringstyle=\color{red}\normalsize\bfseries\sffamily,
  commentstyle=\color{darkgray}\normalfont,
  morecomment=[l][\color{magenta}]{\#},
}



 % Used for C++ (small)
\lstdefinestyle{raycppsmall}{
  captionpos=b,
  %  numbers=left, 
  numbers=none, 
  stepnumber=2, 
  frame=single,
  showstringspaces=false, 
  language=newcpp,
  backgroundcolor=\color{light-gray},
  basicstyle=\linespread{0.8}\ttfamily\scriptsize,
  keywordstyle=\color{blue}\ttfamily,
  stringstyle=\color{red}\ttfamily,
  commentstyle=\color{darkgray}\ttfamily,
  morecomment=[l][\color{magenta}]{\#},
}

% Usage: \lstcppsmall{tagstart}{tagend}{inputfile}
\DeclareRobustCommand{\lstcppsmall}[3]{
\lstinputlisting[style=raycppsmall,
                 numbers=none,
                 rangeprefix=//\ ,% // startmarker //
                 rangesuffix=\ //,% // endmarker //
                 includerangemarker=false,
                 linerange=#1-#2]
{#3}}

\lstdefinestyle{raygensmall}{
        captionpos=b,
%        numbers=left, 
        numbers=none, 
        stepnumber=2, 
        frame=single,
        showstringspaces=false, 
%        language=bash,
        basicstyle=\linespread{0.8}\ttfamily\scriptsize,
        keywordstyle=\color{blue}\ttfamily,
        stringstyle=\color{red}\ttfamily,
        commentstyle=\color{gray}\ttfamily,
        morecomment=[l][\color{magenta}]{\#},}

% Used for input/output
\lstdefinestyle{rayio}{
  captionpos=b,
  %  numbers=left, 
  numbers=none, 
  stepnumber=2, 
  frame=leftline,
%  framexleftmargin=5mm,
  xleftmargin=5mm,
  showstringspaces=false, 
%  language=bash,
%  backgroundcolor=\color{light-gray},
  basicstyle=\linespread{1}\sffamily\footnotesize,
  keywordstyle=\color{blue}\sffamily,
  stringstyle=\color{red}\sffamily,
  commentstyle=\color{darkgray}\sffamily,
  morecomment=[l][\color{magenta}]{\#},
}
     

%%%%%%%%%%%%%%%%%%%%%%%%%%%%%%%%%%%%%%%%%%%%%%%%%%%%%%%%%%%%%%%%%%%%%%%%%%%%
%%%%%%%%%%%%%%%%%%%%%%%%%%%%%%%%%%%%%%%%%%%%%%%%%%%%%%%%%%%%%%%%%%%%%%%%%%%%

% cpp11 box
\newcommand*{\cppll}[1]{\colorbox{yellow!75}{\framebox{\strut C++ 11}}}


% Core box (person studying)
\newcommand*{\cppcore}[1]{\colorbox{yellow!75}{\framebox{\strut Core}}}

% Detail box (magnifying class)
\newcommand*{\cppdet}[1]{\colorbox{yellow!75}{\framebox{\strut Detailed}}}

%  Advanced (stack of books)
\newcommand*{\cppadv}[1]{\colorbox{yellow!75}{\framebox{\strut Advance}}}

%%%%%%%%%%%%%%%%%%%%%%%%%%%%%%%%%%%%%%%%%%%%%%%%%%%%%%%%%%%%%%%%%%%%%%%%%%%%
%%%%%%%%%%%%%%%%%%%%%%%%%%%%%%%%%%%%%%%%%%%%%%%%%%%%%%%%%%%%%%%%%%%%%%%%%%%%


%%%%%%%%%%%%%%%%%%%%%%%%%%%%%%%%%%%%%%%%%%%%%%%%%%%%%%%%%%%%%%%%%%%%%%%%%%%%
%%%%%%%%%%%%%%%%%%%%%%%%%%%%%%%%%%%%%%%%%%%%%%%%%%%%%%%%%%%%%%%%%%%%%%%%%%%%
%%%%%%%%%%%%%%%%%%%%%%%%%%%%%%%%%%%%%%%%%%%%%%%%%%%%%%%%%%%%%%%%%%%%%%%%%%%%
%%%%%%%%%%%%%%%%%%%%%%%%%%%%%%%%%%%%%%%%%%%%%%%%%%%%%%%%%%%%%%%%%%%%%%%%%%%%


% For psuedo code
\usepackage[linesnumbered,ruled,vlined,algochapter]{algorithm2e}
\usepackage{algpseudocode} % To use algorithmicx


%https://tex.stackexchange.com/questions/153646/algorithm2e-disabling-line-numbers-for-specific-lines
\let\oldnl\nl% Store \nl in \oldnl
\newcommand{\nonl}{\renewcommand{\nl}{\let\nl\oldnl}}% Remove line number for one line

%https://tex.stackexchange.com/questions/271661/algorithm-title-which-is-unnumbered-is-not-in-place
\makeatletter
\newcommand{\RemoveAlgoNumber}{\renewcommand{\fnum@algocf}{\AlCapSty{\AlCapFnt\algorithmcfname}}}
\newcommand{\RevertAlgoNumber}{\algocf@resetfnum}
\makeatother
%\RemoveAlgoNumber which removes \thealgocf from the caption printing; and
%\RevertAlgoNumber which reverts the removal.
% Usage:
% \RemoveAlgoNumber
% algo here
% \RevertAlgoNumber

%https://tex.stackexchange.com/questions/212301/do-while-loop-in-algorithm2e
\SetKwRepeat{Do}{do}{while}
%The above now allows you to use
%\Do{<end condition>}{<stuff>}

\SetKw{KwDownto}{downto}
\SetKw{KwError}{error}



%%%%%%%%%%%%%%%%%%%%%%%%%%%%%%%%%%%%%%%%%%%%%%%%%%%%%%%%%%%%%%%%%%%%%%%%%%%%
%%%%%%%%%%%%%%%%%%%%%%%%%%%%%%%%%%%%%%%%%%%%%%%%%%%%%%%%%%%%%%%%%%%%%%%%%%%%
%%%%%%%%%%%%%%%%%%%%%%%%%%%%%%%%%%%%%%%%%%%%%%%%%%%%%%%%%%%%%%%%%%%%%%%%%%%%
%%%%%%%%%%%%%%%%%%%%%%%%%%%%%%%%%%%%%%%%%%%%%%%%%%%%%%%%%%%%%%%%%%%%%%%%%%%%







%---------------------------------------------------------------------------
% biblatex with biber backend.
%---------------------------------------------------------------------------

\usepackage{csquotes}

% biblatex setup
\usepackage[backend=biber,
  citestyle=numeric,
  isbn=false,
  doi=false,
  url=true,% for websites, see below for removing it from papers etc.
  backref=true, % print out the page number of reference
]{biblatex}



%% biblatex setup
%\usepackage[backend=biber,
%  citestyle=numeric,
%  firstinits=true, % Only print initials of first names
%  isbn=false,
%  doi=false,
%  url=true,% for websites, see below for removing it from papers etc.
%  backref=true, % print out the page number of reference
%]{biblatex}


% clear urls for papers, proceedings, books
\AtEveryBibitem{%
  \ifentrytype{article}{%
    \clearfield{url}%
    \clearfield{urldate}%
    \clearfield{month}%
  }
  {}% no "else" operation
  %
  \ifentrytype{inproceedings}{%
    \clearfield{url}%
    \clearfield{urldate}%
    \clearfield{month}%
  }
  {}% no "else" operation
  % 
  \ifentrytype{book}{%
    \clearfield{url}%
    \clearfield{urldate}%
    \clearfield{month}%
  }
  {}% no "else" operation
  \ifentrytype{phdthesis}{%
    \clearfield{url}%
    \clearfield{urldate}%
    \clearfield{month}%
  }
  {}% no "else" operation
} % AtEveryBibitem

\renewbibmacro{in:}{%
  \ifentrytype{article}{}{\printtext{\bibstring{in}\intitlepunct}}}

% makes volume of journal bold and adds colon
\DeclareFieldFormat[article]{volume}{\textbf{#1}\addcolon\space}
% removes pagination (p./pp.) before page numbers
\DeclareFieldFormat{pages}{#1}

% Use BibTeX key as the cite key
% https://tex.stackexchange.com/questions/8428/use-bibtex-key-as-the-cite-key
\DeclareFieldFormat{labelalpha}{\thefield{entrykey}}
\DeclareFieldFormat{extraalpha}{}

\DefineBibliographyStrings{english}{%
    backrefpage  = {see p.}, % for single page number
    backrefpages = {see pp.} % for multiple page numbers
}

%%%%%%%%%%%%%%%%%%%%%%%%%%%%%%%%%%%%%%%%%%%%%%%%%%%%%%%%%%%%%%%%%%%%%%%%%%%%
%%%%%%%%%%%%%%%%%%%%%%%%%%%%%%%%%%%%%%%%%%%%%%%%%%%%%%%%%%%%%%%%%%%%%%%%%%%%

%---------------------------------------------------------------------------
% enumerate number per section
%---------------------------------------------------------------------------

\usepackage{enumitem}
% Note: This also allows us to do something like this:
%\begin{enumerate}[label=\FOO*]
% \alph* = a, \Alph* = A, \roman* = i, ii, etc...
% Put in parenthesis for parenthesis. I.e.
% (\alph*) = (a)
% Label: \alph, \Alph, \arabic, \roman and \Roman

\setenumerate[1]{label=\thesection.\arabic*.}
\setenumerate[2]{label*=\arabic*.}


%---------------------------------------------------------------------------
% Other packages, maybe not used.
%---------------------------------------------------------------------------

%this is obsolete, do not use it if someone suggests it.
%\usepackage{subfigure}

% sidecap  --  Typeset captions sideways
%\usepackage{sidecap}






%%%%%%%%%%%%%%%%%%%%%%%%%%%%%%%%%%%%%%%%%%%%%%%%%%%%%%%%%%%%%%%%%%%%%%%%%%%%
%%%%%%%%%%%%%%%%%%%%%%%%%%%%%%%%%%%%%%%%%%%%%%%%%%%%%%%%%%%%%%%%%%%%%%%%%%%%
%
% Mathematic symbols
%
%%%%%%%%%%%%%%%%%%%%%%%%%%%%%%%%%%%%%%%%%%%%%%%%%%%%%%%%%%%%%%%%%%%%%%%%%%%%
%%%%%%%%%%%%%%%%%%%%%%%%%%%%%%%%%%%%%%%%%%%%%%%%%%%%%%%%%%%%%%%%%%%%%%%%%%%%

% from mathtools, 
% floor and ceiling, usage:
% \floor*{ foo }
% and the unstarred version takes an optional argument that can be 
% \big, \Big, etc
% \abs[\Bigg]{\frac{a}{b}}
%% NOTE: for \brace - use \set
\DeclarePairedDelimiter\ceil{\lceil}{\rceil}
\DeclarePairedDelimiter\floor{\lfloor}{\rfloor}
\DeclarePairedDelimiter\braket{\langle}{\rangle} % < >
\DeclarePairedDelimiter\abs{\lvert}{\rvert}      % | |
\DeclarePairedDelimiter\norm{\lVert}{\rVert}     %|| ||
\DeclarePairedDelimiter\paren{\lparen}{\rparen}  % ( )
\DeclarePairedDelimiter\brce{\lbrace}{\rbrace}   % { }
\DeclarePairedDelimiter\brck{\lbrack}{\rbrack}   % [ ]
\DeclarePairedDelimiter\Brck{\lBrack}{\rBrack}   % [[ ]]

%%%%%%%%%%%%%%%%%%%%%%%%%%%%%%%%%%%%%%%%%%%%%%%%%%%%%%%%
%% Auxiliary symbols, like conjuate

% The below monstrocity is for a hacked widebar, used for \conj
% See https://tex.stackexchange.com/questions/16337/can-i-get-a-widebar-without-using-the-mathabx-package/60253#60253
%\makeatletter
%\let\save@mathaccent\mathaccent
%\newcommand*\if@single[3]{%
%  \setbox0\hbox{${\mathaccent"0362{#1}}^H$}%
%  \setbox2\hbox{${\mathaccent"0362{\kern0pt#1}}^H$}%
%  \ifdim\ht0=\ht2 #3\else #2\fi
%  }
%%The bar will be moved to the right by a half of \macc@kerna, which is computed by amsmath:
%\newcommand*\rel@kern[1]{\kern#1\dimexpr\macc@kerna}
%%If there's a superscript following the bar, then no negative kern may follow the bar;
%%an additional {} makes sure that the superscript is high enough in this case:
%\newcommand*\widebar[1]{\@ifnextchar^{{\wide@bar{#1}{0}}}{\wide@bar{#1}{1}}}
%%Use a separate algorithm for single symbols:
%\newcommand*\wide@bar[2]{\if@single{#1}{\wide@bar@{#1}{#2}{1}}{\wide@bar@{#1}{#2}{2}}}
%\newcommand*\wide@bar@[3]{%
%  \begingroup
%  \def\mathaccent##1##2{%
%%Enable nesting of accents:
%    \let\mathaccent\save@mathaccent
%%If there's more than a single symbol, use the first character instead (see below):
%    \if#32 \let\macc@nucleus\first@char \fi
%%Determine the italic correction:
%    \setbox\z@\hbox{$\macc@style{\macc@nucleus}_{}$}%
%    \setbox\tw@\hbox{$\macc@style{\macc@nucleus}{}_{}$}%
%    \dimen@\wd\tw@
%    \advance\dimen@-\wd\z@
%%Now \dimen@ is the italic correction of the symbol.
%    \divide\dimen@ 3
%    \@tempdima\wd\tw@
%    \advance\@tempdima-\scriptspace
%%Now \@tempdima is the width of the symbol.
%    \divide\@tempdima 10
%    \advance\dimen@-\@tempdima
%%Now \dimen@ = (italic correction / 3) - (Breite / 10)
%    \ifdim\dimen@>\z@ \dimen@0pt\fi
%%The bar will be shortened in the case \dimen@<0 !
%    \rel@kern{0.6}\kern-\dimen@
%    \if#31
%      \overline{\rel@kern{-0.6}\kern\dimen@\macc@nucleus\rel@kern{0.4}\kern\dimen@}%
%      \advance\dimen@0.4\dimexpr\macc@kerna
%%Place the combined final kern (-\dimen@) if it is >0 or if a superscript follows:
%      \let\final@kern#2%
%      \ifdim\dimen@<\z@ \let\final@kern1\fi
%      \if\final@kern1 \kern-\dimen@\fi
%    \else
%      \overline{\rel@kern{-0.6}\kern\dimen@#1}%
%    \fi
%  }%
%  \macc@depth\@ne
%  \let\math@bgroup\@empty \let\math@egroup\macc@set@skewchar
%  \mathsurround\z@ \frozen@everymath{\mathgroup\macc@group\relax}%
%  \macc@set@skewchar\relax
%  \let\mathaccentV\macc@nested@a
%%The following initialises \macc@kerna and calls \mathaccent:
%  \if#31
%    \macc@nested@a\relax111{#1}%
%  \else
%%If the argument consists of more than one symbol, and if the first token is
%%a letter, use that letter for the computations:
%    \def\gobble@till@marker##1\endmarker{}%
%    \futurelet\first@char\gobble@till@marker#1\endmarker
%    \ifcat\noexpand\first@char A\else
%      \def\first@char{}%
%    \fi
%    \macc@nested@a\relax111{\first@char}%
%  \fi
%  \endgroup
%}
%\makeatother
%\newcommand{\conj}[1]{\widebar{#1}}
\newcommand{\conj}[1]{\overline{#1}} %conjugate
\newcommand{\rrT}{\intercal} %transpose
\newcommand{\rrInv}{{-1}} %inverse
\newcommand{\rrxor}{\textasciicircum} % XOR, ^

\DeclareMathOperator{\diag}{diag} % diag
\DeclareMathOperator{\lcm}{lcm} % Least common multiple

%%%%%%%%%% This is for variables
% https://tex.stackexchange.com/questions/98149/to-imply-a-dash-and-not-minus-sign-in-math-mode
% This is so that hyphen will get interpreted as hyphen in mathmode, not as
% minus sign.
% Non-italic version
\newcommand{\varA}[1]{{\operatorname{#1}}}
% Italic version
\newcommand{\varB}[1]{{\ensuremath{\operatorname{\mathit{#1}}}}}

%%%%%%%%%%%%%%%%%%%%%%%%%%%%%%%%%%%%%%%%%%%%%%%%%%%%%%%%

% Use the set with \given, e.g.
%$\set{ x \given x > 0 }$
%$\set[\big]{ x \given x > 0 }$
%$\set[\Big]{ x \given x > 0 }$
%$\set[\bigg]{ x \given x > 0 }$
%$\set[\Bigg]{ x \given x > 0 }$
%$\set*{ x \given x > 0 \rule{0cm}{2cm} }$ % \rule is just here so you can see that it autostretches
\DeclarePairedDelimiterX\set[1]\lbrace\rbrace{\def\given{\;\delimsize\vert\;}#1}

% Probability, we use \Pr\set*{} to get Pr{...}
% Usage: \Prr[optional: power]{mandatory X}
% o = optional, m = mandatory
\NewDocumentCommand{\Prr}{o m}{%
  \IfValueTF{#1}{%
    \Pr^{#1}\set*{#2}%
  }{%
    \Pr\set*{#2}%
  }%
}

% Usage: \Err[optional: power]{mandatory X}
\DeclareMathOperator{\Expct}{E}
% o = optional, m = mandatory
\NewDocumentCommand{\Err}{o m}{%
  \IfValueTF{#1}{%
    \Expct^{#1}\brck*{#2}%
  }{%
    \Expct\brck*{#2}%
  }%
}

% Usage: \Vrr[optional: power]{mandatory X}
\DeclareMathOperator{\Var}{Var}
% o = optional, m = mandatory
\NewDocumentCommand{\Vrr}{o m}{%
  \IfValueTF{#1}{%
    \Var^{#1}\brck*{#2}%
  }{%
    \Var\brck*{#2}%
  }%
}

% Indicator vairable, used in ch5 CLRS
% Usage: \Irr[optional: power]{mandatory X}
\DeclareMathOperator{\Indct}{I}
% o = optional, m = mandatory
\NewDocumentCommand{\Irr}{o m}{%
  \IfValueTF{#1}{%
    \Indct^{#1}\set*{#2}%
  }{%
    \Indct\set*{#2}%
  }%
}

% Black-Height, used in ch13 CLRS.
% Usage: \Bh[optional: power]{mandatory X}
\DeclareMathOperator{\RRWBh}{bh}
% o = optional, m = mandatory
\NewDocumentCommand{\Bh}{o m}{%
  \IfValueTF{#1}{%
    \RRWBh^{#1}\paren*{#2}%
  }{%
    \RRWBh\paren*{#2}%
  }%
}




%%%%%%%%%%%%%%%%%%%%%%%%%%%%%%%%%%%%%%%%%%%%%%%%%%%%%%%%%%%%%%%%%%%%%%%%%%%%
% Usage: \Reach[optional: via node]
% o = optional, m = mandatory
\NewDocumentCommand{\Reach}{o}{
  \IfValueTF{#1}{
    \overset{#1}{\leadsto}
  }{
    \leadsto
  }
}


% https://tex.stackexchange.com/questions/5502/how-to-get-a-mid-binary-relation-that-grows
% usage: \remiddle|
% NOTE: use \given instead
%\newcommand{\relmiddle}[1]{\mathrel{}\middle#1\mathrel{}}

% big-o notation for complexity
% Also tests for empty, so I can use \bigo{} and it doesn't provide
% brackets.
\DeclareRobustCommand{\comBigOh}[1]{
\ifthenelse{\equal{#1}{}}{\mathcal{O}}
                         {\mathcal{O}\left(#1\right)}}

\DeclareRobustCommand{\comLittleOh}[1]{
\ifthenelse{\equal{#1}{}}{o}
                         {o\left(#1\right)}}
% theta- for worse case complexity.
% Okay, this is not worse case. I should call these comTheta and such.
\DeclareRobustCommand{\comTheta}[1]{
\ifthenelse{\equal{#1}{}}{\Theta}
                         {\Theta\left(#1\right)}}

\DeclareRobustCommand{\comBigOmega}[1]{
\ifthenelse{\equal{#1}{}}{\Omega}
                         {\Omega\left(#1\right)}}
\DeclareRobustCommand{\comLittleOmega}[1]{
\ifthenelse{\equal{#1}{}}{\omega}
                         {\omega\left(#1\right)}}


%% Set notation %%%%%%%%%%%%%%%%%%%%%%%%%%%%%%%%%%%%%%%%%%%%%%%%%%%%%%%%%%%%
% empty set:
% Try \varnothing from the amssymb package. It is perfectly round, and the 
% comprehensive LaTeX symbol list states that it is preferred by many to 
% \emptyset.
% https://tex.stackexchange.com/questions/22798/nice-looking-empty-set
\let\oldemptyset\emptyset
\let\emptyset\varnothing

\newcommand{\setEmpty}{\emptyset}%empty set
\newcommand{\setPrime}{\mathbb{P}} %prime numbers
\newcommand{\setWhole}{\mathbb{W}} %whole numbers
\newcommand{\setNat}{\mathbb{N}} %natural numbers
\newcommand{\setInt}{\mathbb{Z}} %integers,
\newcommand{\setIrrat}{\mathbb{I}} %irrational numbers
\newcommand{\setRat}{\mathbb{Q}} %rational numbers
\newcommand{\setReal}{\mathbb{R}} %real numbers
\newcommand{\setCom}{\mathbb{C}} %complex numbers


%% Constants %%%%%%%%%%%%%%%%%%%%%%%%%%%%%%%%%%%%%%%%%%%%%%%%%%%%%%%%%%%%
\newcommand\ct[1]{\text{\rmfamily\upshape #1}}
\newcommand{\rNatNum}{\ct{e}}% natural number
\newcommand{\rGoldRat}{\upphi} % Golden Ratio
\newcommand{\rPi}{\uppi} % pi
\newcommand{\rIm}{\ct{i}}% Imaginary
\newcommand{\zmat}{0} % zero matrix
\newcommand{\eyemat}{I} % identity matrix


%% Integrals and differentials %%%%%%%%%%%%%%%%%%%%%%%%%%%%%%%%%%%%%%%%%%
% https://tex.stackexchange.com/questions/60545/should-i-mathrm-the-d-in-my-integrals
\newcommand*{\diff}[1]{
\ifthenelse{\isempty{#1}}{\mathop{}\!\mathrm{d}}
                         {\mathop{}\!\mathrm{d^{#1}}}}


%%%%%%%%%%%%%%%%%%%%%%%%%%%%%%%%%%%%%%%%%%%%%%%%%%%%%%%%%%%%%%%%%%%%%%%%%%%%
%%%%%%%%%%%%%%%%%%%%%%%%%%%%%%%%%%%%%%%%%%%%%%%%%%%%%%%%%%%%%%%%%%%%%%%%%%%%
%
% END OF Mathematic symbols
%
%%%%%%%%%%%%%%%%%%%%%%%%%%%%%%%%%%%%%%%%%%%%%%%%%%%%%%%%%%%%%%%%%%%%%%%%%%%%
%%%%%%%%%%%%%%%%%%%%%%%%%%%%%%%%%%%%%%%%%%%%%%%%%%%%%%%%%%%%%%%%%%%%%%%%%%%%

%%%%%%%%%%%%%%%%%%%%%%%%%%%%%%%%%%%%%%%%%%%%%%%%%%%%%%%%%%%%%%%%%%%%%%%%%%%%
% diagbox to diagonally split a cell in a table
% https://ctan.org/pkg/diagbox
% Usage: \diagbox{Col name}{Row name}
% https://tex.stackexchange.com/questions/110018/professional-slashbox-alternative
\usepackage{diagbox}

%%%%%%%%%%%%%%%%%%%%%%%%%%%%%%%%%%%%%%%%%%%%%%%%%%%%%%%%%%%%%%%%%%%%%%%%%%%%
% Long table, see: http://tex.stackexchange.com/questions/301227/how-to-make-table-split-in-two-pages
\usepackage{longtable}
% Usage:
%To switch from a table/tabular combination to a longtable setup, only the
%following changes are required:
%
%1)Terminate the \caption directive with a double backslash;
%2)organize the header and footer material with \endhead, \endfoot, and 
%  possibly \endfirsthead and \endlastfoot directives; and
%3)omit the \centering directive that's usually provided for tabular 
%  material.
%
%\begin{longtable}{ccc}
%\caption{With \texttt{longtable} environment}\label{tab:b}\\
%% header and footer information
%\hline
%H1 & H2 & H3 \\
%\hline
%\endhead
%\hline
%\endfoot
%% body of table
%123 & 456 & 789 \\
%\end{longtable}

% Now making it smaller with footnotesize
%\begin{footnotesize}
%\begin{longtable}{|l p{10cm}|}
%\caption[Operations to Find Elements in an Associative Container]
%{Operations to Find Elements in an Associative Container}
%\label{tabC11N7OperationsToFindElementsInAnAssociativeContainer}\\
%% header and footer information
%\hline
%\endfirsthead
%\hline
%\endlastfoot
%% body of table
%\multicolumn{2}{|l|}{\textbf{Type Aliases}} \\
%\ctt{iterator}&Type of the iterator for this container type \\
%\end{longtable}
%\end{footnotesize}

%%%%%%%%%%%%%%%%%%%%%%%%%%%%%%%%%%%%%%%%%%%%%%%%%%%%%%%%%%%%%%%%%%%%%%%%%%%%

% How to use itemize inside table? Use fake items:
% http://tex.stackexchange.com/questions/150492/how-to-use-itemize-in-table-environment
\usepackage{booktabs}% http://ctan.org/pkg/booktabs
\newcommand{\tabitem}{~~\llap{\textbullet}~~}


%%%%%%%%%%%%%%%%%%%%%%%%%%%%%%%%%%%%%%%%%%%%%%%%%%%%%%%%%%%%%%%%%%%%%%%%%%%%
%%%%%%%%%%%%%%%%%%%%%%%%%%%%%%%%%%%%%%%%%%%%%%%%%%%%%%%%%%%%%%%%%%%%%%%%%%%%
%%% NEW COMMANDS
%%%%%%%%%%%%%%%%%%%%%%%%%%%%%%%%%%%%%%%%%%%%%%%%%%%%%%%%%%%%%%%%%%%%%%%%%%%%
%%%%%%%%%%%%%%%%%%%%%%%%%%%%%%%%%%%%%%%%%%%%%%%%%%%%%%%%%%%%%%%%%%%%%%%%%%%%


% Multiline cell: (rayray multi cell)
\newcommand{\rrmcell}[2][c]{%
  \begin{tabular}[#1]{@{}c@{}}#2\end{tabular}}
% From here: http://tex.stackexchange.com/questions/2441/how-to-add-a-forced-line-break-inside-a-table-cell
% Usage:
%Foo bar & \specialcell{Foo\\bar} & Foo bar \\    % vertically centered
%Foo bar & \specialcell[t]{Foo\\bar} & Foo bar \\ % aligned with top rule
%Foo bar & \specialcell[b]{Foo\\bar} & Foo bar \\ % aligned with bottom rule
%For those wanting to control the horizontal alignment, 
% change c@ to l@ or r@ (or make it another parameter like the vertical 
% alignment?)
% Note: This doesn't seem to work with \ctt, so I'll use parbox:
% \parbox[t]{5cm}{foo\\bar}


%---------------------------------------------------------------------------
% Custom higlighting.
%---------------------------------------------------------------------------
% use \ctt{} for code
%     \rnhl{} for highlighting importanting keywords.
%     \rnblue{} for random notes
%     \rngreen{} for random notes



%\definecolor{darkred}{rgb}{0.8,0.1,0.1}


% We use the soul package to do high-lighting (change background colour)
% of code words. We are not using \colorbox since text inside \colorbox can 
% not be split into several lines if it is too long. So the macro is most 
% suited for coloring a few words. If you want to color several sentences 
% which spread across several lines, it is better to use soul.
\usepackage{soul} % \sethlcolor{}, \hl
                  % \setulcolor{}, \ul


% Now use hl to change the background.
% Note: Putting \hl INSIDE \textcolor is important, if we put it outside,
% it would not work.

% The command \sethlcolor is fragile and will break in moving arguments. 
% You can either \protect it in moving arguments or, better, declare it 
% robust from the beginning:
\DeclareRobustCommand{\ctt}[1]{\texttt{\sethlcolor{light-gray}\hl{#1}}}
\DeclareRobustCommand{\cit}[1]{\textit{\sethlcolor{light-gray}\hl{#1}}}
\DeclareRobustCommand{\cbt}[1]{\textbf{\texttt{\sethlcolor{light-gray}\hl{#1}}}}

% To be used in section titles, since we need texorpdfstring.
% Reason: The PDF bookmarks are a different thing than the table of 
% contents. The bookmarks are not typeset by TeX: they simply are strings 
% of characters, so no math or general formatting instructions are allowed.
%\DeclareRobustCommand{\sctt}[1]{\texorpdfstring{\ctt{#1}}\texorpdfstring{\,}}

%$ ray ray text less than, greater than
%\newcommand*{\cttltgt}[1]{\textless{} #1 \textgreater{}}
% to work.

\definecolor{mynotescolourblue}{rgb}{0.0,0.0,0.3}
\definecolor{mynotescolourgreen}{rgb}{0, 0.3, 0}
\colorlet{mynotescolourred}{red!50!black!90!}

%% Now we do a few more of these, I like these...
%\DeclareRobustCommand{\hlcyan}[1]{{\sethlcolor{cyan}\hl{#1}}}
% rnhl = ray notes high light
% for highlighting keywords.
\DeclareRobustCommand{\rrhl}[1]{{\textcolor{mynotescolourred}{\sethlcolor{GreenYellow}\hl{\textbf{#1}}}}}

% ray attention: for things I should pay more attention to, for example, if 
% the book says "we read from a file", I may not pay attention to the word
% "file" much, although it is vital to the design of my program.
\DeclareRobustCommand{\rra}[1]{\setulcolor{mynotescolourred}\ul{#1}}

% rrgreen for green text for things I understand, but wanted to re-word.
\DeclareRobustCommand{\rrgreen}[1]{{\textcolor{mynotescolourgreen}{#1}}}
% Italic version
\DeclareRobustCommand{\rrgreenit}[1]{\rrgreen{\textit{#1}}}
% Typewriter version
\DeclareRobustCommand{\rrgreentt}[1]{\texttt[mynotescolourgreen]{#1}}


% rrred for red text for things I do not understand.
\DeclareRobustCommand{\rrred}[1]{{\textcolor{mynotescolourred}{#1}}}
% Italic version
\DeclareRobustCommand{\rrredit}[1]{\rrred{\textit{#1}}}
% Typewriter version
\DeclareRobustCommand{\rrredtt}[1]{\texttt[mynotescolourred]{#1}}


% rrblue blue text. There is not yet a specific use for this.
\DeclareRobustCommand{\rrblue}[1]{{\textcolor{mynotescolourblue}{#1}}}
% Italic version
\DeclareRobustCommand{\rrblueit}[1]{\rrblue{\textit{#1}}}
% Typewriter version
\DeclareRobustCommand{\rrbluett}[1]{\texttt[mynotescolourblue]{#1}}


% Usage: \source{Title}{Url}
% Title will be bold and underline. Url will be red{textbf{Source}}: Url
\DeclareRobustCommand{\sourceurl}[2]
{\noindent{\textbf{\underline{#1}}}\newline
 \textcolor{mynotescolourred}{\textbf{Source: }}\url{#2}}

\DeclareRobustCommand{\bkpgrf}[1]{\textbf{\rrred{BookPage: #1}}}

% Usage: \mdframedrule = a horizontal line for division.
\DeclareRobustCommand{\mdframedrule}{\noindent\rule{\textwidth}{1pt}}

%\newcommand{\pagerule}{\noindent\hrulefill}
%\newcommand{\dpagerule}{\noindent\hrulefill\newline\noindent\hrulefill}




\newlength{\qqseplinewidth}
\newlength{\qqseplinesep}
\setlength{\qqseplinewidth}{1mm}
\setlength{\qqseplinesep}{2mm}
\colorlet{qqsepline}{PaleVioletRed3}

\newcommand*{\qqsepline}{%
  \par
  \vspace{\dimexpr\qqseplinesep+.5\parskip}%
  \cleaders\vbox{%
    \begingroup % because of color
      \color{qqsepline}%
      \hrule width\linewidth height\qqseplinewidth
    \endgroup
  }\vskip\qqseplinewidth
  \vspace{\dimexpr\qqseplinesep-.5\parskip}%
}

\newcommand*{\qasepline}{\noindent\hfil\rule{0.5\textwidth}{.4pt}\hfil}

%https://tex.stackexchange.com/questions/19902/drawing-horizontal-line-same-width-as-the-page-width
\newcommand*{\linemarginlength}{\noindent\makebox[\linewidth]{\rule{\textwidth}{1pt}}}
\newcommand*{\linepagelength}{\noindent\makebox[\linewidth]{\rule{\paperwidth}{1pt}}}




\newcommand*{\exersepline}{\noindent\hfil\colorbox{qqsepline}{\rule{0.5\textwidth}{.4pt}}\hfil}


\newlength{\rrseplinewidth}
\newlength{\rrseplinesep}
\setlength{\rrseplinewidth}{1mm}
\setlength{\rrseplinesep}{2mm}
\colorlet{rrsepline}{BlueGreen}

\newcommand*{\rrsepline}{%
  \par
  \vspace{\dimexpr\rrseplinesep+.5\parskip}%
  \cleaders\vbox{%
    \begingroup % because of color
      \color{rrsepline}%
      \hrule width\linewidth height\rrseplinewidth
    \endgroup
  }\vskip\rrseplinewidth
  \vspace{\dimexpr\rrseplinesep-.5\parskip}%
}

\newcommand*{\rrrsepline}{\noindent\hfil\colorbox{rrsepline}{\rule{0.5\textwidth}{.4pt}}\hfil}


% Underlined and bold and newlined
% the \phantomsection is for when we use label{rh:foo}
\DeclareRobustCommand{\rrheaderunderline}[1]{\par\medskip\phantomsection\noindent\textbf{\underline{#1}}\medskip}
\DeclareRobustCommand{\rrheader}[1]{\par\medskip\phantomsection\noindent{\normalsize\textbf{#1}}\medskip}
\DeclareRobustCommand{\rrheaderlarge}[1]{\par\medskip\phantomsection\noindent{\large\textbf{#1}}\medskip}
\DeclareRobustCommand{\rrheaderLarge}[1]{\par\medskip\phantomsection\noindent{\Large\textbf{#1}}\medskip}


%\setlength{\parindent}{0pt}
%\nonzeroparskip % I forgot why I needed this. But it doesn't
% work in report mode.


% For my own notes.
\newcommand{\RayNotesBegin}{\rrsepline{}\begingroup\color{mynotescolourblue}}
\newcommand{\RayNotesEnd}{\endgroup\rrsepline{}}

%%%%%%%%%%%%%%%%%%%%%%%%%%%%%%%%%%%%%%%%%%%%%%%%%%%%%%%%%%%%%%%%%%%%%%%%%%%%
%%%%%%%%%%%%%%%%%%%%%%%%%%%%%%%%%%%%%%%%%%%%%%%%%%%%%%%%%%%%%%%%%%%%%%%%%%%%
%%%%%%%%%%%%%%%%%%%%%%%%%%%%%%%%%%%%%%%%%%%%%%%%%%%%%%%%%%%%%%%%%%%%%%%%%%%%
% https://tex.stackexchange.com/questions/133113/continued-figures
% continued figure coutners:

%Before the continued figure add these lines:
%
%\renewcommand{\thefigure}{\arabic{figure} (Cont.)} or
%\renewcommand{\thefigure}{\arabic{section}.\arabic{figure} (Cont.)}
%\renewcommand{\thefigure}{\arabic{chapter}.\arabic{figure} (Cont.)}
%\addtocounter{figure}{-1}
%and after it these line:
%
%\renewcommand{\thefigure}{\arabic{figure}}
%

\usepackage{chngcntr}


\usepackage{lipsum} % Just to put in some text




%%%%%%%%%%%%%%%%%%%%%%%%%%%%%%%%%%%%%%%%%%%%%%%%%%%%%%%%%%%%%%%%%%%%%%%%%%%%
% Cleveref, allows using cref
% NOTE: Cleverref has to be loaded LAST
% https://tex.stackexchange.com/questions/148699/equation-reference-undefined-when-using-cref-and-packageamsmath
% Otherwise it will not work:
% However, care must be taken when using cleveref in conjunction with other 
% packages that modify LaTeX's referencing system (see Section 11). 
% Basically, cleveref must be loaded last.
%%%%%%%%%%%%%%%%%%%%%%%%%%%%%%%%%%%%%%%%%%%%%%%%%%%%%%%%%%%%%%%%%%%%%%%%%%%%
\usepackage[sort&compress, % on multiple refs sort them and write as range
            capitalise, % Use Section not section etc.
            noabbrev, % Use Table not Tab. etc.
            nameinlink % Make the name (eg Section) part of the hyperlink
            ]{cleveref}

%%%%%%%%%%%%%%%%%%%%%%%%%%%%%%%%%%%%%%%%%%%%%%%%%%%%%%%%%%%%%%%%%%%%%%%%%%%%
%% New Theorems and environments, they have to come after loading cleverref
%% and clever ref must be the last package loaded.
%https://tex.stackexchange.com/questions/19104/cleveref-with-counters-with-same-name

\newcounter{exer}[section]
\newenvironment{exer}[1][]{\refstepcounter{exer}\par\medskip
  \noindent\textbf{Exercise~\theexer:#1} \rmfamily}{\medskip}

%\newtheoremstyle{exercise}
%  {\topsep}   % above space
%  {\topsep}   % below space
%  {\itshape}  % body font
%  {0pt}       % indent
%  {\bfseries} % head font
%  {}         % head punctuation
%  {5pt plus 1pt minus 1pt} % HEADSPACE
%  {}          % CUSTOM-HEAD-SPEC
%
%\theoremstyle{exercise}
%\newcounter{exercises}[chapter]
%\newtheorem{exer}[exercises]{Exercise}
%%\newtheorem{exers}{Exer}
%
\counterwithin{exer}{section}

%%%%%%%%%%%%%%%%%%%%%%%%%%%%%%%%%%%%%%%%%%%%%%%%%%%%%%%

\newcounter{prob}[chapter]
\newenvironment{prob}[1][]{\refstepcounter{prob}\par\medskip
  \noindent\textbf{Problem~\theprob:#1} \rmfamily}{\medskip}

%\newtheoremstyle{exercise}
%  {\topsep}   % above space
%  {\topsep}   % below space
%  {\itshape}  % body font
%  {0pt}       % indent
%  {\bfseries} % head font
%  {}         % head punctuation
%  {5pt plus 1pt minus 1pt} % HEADSPACE
%  {}          % CUSTOM-HEAD-SPEC
%
%\theoremstyle{exercise}
%\newcounter{exercises}[chapter]
%\newtheorem{exer}[exercises]{Exercise}
%%\newtheorem{exers}{Exer}
%
\counterwithin{prob}{chapter}


%%%%%%%%%%%%%%%%%%%%%%%%%%%%%%%%%%%%%%%%%%%%%%%%%%%%%%

%https://tex.stackexchange.com/questions/64931/using-newtheorem
%The \newtheorem command has two mutually exlusive optional arguments:
%
%Using
%\newtheorem{<name>}{<heading>}[<counter>]
%will create an environment <name> for a theorem-like structure; the counter
%for this structure will be subordinated to <counter>. 
%
%On the other hand, using
%\newtheorem{<name>}[<counter>]{<heading>}
%will create an environment <name> for a theorem-like structure; the counter
%for this structure will share the previously defined <counter> counter.
%
%In the definition of defn you need to use the first optional argument of
%\newtheorem to indicate that this environment shares the counter of the
%previously defined thm environment.
%
%If the counters need to be subordinate to the section counter, use section
%for the second optional argument of \newtheorem in the definition of thm.

\newtheorem{theorem}{Theorem}[chapter]

% A environment called corollary is created, the counter of this new
% environment will be reset every time a new theorem environment is used.
%\newtheorem{corollary}{Corollary}[theorem]
% changed so it shares the counter with theorem:
\newtheorem{corollary}[theorem]{Corollary}

% In this case, the even though a new environment called lemma is created,
% it will use the same counter as the theorem environment.
\newtheorem{lemma}[theorem]{Lemma}

\theoremstyle{remark}
\newtheorem*{remark}{Remark}

\theoremstyle{definition}
\newtheorem{definition}{Definition}[chapter]

% change the end of proof symbol
\renewcommand\qedsymbol{$\blacksquare$}

%%%%%%%%%%%%%%%%%%%%%%%%%%%%%%%%%%%%%%%%%%%%%%%%%%%%%%%%%%%%%%%%%%%%%%%%%%%%
%%%%%%%%%%%%%%%%%%%%%%%%%%%%%%%%%%%%%%%%%%%%%%%%%%%%%%%%%%%%%%%%%%%%%%%%%%%%
%% Continue with cleveref setup.


% Call subsections sections
%\crefname{subsection}{Section}{Sections}
%\Crefname{subsection}{Section}{Sections}

% Use the section symbol §
\crefname{subsection}{\S}{\S\S}
\Crefname{subsection}{\S}{\S\S}
\crefname{section}{\S}{\S\S}
\Crefname{section}{\S}{\S\S}


% just use (...) for equations
\crefformat{equation}{#2(#1)#3}
\crefrangeformat{equation}{#3(#1)#4--#5(#2)#6}
\crefmultiformat{equation}{(#2#1#3)}{ and~(#2#1#3)}{, (#2#1#3)}{ and~(#2#1#3)}

% Except for start of sentences where we need to say "Equations"
\Crefformat{equation}{Equation~#2(#1)#3}
\Crefrangeformat{equation}{Equations~#3(#1)#4--#5(#2)#6}
\Crefmultiformat{equation}{Equations~(#2#1#3)}{ and~(#2#1#3)}{, (#2#1#3)}{ and~(#2#1#3)}

%\newcommand\crefpairconjunction{--}
%\crefname{equation}{}{} % no "equation[s]" label mid-sentence
%\Crefname{equation}{Equation}{Equations}

% a reference for "this X"
%\newcommand{\thisref}[1]{this \lcnamecref{#1}}
%\newcommand{\Thisref}[1]{This \lcnamecref{#1}}

% Need this for listings
\crefname{lstlisting}{listing}{listings}
\Crefname{lstlisting}{Listing}{Listings}

% Need this for listings
\crefname{exer}{exercise}{exercises}
\Crefname{exer}{Exercise}{Exercises}

% Need this for listings
\crefname{prob}{problem}{problems}
\Crefname{prob}{Problem}{Problems}

%\crefname{lemma}{lemma}{lemmas}
%\Crefname{lemma}{Lemma}{Lemmas}
%\crefname{theorem}{theorem}{theorems}
%\Crefname{theorem}{Theorem}{Theorems}



% Teach hyperref how to handle cref inside sections, note, this is not used.
% But I'm keeping it here so I know how to do this in the future.
\pdfstringdefDisableCommands{\def\secexref#1{Example \ref{#1}}}
\DeclareRobustCommand*{\refcfirst}{\ref*{ex:sqlgr}}

% wrapper around cref to get the page number as well.
\DeclareRobustCommand{\pagecref}[1]{\cref{#1} p.\pageref{#1}}
\DeclareRobustCommand{\Pagecref}[1]{\Cref{#1} p.\pageref{#1}}

% Used for referencing exercises for the answer sections
%\DeclareRobustCommand{\Exerref}[1]{\noindent\textbf{\Pagecref{#1}:}}
\newcommand*{\Exerref}[1]{\noindent\textbf{\Pagecref{#1}:}}


%https://tex.stackexchange.com/questions/10102/multiple-references-to-the-same-footnote-with-hyperref-support-is-there-a-bett/10116#10116
% How to use:
% text...\footnote{\label{first}First footnote}
% text...\cref{first}
\crefformat{footnote}{#2\footnotemark[#1]#3}
%%%%%%%%%%%%%%%%%%%%%%%%%%%%%%%%%%%%%%%%%%%%%%%%%%%%%%%%%%%%%%%%%%%%%%%%%%%%
%%%%%%%%%%%%%%%%%%%%%%%%%%%%%%%%%%%%%%%%%%%%%%%%%%%%%%%%%%%%%%%%%%%%%%%%%%%%


