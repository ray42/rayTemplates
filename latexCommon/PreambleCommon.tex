%%%% Stuff common to all latex documents.

% New versions of latex has deprecated \it and \tt with \itshape and
% \ttfamily. But some packages (such as lstlisting)  still use it. So we
% alias for compatibility
\renewcommand{\it}{\itshape}
\renewcommand{\tt}{\ttfamily}

%%%%%%%%%%%%%%%%%%%%%%%%%%%%%%%%%%%%%%%%%%%%%%%%%%%%%%%%%%%%%%%%%%%%%%%%%%%%
%%%%%%%%%%%%%%%%%%%%%%%%%%%%%%%%%%%%%%%%%%%%%%%%%%%%%%%%%%%%%%%%%%%%%%%%%%%%
%%%%%%%%%%%%%%%%%%%%%%%%%%%%%%%%%%%%%%%%%%%%%%%%%%%%%%%%%%%%%%%%%%%%%%%%%%%%
%%%%%%%%%%%%%%%%%%%%%%%%%%%%%%%%%%%%%%%%%%%%%%%%%%%%%%%%%%%%%%%%%%%%%%%%%%%%

%%% How to use scripts:
%%% Update mendeley references
%\immediate\write18{./Bin/update_mendeley_ref.sh}

%%% Generate the word count.
%\immediate\write18{./Bin/runtexcount.sh > ./Temp/texcountoutput.tex}

%%%%%%%%%%%%%%%%%%%%%%%%%%%%%%%%%%%%%%%%%%%%%%%%%%%%%%%%%%%%%%%%%%%%%%%%%%%%
%%%%%%%%%%%%%%%%%%%%%%%%%%%%%%%%%%%%%%%%%%%%%%%%%%%%%%%%%%%%%%%%%%%%%%%%%%%%
%%% PACKAGES %%%%%%%%%%%%%%%%%%%%%%%%%%%%%%%%%%%%%%%%%%%%%%%%%%%%%%%%%%%%%%%
%%%%%%%%%%%%%%%%%%%%%%%%%%%%%%%%%%%%%%%%%%%%%%%%%%%%%%%%%%%%%%%%%%%%%%%%%%%%
%%%%%%%%%%%%%%%%%%%%%%%%%%%%%%%%%%%%%%%%%%%%%%%%%%%%%%%%%%%%%%%%%%%%%%%%%%%%

%---------------------------------------------------------------------------
% Apply patches to other packages. Currently over required for mdframed
%---------------------------------------------------------------------------
\usepackage{xpatch}

%---------------------------------------------------------------------------
% Silence warnings
%---------------------------------------------------------------------------

% Silence warnings by hyper link, and mdframed
\usepackage{silence}

%\WarningFilter{latexfont}{Some font shapes were not available, defaults substituted}
\WarningFilter{latex}{Empty bibliography}

% The below is only for use when includeonly is used.
% latex gets confused and thinks there are multiple labels, but when
% we comment out includeonly{}, everything is fine.
% NOTE: Make sure to commen this out when we are defining new labels
% since we DO want to know if we have multiply-defined labels.
% UNCOMMENT THIS THIS THIS THIS THIS THIS
% UNCOMMENT THIS THIS THIS THIS THIS THIS
% UNCOMMENT THIS THIS THIS THIS THIS THIS
%\WarningFilter{latex}{Label}
%\WarningFilter{latex}{There were multiply-defined labels}

\WarningFilter{memoir}{You are using the caption package}
% We filter out this message since we use the caption package.
% the caption package is required to make the longtable captions work
% properly:
%Class memoir Warning: You are using the caption package with the memoir
%class. To prepare we will now reset all captioning macros and configurations
%to kernel defaults, and then let the caption package take over. Please
%remember to use the caption package interfaces in order to configure your
%captions. .

% Note: use this variable for array stretch
\newcommand{\arraystretchsize}{1.3}


%---------------------------------------------------------------------------
% Colours
%---------------------------------------------------------------------------
%\usepackage{color} % loaded by xcolor

% If you are using tikz or pstricks package you must declare the xcolor 
% package before that, otherwise it will not work.
% http://ctan.org/pkg/xcolor
\usepackage[usenames,dvipsnames,svgnames,x11names]{xcolor}

%---------------------------------------------------------------------------
% Float placements
%---------------------------------------------------------------------------
\usepackage{placeins}
% This is to use \FloatBarrier, to make latex flush all floats.

%---------------------------------------------------------------------------
% Tikz, pgfplots etc...
%---------------------------------------------------------------------------

%Tools for drawing Euclidean geometry
\usepackage{tkz-euclide}

\usepackage{pgfplots}
% Compatibility mode, because I have missing tikz labels or something.
\pgfplotsset{compat=1.9}

\usepackage{tikz}
\usetikzlibrary{calc}
\usetikzlibrary{matrix,arrows}
\usetikzlibrary{decorations.markings}
\tikzset{->-/.style={decoration={
  markings,
  mark=at position #1 with {\arrow{>}}},postaction={decorate}}}

%\tikzset{->-/.style={decoration={
%  markings,
%  mark=at position 0.25 and 0.9 step 0.25 with {\arrow{>}}},postaction={decorate}}}

%\tikzset{->>>>-/.style={decoration={
%  markings,
%  mark=at position 0.25 with {\arrow{>}}},postaction={decorate}},
%  mark=at position 0.5 with {\arrow{>}}},postaction={decorate}};}

% This is for the braces in the Knudsen diagram in Chapter 1
\usetikzlibrary{decorations.pathreplacing}

% This is for the flowchart in development of the BPF
\usetikzlibrary{shapes.geometric, arrows,chains}
\usetikzlibrary{positioning}


%---------------------------------------------------------------------------
% Frame around text.
%---------------------------------------------------------------------------
\usepackage[framemethod=TikZ]{mdframed}

% Apply the patch, this uses package xpatch: 
% source: http://tex.stackexchange.com/questions/162640/how-to-get-more-than-3-levels-of-nesting-in-mdframed-environments
\makeatletter
\xpatchcmd{\mdf@preenvsetting}
  {\mdf@envdepth >\tw@}
  {\mdf@envdepth >20}
  {}
  {}
\makeatother

% Ignore bad break warning
\WarningFilter{mdframed}{You got a bad break}

\makeatletter

\mdf@PackageWarning{You got a bad break\MessageBreak
  because the last split box is empty\MessageBreak
  You have to change the settings}

\makeatother


%\mdfdefinestyle{mdframed1}{
%  frametitle={}, % next we do the placement of the frames
%  skipabove=0pt,
%  skipbelow=0pt,
%  leftmargin=0pt,
%  rightmargin=0pt,
%  innerleftmargin=10pt,
%  innerrightmargin=10pt,
%  innertopmargin=5pt,
%  innerbottommargin=5pt, % next we control the line
%  linewidth=0.4pt,
%  innerlinewidth=0pt,
%  middlelinewidth=0.4pt,
%  outerlinewidth=0pt,
%  roundcorner=0pt,% Next we do the colours
%  linecolor=black,
%  innerlinecolor=black,
%  middlelinecolor=black,
%  outerlinecolor=black,}

% Use the styles with begin{mdframed}[style=mdframed1]
% Here as use the standard latex colors: black, red, green, blue, cyan,
% magenta, yellow
\newcommand{\mdfinnerrightmargin}{2pt}
\newcommand{\mdflinewidth}{1pt}

\mdfdefinestyle{mdframed1}{
  innerrightmargin=\mdfinnerrightmargin,
  linewidth=\mdflinewidth,
  linecolor=black}
\mdfdefinestyle{mdframed2}{
  innerrightmargin=\mdfinnerrightmargin,
  linewidth=\mdflinewidth,
  linecolor=red}
\mdfdefinestyle{mdframed3}{
  innerrightmargin=\mdfinnerrightmargin,
  linewidth=\mdflinewidth,
  linecolor=green}
\mdfdefinestyle{mdframed4}{
  innerrightmargin=\mdfinnerrightmargin,
  linewidth=\mdflinewidth,
  linecolor=blue}
\mdfdefinestyle{mdframed5}{
  innerrightmargin=\mdfinnerrightmargin,
  linewidth=\mdflinewidth,
  linecolor=cyan}
\mdfdefinestyle{mdframed6}{
  innerrightmargin=\mdfinnerrightmargin,
  linewidth=\mdflinewidth,
  linecolor=magenta}
\mdfdefinestyle{mdframed7}{
  innerrightmargin=\mdfinnerrightmargin,
  linewidth=\mdflinewidth,
  linecolor=yellow}

\mdfdefinestyle{mdfexercise}{
  roundcorner=8pt}

%%%%%%%%%%%%%%%%%%%%%%%%%%%%%%%%%%%%%%%%%%%%%%%%%%%
%% curly boxes with background colours
%%%%%%%%%%%%%%%%%%%%%%%%%%%%%%%%%%%%%%%%%%%%%%%%%%%
%\mdfdefinestyle{mdfcurlyblue}{
%  backgroundcolor=blue!20,
%%  shadow=true,
%  roundcorner=8pt}
%
%\mdfdefinestyle{mdfcurlyyellow}{
%  backgroundcolor=yellow!20,
%%  shadow=true,
%  roundcorner=8pt}

\mdfdefinestyle{mdfNOTE}{
  backgroundcolor=blue!30,
%  shadow=true,
  roundcorner=8pt}

\mdfdefinestyle{mdfKEYCONCEPT}{
  backgroundcolor=green!30,
%  shadow=true,
  roundcorner=8pt}

\mdfdefinestyle{mdfWARNING}{
  backgroundcolor=yellow!30,
%  shadow=true,
  roundcorner=8pt}

\mdfdefinestyle{mdfREMEMBER}{
  backgroundcolor=red!25,
  frametitlefont={\bfseries\color{red!65}},
  frametitlebackgroundcolor=DarkBlue,
  frametitlealignment=\center,
%  shadow=true,
  roundcorner=8pt}

\mdfdefinestyle{mdfRAYSNOTES}{
  backgroundcolor=blue!12,
  frametitlealignment=\center,
%  shadow=true,
  roundcorner=8pt}

\mdfdefinestyle{mdfOUTANS}{
  backgroundcolor=green!20,
%  shadow=true,
  roundcorner=8pt}

\mdfdefinestyle{mdfINANS}{
  backgroundcolor=yellow!20,
%  shadow=true,
  roundcorner=8pt}

%---------------------------------------------------------------------------
% url, hyperref, bookmark, etc...
%---------------------------------------------------------------------------

\WarningFilter{hyperref}{You have enabled option `breaklinks'.}

% url is loaded by hyperref, which is loaded by bookmark.

% The command \PassOptionsToPackage (used below) tells LaTeX to behave as 
% if its options were passed, when it finally loads a package. As you would 
% expect from its name, \PassOptionsToPackage can deal with a list of 
% options, just as you would have in the the options brackets of 
% \usepackage.

% Ordinarily, breaks are not allowed after “-” characters because this leads
% to confusion. (Is the “-” part of the address or just a hyphen?) The 
% package option “[hyphens]” allows breaks after explicit hyphen characters.
% The \url command will never ever hyphenate words.
% Also can use \path{}
\PassOptionsToPackage{hyphens,obeyspaces,spaces}{url}

% Allows link text to break across lines; since this cannot be accommodated 
% in PDF, it is only set true by default if the pdftex driver is used. This
% makes links on multiple lines into different PDF links to the same target.
\PassOptionsToPackage{breaklinks=true}{hyperref}

% http://tex.stackexchange.com/questions/167948/package-rerunfilecheck-warning-file-out-has-changed
% Or add \usepackage{bookmark}. Then a more modern implementation of the 
% bookmarks managing is used without .out file. The bookmarks are updated 
% earlier, thus in most cases only one LaTeX run is needed.
\usepackage{bookmark}

%% breakurl needs hyperref, but hyperref is loaded by bookmark!
\usepackage[anythingbreaks]{breakurl}

% Moved this into individual main.tex, since I need to set the pdf title,
% which is unique per project
%%% Internal links
%\hypersetup{
%colorlinks=true,
%linkcolor=blue,
%filecolor=magenta,      
%urlcolor=cyan,
%citecolor=cyan,
%%pdfborder={0 0 0},      % No borders around internal hyperlinks
%%pdftitle={RNO LeetCode},
%pdfauthor={Ray White} % author
%}
%\urlstyle{same}

%memhfixc  --  Adjustment for using hyperref in memoir documents
%Any recent version of hyperref will automatically load this package if it
%finds itself running under the memoir class.
%Otherwise, the package should simply be loaded (without options) after 
% hyperref has been loaded.
%\usepackage{memhfixc}


%---------------------------------------------------------------------------
% graphic, graphicx
%---------------------------------------------------------------------------

%% This is for includegraphics{}
% \usepackage{graphics} % Already loaded by graphicsx
% pdftex (default if compiling with pdflatex), if you are compiling with 
% pdftex to get a PDF that you will see with any PDF viewer.

% This gets an error, reason is below:
%\usepackage[pdftex]{graphicx}

% PDFLaTeX does not support EPS files. However, some modern LaTeX 
% distributions will try to automatically convert it to a PDF image, which 
% maybe this doesn't work in your case. It would be also possible that the 
% size information, namely the bounding box, in that EPS file is missing.
%
%If you are using latex, i.e. the DVI mode of LaTeX, which supports EPS then
%you should remove the incorrect pdftex option from the graphicx package. 
%The packages should be able to figure out the used driver by themselves, so
%you should avoid such options anyway. If you are using pdflatex then it 
%would be better to convert the EPS to a PDF file manually, e.g. using the 
%epstopdf tool (I'm not sure if it comes with the Windows version, but I 
%think so). You need to change the .eps extension to .pdf then, of course. 
%You can also drop the extension and LaTeX will look for files with the 
%given base name with all supported extensions, i.e. \includegraphics{file} 
%will use file.eps for latex and file.pdf (or file.png, file.jpg) for 
%pdflatex.

% TL;DR: Take out pdftex, and convert eps to pdf manually (although the
% engine should do it for us, if required)
\usepackage{graphicx}


%---------------------------------------------------------------------------
% fonts, symbols and encodings
%---------------------------------------------------------------------------

% I want to use sans serif
\usepackage{helvet}
\renewcommand{\familydefault}{\sfdefault}

%https://tex.stackexchange.com/questions/174021/typeset-whole-document-in-sans-serif-including-math-mode
%\usepackage{sansmath} % Enables turning on sans-serif math mode, and using other environments
%\sansmath % Enable sans-serif math for rest of document


%\usepackage{textcomp}
\usepackage[official]{eurosym}

\usepackage[utf8]{inputenc} % If utf8 encoding
%If not utf8 encoding, then this is probably the way to go
%\usepackage[lantin1]{inputenc} 
\usepackage[T1]{fontenc}    %

% Multilingual support for Plain TEX or LATEX
\usepackage[english]{babel} % English please

\usepackage[final]{microtype} % Less badboxes

% For bold texttt, usage: \texttt{\textbf{foo}}
\usepackage[scaled=0.85]{beramono}

%\usepackage{lmodern}% http://ctan.org/pkg/lm

%\usepackage{courier}

% \usepackage{kpfonts} %Font

% Note: difference between newcommand and newcommand*:

%\newcommand{\examplea}[1]

%\newcommand*{\exampleb}[1]

% Most of the time, \newcommand* is the best choice as you want the 
% error-checking that it provides. That is why examples given by experienced
% LaTeX users normally use this form, rather than just \newcommand
% So I will use newcommand*

% rr stands for rayray
% so rrbf is rayray's bold font, etc...

%\texttt[<color>]{<stuff>} has been redefined to take an optional <color> 
% (default is black), source: 
% http://tex.stackexchange.com/questions/36455/how-can-i-ensure-all-teletype-monospaced-text-is-in-a-certain-color
\let\oldtexttt\texttt% Store \texttt
% \texttt[<color>]{<stuff>}
\renewcommand{\texttt}[2][black]{\textcolor{#1}{\ttfamily #2}}



\newcommand*{\rrbf}[1]{\textbf{#1}}
\newcommand*{\rrbftt}[1]{\texttt{\textbf{#1}}}
\newcommand*{\rrit}[1]{\textit{#1}}
\newcommand*{\rrtt}[1]{\texttt{#1}}
\newcommand*{\rrem}[1]{\emph{#1}}

\newcommand*{\rrbfb}[1]{\textcolor{blue}{\textbf{#1}}}
\newcommand*{\rrbfbtt}[1]{\texttt[blue]{\textbf{#1}}}

%% Other shorthands

%% Here, again rr stands for rayray, but the t stands for text.
%% This means these should be used in text mode, rrt = rayray text mode.

%% Text backslash
\newcommand*{\rrtbslash}{\textbackslash{}}
%% Text less than
\newcommand*{\rrtlt}{\textless{}}
\newcommand*{\rrtltlt}{\textless{}\textless{}}
%% Text greather than
\newcommand*{\rrtgt}{\textgreater{}}
\newcommand*{\rrtgtgt}{\textgreater{}\textgreater{}}



\newcommand{\rcindent}[1]{\noindent\hspace*{2em}}

\newcommand{\definedtermsskip}{\bigskip}



%---------------------------------------------------------------------------
% Maths
%---------------------------------------------------------------------------

%http://tex.stackexchange.com/questions/32100/what-does-each-ams-package-do
%Most of the answer was extracted from the Introduction sections of the 
%documentation of amsmath and amsthm:
%\usepackage{amsmath}%AMS mathematical facilities for LATEX
\usepackage{mathtools} % mathtools loads/extends amsmath.
\usepackage{amsthm} % ams theorem
\usepackage{amssymb} % loads amsfonts

%https://tex.stackexchange.com/questions/232922/a-hack-for-getting-a-capital-weierstrass-p-in-order-to-represent-the-power-set
\usepackage{mathrsfs}
\usepackage[mathscr]{euscript}
\let\euscr\mathscr \let\mathscr\relax% just so we can load this and rsfs
\usepackage[scr]{rsfso}
\newcommand{\powerset}{\raisebox{.15\baselineskip}{\Large\ensuremath{\wp}}}
%Usage:
%$\mathcal{P}(X)$ 
%$\euscr{P}(X)$
%$\mathscr{P}(X)$
%$\powerset(X)$

% For boxes around equations
% https://tex.stackexchange.com/questions/20575/attractive-boxed-equations
% https://tex.stackexchange.com/questions/193242/how-to-make-a-box-around-an-equation-in-align-environment
\usepackage{empheq}
\usepackage[most]{tcolorbox}
%\begin{empheq}[box=\tcbhighmath]{equation*}
%    c_i = \langle\psi|\phi\rangle
%\end{empheq}



%1)amsmath provides miscellaneous enhancements for improving the information 
%structure and printed output of documents containing mathematical formulas.
%Some of the features provided by this package are:
%  The \DeclareMathOperator command (through the auxiliary package amsopn) 
%  to define new "operator name" commands analogous to \sin and \lim, 
%  including proper side spacing and automatic selection of the correct font
%  style and size (even when used in sub- or superscripts).
%
%  Multiple substitutes for the eqnarray environment to make various kinds 
%  of equation arrangements easier to write.
%
%  Equation numbers automatically adjust up or down to avoid overprinting on
%  the equation contents (unlike eqnarray).
%
%  Spacing around equals signs matches the normal spacing in the equation 
%  environment (unlike eqnarray).
%
%  A way to produce multiline subscripts as are often used with summation or
%  product symbols.
%
%  An easy way to substitute a variant equation number for a given equation
%  instead of the automatically supplied number.
%
%  An easy way to produce subordinate equation numbers of the form (1.3a) 
%  (1.3b) (1.3c) for selected groups of equations.
%
%  The \text command (through the auxiliary package amstext) for typesetting
%  a fragment of text inside a display.

%2)amsthm helps to define theorem-like structures; the introduction to the 
%documentation gives a nice concise description of the package:
%  The amsthm package provides an enhanced version of LaTeX's  \newtheorem 
%  command for defining theorem-like environments. The enhanced \newtheorem
%  recognizes a \theoremstyle specification (as in Mittelbach's theorem 
%  package) and has a * form for defining unnumbered environments. The
%  amsthm package also defines a proof environment that automatically adds a
%  QED symbol at the end. AMS document classes incorporate the amsthm 
%  package, so everything described here applies to them as well.
%
%  If the amsthm package is used with a non-AMS document class and with the
%  amsmath package, amsthm must be loaded after amsmath, not before.

%3)amssymb provides an extended symbol collection. For example, after 
%loading amssymb you have the following additional binary relation symbols:
%\barwedge, \boxdot, \boxminus, \boxplus, \boxtimes, \Cap, \Cup (and many 
%more), the arrow \leadsto, and some other symbols such as \Box and 
%\Diamond. Another useful feature is the \mathbb command to produce 
%blackboard bold characters.

% Since amssymb internally loads amsfonts, it's enough to load the former.


%---------------------------------------------------------------------------
% New column type, uses array package
% http://tex.stackexchange.com/questions/154958/vertically-centered-and-right-left-center-horizontal-alignment-in-tabular
%---------------------------------------------------------------------------
\usepackage{array}
\newcolumntype{C}[1]{>{\centering\arraybackslash}m{#1}}   %% centered
\newcolumntype{R}[1]{>{\raggedleft\arraybackslash}m{#1}}  %% right aligned
% Use:
%\begin{tabular}{|m{5em}| C{5em}| R{5em}|}
%left & center & right
%\end{tabular}
% produces:
% | left    |  center  |    right |


%%%%%%%%%%%%%%%%%%%%%%%%%%%%%%%%%%%%%%%%%%%%%%%%%%%%%%%%%%%%%%%%%%%%%%%%%%%%
%%%%%%%%%%%%%%%%%%%%%%%%%%%%%%%%%%%%%%%%%%%%%%%%%%%%%%%%%%%%%%%%%%%%%%%%%%%%

%---------------------------------------------------------------------------
% Other packages
%---------------------------------------------------------------------------

\usepackage{verbatim} % extension of verbatim

\usepackage{fancyvrb}

% Stop footnote from running across pages:
% http://tex.stackexchange.com/questions/32208/footnote-runs-onto-second-page
%\interfootnotelinepenalty=\@M
% http://www.tex.ac.uk/FAQ-splitfoot.html
\interfootnotelinepenalty=10000

% Use verbatim in footnotes.
\VerbatimFootnotes
\fvset{fontfamily=courier,
%       fontseries=b,
       fontsize=\small,
       xleftmargin=0.05cm,
       frame=leftline,}
%       showspaces=true}
%% fancyvrb allows \begin{Verbatim}[samepage=true],
%% fancyvrb allows \begin{Verbatim}[xleftmargin=.5in],

\usepackage{upgreek} % For uptau, used for tangent vectors
\usepackage{bm} % Bold maths symbols \bm{}
\usepackage{xifthen}% provides \isempty test, this is used for optional arg.

%https://tex.stackexchange.com/questions/287657/learning-to-use-xparse
%https://tex.stackexchange.com/questions/33749/xparse-define-new-command-with-multiple-optional-parameters
%https://tex.stackexchange.com/questions/60893/new-command-with-optional-argument-being-first-argument
% For using \NewDocumentCommand
\usepackage{xparse}

%---------------------------------------------------------------------------
% Code listing
%---------------------------------------------------------------------------

\usepackage{listings}

%\lstMakeShortInline[columns=fixed]|

\lstdefinelanguage{newcpp}{
language=C++,
morekeywords={auto,concept,constexpr,decltype,final,override,requires,
              concept},
}

% Define a new colour
\definecolor{light-gray}{gray}{0.92}

\lstdefinelanguage{pseudocode}{
%language=C++,
comment=[l]{//},
morecomment=[s]{/*}{*/},
morecomment=[s][\color{blue}]{/*+}{*/},
morecomment=[s][\color{red}]{/*-}{*/},
morekeywords={for, For, to, downto, input, output, return, datatype,
  function, in, if, If, else, Else, foreach, while, While, begin, end, do,
  procedure},
}

\lstdefinestyle{pseudostyle}{
  captionpos=b,
  numbers=left, 
  stepnumber=1, 
  numberstyle=\tiny,
  frame=leftline,
%  framexleftmargin=5mm,
  xleftmargin=5mm,
  showstringspaces=false, 
  language=pseudocode,
  mathescape=true,
  backgroundcolor=\color{light-gray},
  basicstyle=\linespread{0.8}\ttfamily\small,
%  keywordstyle=\color{blue}\ttfamily,
  keywordstyle=\color{black}\bfseries\em,
  stringstyle=\color{magenta}\ttfamily,
  commentstyle=\color{darkgray}\ttfamily,
  morecomment=[l][\color{magenta}]{\#},
  moredelim=**[is][\itshape]{@it}{@}, % this keeps keyword syntax
  moredelim=**[is][\bfseries]{@bf}{@}, % this keeps keyword syntax
  moredelim=**[is][\bfseries\itshape]{@ibf}{@}, % this keeps keyword syntax
  moredelim=**[is][\color{red!70!black}]{@R}{@},
%  moredelim=[is][\color{red!70!black}]{@@}{@@},
}

\lstdefinestyle{raygeneric}{
  captionpos=b,
  numbers=none, 
  stepnumber=1, 
  numberstyle=\tiny,
  frame=leftline,
%  framexleftmargin=5mm,
  xleftmargin=5mm,
  showstringspaces=false, 
  mathescape=true,
  backgroundcolor=\color{light-gray},
  basicstyle=\linespread{0.8}\ttfamily\small,
%  keywordstyle=\color{blue}\ttfamily,
  keywordstyle=\color{black}\bfseries\em,
  stringstyle=\color{magenta}\ttfamily,
  commentstyle=\color{darkgray}\ttfamily,
  morecomment=[l][\color{magenta}]{\#},
  moredelim=**[is][\itshape]{@}{@}, % this keeps keyword syntax
  moredelim=**[is][\bfseries]{`}{`}, % this keeps keyword syntax
  moredelim=**[is][\bfseries\itshape]{¬}{¬}, % this keeps keyword syntax
  moredelim=**[is][\color{red!70!black}]{@@}{@@},
%  moredelim=[is][\color{red!70!black}]{@@}{@@},
}


% Used for C++
\lstdefinestyle{raycpp}{
  captionpos=b,
%  numbers=left, 
  numbers=none, 
  stepnumber=2, 
  frame=single,
  showstringspaces=false, 
  language=newcpp,
  backgroundcolor=\color{light-gray},
  basicstyle=\linespread{0.8}\ttfamily\footnotesize,
  keywordstyle=\color{blue}\ttfamily,
  stringstyle=\color{red}\ttfamily,
  commentstyle=\color{darkgray}\ttfamily,
  morecomment=[l][\color{magenta}]{\#},
}

% Python
\lstdefinestyle{raypython}{
  captionpos=b,
  %  numbers=left, 
  numbers=none, 
  stepnumber=2, 
  frame=leftline,
%  framexleftmargin=5mm,
  xleftmargin=5mm,
  showstringspaces=false, 
  language=Python,
  mathescape=true,
  backgroundcolor=\color{light-gray},
  basicstyle=\linespread{0.8}\ttfamily\small,
  keywordstyle=\color{blue}\ttfamily,
  stringstyle=\color{magenta}\ttfamily,
  commentstyle=\color{darkgray}\ttfamily,
  morecomment=[l][\color{magenta}]{\#},
%  moredelim=**[is][\color{red}]{@}{@}, % this keeps keyword syntax
  moredelim=[is][\color{red!70!black}]{@}{@},
}



%% Used for bash scripts
\lstdefinestyle{raybash}{
        captionpos=b,
%        numbers=left, 
        numbers=none, 
        stepnumber=2, 
        frame=single,
        showstringspaces=false, 
        language=bash,
        basicstyle=\linespread{0.8}\ttfamily\footnotesize,
        keywordstyle=\color{blue}\ttfamily,
        stringstyle=\color{red}\ttfamily,
        commentstyle=\color{gray}\ttfamily,
        morecomment=[l][\color{magenta}]{\#},}

\lstdefinestyle{raybashsmall}{
        captionpos=b,
%        numbers=left, 
        numbers=none, 
        stepnumber=2, 
        frame=single,
        showstringspaces=false, 
        language=bash,
        basicstyle=\linespread{0.8}\ttfamily\scriptsize,
        keywordstyle=\color{blue}\ttfamily,
        stringstyle=\color{red}\ttfamily,
        commentstyle=\color{gray}\ttfamily,
        morecomment=[l][\color{magenta}]{\#},}

% Note: Use this for captions:
%\begin{lstlisting}[style=raycppsmall,
%  caption={Grade Clustering Program},captionpos=b,
%  label={lstC3N1Name}]

\newcommand{\lred}[1]{\textcolor{red}{\mathtt{#1}}}
\newcommand{\lredt}[1]{\textcolor{red}{\mathtt{\text{#1}}}}
\newcommand{\lr}[1]{\textcolor{red}{\mathtt{#1}}}
\newcommand{\lrt}[1]{\textcolor{red}{\mathtt{\text{#1}}}}

% Used for C++ snippets of code, unimportant snippets.
% For color text: https://tex.stackexchange.com/questions/115547/textcolor-within-lstlisting
%\definecolor{darkred}{red}{0.5}
\lstdefinestyle{raycppsnippet}{
  captionpos=b,
  %  numbers=left, 
  numbers=none, 
  stepnumber=2, 
  frame=leftline,
%  framexleftmargin=5mm,
  xleftmargin=5mm,
  showstringspaces=false, 
  language=newcpp,
  mathescape=true,
  backgroundcolor=\color{light-gray},
  basicstyle=\linespread{0.8}\color{black}\ttfamily\small,
  keywordstyle=\color{blue}\ttfamily,
  stringstyle=\color{magenta}\ttfamily,
  commentstyle=\color{darkgray}\ttfamily,
  morecomment=[l][\color{magenta}]{\#},
%  moredelim=**[is][\color{red}]{@}{@}, % this keeps keyword syntax
  moredelim=[is][\color{red!70!black}]{@}{@},
}

\lstdefinestyle{raycppcheight}{
  captionpos=b,
  %  numbers=left, 
  numbers=none, 
  stepnumber=2, 
  frame=leftline,
%  framexleftmargin=5mm,
  xleftmargin=5mm,
  showstringspaces=false, 
  language=newcpp,
  mathescape=true,
  backgroundcolor=\color{light-gray},
  basicstyle=\linespread{0.8}\ttfamily\small,
  keywordstyle=\color{blue}\ttfamily,
  stringstyle=\color{magenta}\ttfamily,
  commentstyle=\color{darkgray}\ttfamily,
  morecomment=[l][\color{magenta}]{\#},
  moredelim=**[is][\itshape]{@}{@}, % this keeps keyword syntax
  moredelim=**[is][\bfseries]{`}{`}, % this keeps keyword syntax
  moredelim=**[is][\bfseries\itshape]{¬}{¬}, % this keeps keyword syntax
%  moredelim=[is][\color{red!70!black}]{@}{@},
}

% Use this, this is what I use. Shortcut is:
% lstpp{TAB}
\lstdefinestyle{raycppnewsnippet}{
  captionpos=b,
  %  numbers=left, 
  numbers=none, 
  stepnumber=2, 
  frame=leftline,
%  framexleftmargin=5mm,
  xleftmargin=5mm,
  showstringspaces=false, 
  language=newcpp,
  mathescape=true,
  backgroundcolor=\color{light-gray},
  basicstyle=\linespread{0.8}\ttfamily\small,
  keywordstyle=\color{blue}\ttfamily,
  stringstyle=\color{magenta}\ttfamily,
  commentstyle=\color{darkgray}\ttfamily,
  morecomment=[l][\color{magenta}]{\#},
  moredelim=**[is][\itshape]{@it}{@}, % this keeps keyword syntax
  moredelim=**[is][\bfseries]{@bf}{@}, % this keeps keyword syntax
  moredelim=**[is][\bfseries\itshape]{@ibf}{@}, % this keeps keyword syntax
  moredelim=**[is][\color{red!70!black}]{@R}{@},
%  moredelim=[is][\color{red!70!black}]{@@}{@@},
}

% Used for C++ code displaying, displaying important code 
% i.e. for the first time.
\lstdefinestyle{raycppdisplay}{
  captionpos=b,
  %  numbers=left, 
  numbers=none, 
  stepnumber=2, 
  frame=leftline,
%  framexleftmargin=5mm,
  xleftmargin=5mm,
  showstringspaces=false, 
  language=newcpp,
  mathescape=true,
  backgroundcolor=\color{light-gray},
  basicstyle=\linespread{1}\normalsize\bfseries\sffamily,
  keywordstyle=\color{blue}\normalsize\bfseries\sffamily,
  stringstyle=\color{red}\normalsize\bfseries\sffamily,
  commentstyle=\color{darkgray}\normalfont,
  morecomment=[l][\color{magenta}]{\#},
}



 % Used for C++ (small)
\lstdefinestyle{raycppsmall}{
  captionpos=b,
  %  numbers=left, 
  numbers=none, 
  stepnumber=2, 
  frame=single,
  showstringspaces=false, 
  language=newcpp,
  backgroundcolor=\color{light-gray},
  basicstyle=\linespread{0.8}\ttfamily\scriptsize,
  keywordstyle=\color{blue}\ttfamily,
  stringstyle=\color{red}\ttfamily,
  commentstyle=\color{darkgray}\ttfamily,
  morecomment=[l][\color{magenta}]{\#},
}

% Usage: \lstcppsmall{tagstart}{tagend}{inputfile}
\DeclareRobustCommand{\lstcppsmall}[3]{
\lstinputlisting[style=raycppsmall,
                 numbers=none,
                 rangeprefix=//\ ,% // startmarker //
                 rangesuffix=\ //,% // endmarker //
                 includerangemarker=false,
                 linerange=#1-#2]
{#3}}

\lstdefinestyle{raygensmall}{
        captionpos=b,
%        numbers=left, 
        numbers=none, 
        stepnumber=2, 
        frame=single,
        showstringspaces=false, 
%        language=bash,
        basicstyle=\linespread{0.8}\ttfamily\scriptsize,
        keywordstyle=\color{blue}\ttfamily,
        stringstyle=\color{red}\ttfamily,
        commentstyle=\color{gray}\ttfamily,
        morecomment=[l][\color{magenta}]{\#},}

% Used for input/output
\lstdefinestyle{rayio}{
  captionpos=b,
  %  numbers=left, 
  numbers=none, 
  stepnumber=2, 
  frame=leftline,
%  framexleftmargin=5mm,
  xleftmargin=5mm,
  showstringspaces=false, 
%  language=bash,
%  backgroundcolor=\color{light-gray},
  basicstyle=\linespread{1}\sffamily\footnotesize,
  keywordstyle=\color{blue}\sffamily,
  stringstyle=\color{red}\sffamily,
  commentstyle=\color{darkgray}\sffamily,
  morecomment=[l][\color{magenta}]{\#},
}
     

%%%%%%%%%%%%%%%%%%%%%%%%%%%%%%%%%%%%%%%%%%%%%%%%%%%%%%%%%%%%%%%%%%%%%%%%%%%%
%%%%%%%%%%%%%%%%%%%%%%%%%%%%%%%%%%%%%%%%%%%%%%%%%%%%%%%%%%%%%%%%%%%%%%%%%%%%

% cpp11 box
\newcommand*{\cppll}[1]{\colorbox{yellow!75}{\framebox{\strut C++ 11}}}


% Core box (person studying)
\newcommand*{\cppcore}[1]{\colorbox{yellow!75}{\framebox{\strut Core}}}

% Detail box (magnifying class)
\newcommand*{\cppdet}[1]{\colorbox{yellow!75}{\framebox{\strut Detailed}}}

%  Advanced (stack of books)
\newcommand*{\cppadv}[1]{\colorbox{yellow!75}{\framebox{\strut Advance}}}

%%%%%%%%%%%%%%%%%%%%%%%%%%%%%%%%%%%%%%%%%%%%%%%%%%%%%%%%%%%%%%%%%%%%%%%%%%%%
%%%%%%%%%%%%%%%%%%%%%%%%%%%%%%%%%%%%%%%%%%%%%%%%%%%%%%%%%%%%%%%%%%%%%%%%%%%%


%%%%%%%%%%%%%%%%%%%%%%%%%%%%%%%%%%%%%%%%%%%%%%%%%%%%%%%%%%%%%%%%%%%%%%%%%%%%
%%%%%%%%%%%%%%%%%%%%%%%%%%%%%%%%%%%%%%%%%%%%%%%%%%%%%%%%%%%%%%%%%%%%%%%%%%%%
%%%%%%%%%%%%%%%%%%%%%%%%%%%%%%%%%%%%%%%%%%%%%%%%%%%%%%%%%%%%%%%%%%%%%%%%%%%%
%%%%%%%%%%%%%%%%%%%%%%%%%%%%%%%%%%%%%%%%%%%%%%%%%%%%%%%%%%%%%%%%%%%%%%%%%%%%


% For psuedo code
\usepackage[linesnumbered,ruled,vlined,algochapter]{algorithm2e}
\usepackage{algpseudocode} % To use algorithmicx


%https://tex.stackexchange.com/questions/153646/algorithm2e-disabling-line-numbers-for-specific-lines
\let\oldnl\nl% Store \nl in \oldnl
\newcommand{\nonl}{\renewcommand{\nl}{\let\nl\oldnl}}% Remove line number for one line

%https://tex.stackexchange.com/questions/271661/algorithm-title-which-is-unnumbered-is-not-in-place
\makeatletter
\newcommand{\RemoveAlgoNumber}{\renewcommand{\fnum@algocf}{\AlCapSty{\AlCapFnt\algorithmcfname}}}
\newcommand{\RevertAlgoNumber}{\algocf@resetfnum}
\makeatother
%\RemoveAlgoNumber which removes \thealgocf from the caption printing; and
%\RevertAlgoNumber which reverts the removal.
% Usage:
% \RemoveAlgoNumber
% algo here
% \RevertAlgoNumber

%https://tex.stackexchange.com/questions/212301/do-while-loop-in-algorithm2e
\SetKwRepeat{Do}{do}{while}
%The above now allows you to use
%\Do{<end condition>}{<stuff>}

\SetKw{KwDownto}{downto}
\SetKw{KwError}{error}



%%%%%%%%%%%%%%%%%%%%%%%%%%%%%%%%%%%%%%%%%%%%%%%%%%%%%%%%%%%%%%%%%%%%%%%%%%%%
%%%%%%%%%%%%%%%%%%%%%%%%%%%%%%%%%%%%%%%%%%%%%%%%%%%%%%%%%%%%%%%%%%%%%%%%%%%%
%%%%%%%%%%%%%%%%%%%%%%%%%%%%%%%%%%%%%%%%%%%%%%%%%%%%%%%%%%%%%%%%%%%%%%%%%%%%
%%%%%%%%%%%%%%%%%%%%%%%%%%%%%%%%%%%%%%%%%%%%%%%%%%%%%%%%%%%%%%%%%%%%%%%%%%%%







%---------------------------------------------------------------------------
% biblatex with biber backend.
%---------------------------------------------------------------------------

\usepackage{csquotes}

% biblatex setup
\usepackage[backend=biber,
  citestyle=numeric,
  isbn=false,
  doi=false,
  url=true,% for websites, see below for removing it from papers etc.
  backref=true, % print out the page number of reference
]{biblatex}



%% biblatex setup
%\usepackage[backend=biber,
%  citestyle=numeric,
%  firstinits=true, % Only print initials of first names
%  isbn=false,
%  doi=false,
%  url=true,% for websites, see below for removing it from papers etc.
%  backref=true, % print out the page number of reference
%]{biblatex}


% clear urls for papers, proceedings, books
\AtEveryBibitem{%
  \ifentrytype{article}{%
    \clearfield{url}%
    \clearfield{urldate}%
    \clearfield{month}%
  }
  {}% no "else" operation
  %
  \ifentrytype{inproceedings}{%
    \clearfield{url}%
    \clearfield{urldate}%
    \clearfield{month}%
  }
  {}% no "else" operation
  % 
  \ifentrytype{book}{%
    \clearfield{url}%
    \clearfield{urldate}%
    \clearfield{month}%
  }
  {}% no "else" operation
  \ifentrytype{phdthesis}{%
    \clearfield{url}%
    \clearfield{urldate}%
    \clearfield{month}%
  }
  {}% no "else" operation
} % AtEveryBibitem

\renewbibmacro{in:}{%
  \ifentrytype{article}{}{\printtext{\bibstring{in}\intitlepunct}}}

% makes volume of journal bold and adds colon
\DeclareFieldFormat[article]{volume}{\textbf{#1}\addcolon\space}
% removes pagination (p./pp.) before page numbers
\DeclareFieldFormat{pages}{#1}

% Use BibTeX key as the cite key
% https://tex.stackexchange.com/questions/8428/use-bibtex-key-as-the-cite-key
\DeclareFieldFormat{labelalpha}{\thefield{entrykey}}
\DeclareFieldFormat{extraalpha}{}

\DefineBibliographyStrings{english}{%
    backrefpage  = {see p.}, % for single page number
    backrefpages = {see pp.} % for multiple page numbers
}

%%%%%%%%%%%%%%%%%%%%%%%%%%%%%%%%%%%%%%%%%%%%%%%%%%%%%%%%%%%%%%%%%%%%%%%%%%%%
%%%%%%%%%%%%%%%%%%%%%%%%%%%%%%%%%%%%%%%%%%%%%%%%%%%%%%%%%%%%%%%%%%%%%%%%%%%%















