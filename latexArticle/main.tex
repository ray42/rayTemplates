\documentclass[12pt,a4paper]{article}

% New versions of latex has deprecated \it and \tt with \itshape and
% \ttfamily. But some packages (such as lstlisting)  still use it. So we
% alias for compatibility
\renewcommand{\it}{\itshape}
\renewcommand{\tt}{\ttfamily}

%%%%%%%%%%%%%%%%%%%%%%%%%%%%%%%%%%%%%%%%%%%%%%%%%%%%%%%%%%%%%%%%%%%%%%%%%%%%
%%%%%%%%%%%%%%%%%%%%%%%%%%%%%%%%%%%%%%%%%%%%%%%%%%%%%%%%%%%%%%%%%%%%%%%%%%%%
%%% PACKAGES
%%%%%%%%%%%%%%%%%%%%%%%%%%%%%%%%%%%%%%%%%%%%%%%%%%%%%%%%%%%%%%%%%%%%%%%%%%%%
%%%%%%%%%%%%%%%%%%%%%%%%%%%%%%%%%%%%%%%%%%%%%%%%%%%%%%%%%%%%%%%%%%%%%%%%%%%%

%---------------------------------------------------------------------------
% Apply patches to other packages
%---------------------------------------------------------------------------
\usepackage{xpatch} % Currently not used.


%---------------------------------------------------------------------------
% Silence warnings
%---------------------------------------------------------------------------
% Silence warnings by hyper link
\usepackage{silence}


%---------------------------------------------------------------------------
% Colours
%---------------------------------------------------------------------------
%\usepackage{color} % loaded by xcolor

% If you are using tikz or pstricks package you must declare the xcolor 
% package before that, otherwise it will not work.
% http://ctan.org/pkg/xcolor
\usepackage[usenames,dvipsnames,svgnames,x11names]{xcolor}

%---------------------------------------------------------------------------
% Maths
%---------------------------------------------------------------------------

%http://tex.stackexchange.com/questions/32100/what-does-each-ams-package-do
%Most of the answer was extracted from the Introduction sections of the 
%documentation of amsmath and amsthm:
%\usepackage{amsmath}%AMS mathematical facilities for LATEX
\usepackage{mathtools} % mathtools loads/extends amsmath.
\usepackage{amsthm} % ams theorem
\usepackage{amssymb} % loads amsfonts

%https://tex.stackexchange.com/questions/232922/a-hack-for-getting-a-capital-weierstrass-p-in-order-to-represent-the-power-set
\usepackage{mathrsfs}
\usepackage[mathscr]{euscript}
\let\euscr\mathscr \let\mathscr\relax% just so we can load this and rsfs
\usepackage[scr]{rsfso}
\newcommand{\powerset}{\raisebox{.15\baselineskip}{\Large\ensuremath{\wp}}}
%Usage:
%$\mathcal{P}(X)$ 
%$\euscr{P}(X)$
%$\mathscr{P}(X)$
%$\powerset(X)$

% For boxes around equations
% https://tex.stackexchange.com/questions/20575/attractive-boxed-equations
% https://tex.stackexchange.com/questions/193242/how-to-make-a-box-around-an-equation-in-align-environment
\usepackage{empheq}
\usepackage[most]{tcolorbox}
%\begin{empheq}[box=\tcbhighmath]{equation*}
%    c_i = \langle\psi|\phi\rangle
%\end{empheq}



%1)amsmath provides miscellaneous enhancements for improving the information 
%structure and printed output of documents containing mathematical formulas.
%Some of the features provided by this package are:
%  The \DeclareMathOperator command (through the auxiliary package amsopn) 
%  to define new "operator name" commands analogous to \sin and \lim, 
%  including proper side spacing and automatic selection of the correct font
%  style and size (even when used in sub- or superscripts).
%
%  Multiple substitutes for the eqnarray environment to make various kinds 
%  of equation arrangements easier to write.
%
%  Equation numbers automatically adjust up or down to avoid overprinting on
%  the equation contents (unlike eqnarray).
%
%  Spacing around equals signs matches the normal spacing in the equation 
%  environment (unlike eqnarray).
%
%  A way to produce multiline subscripts as are often used with summation or
%  product symbols.
%
%  An easy way to substitute a variant equation number for a given equation
%  instead of the automatically supplied number.
%
%  An easy way to produce subordinate equation numbers of the form (1.3a) 
%  (1.3b) (1.3c) for selected groups of equations.
%
%  The \text command (through the auxiliary package amstext) for typesetting
%  a fragment of text inside a display.

%2)amsthm helps to define theorem-like structures; the introduction to the 
%documentation gives a nice concise description of the package:
%  The amsthm package provides an enhanced version of LaTeX's  \newtheorem 
%  command for defining theorem-like environments. The enhanced \newtheorem
%  recognizes a \theoremstyle specification (as in Mittelbach's theorem 
%  package) and has a * form for defining unnumbered environments. The
%  amsthm package also defines a proof environment that automatically adds a
%  QED symbol at the end. AMS document classes incorporate the amsthm 
%  package, so everything described here applies to them as well.
%
%  If the amsthm package is used with a non-AMS document class and with the
%  amsmath package, amsthm must be loaded after amsmath, not before.

%3)amssymb provides an extended symbol collection. For example, after 
%loading amssymb you have the following additional binary relation symbols:
%\barwedge, \boxdot, \boxminus, \boxplus, \boxtimes, \Cap, \Cup (and many 
%more), the arrow \leadsto, and some other symbols such as \Box and 
%\Diamond. Another useful feature is the \mathbb command to produce 
%blackboard bold characters.

% Since amssymb internally loads amsfonts, it's enough to load the former.




%%%%%%%%%%%%%%%%%%%%%%%%%%%%%%%%%%%%%%%%%%%%%%%%%%%%%%%%%%%%%%%%%%%%%%%%%%%%
%%%%%%%%%%%%%%%%%%%%%%%%%%%%%%%%%%%%%%%%%%%%%%%%%%%%%%%%%%%%%%%%%%%%%%%%%%%%


%---------------------------------------------------------------------------
% enumerate number per section
%---------------------------------------------------------------------------

\usepackage{enumitem}
% Note: This also allows us to do something like this:
%\begin{enumerate}[label=\FOO*]
% \alph* = a, \Alph* = A, \roman* = i, ii, etc...
% Put in parenthesis for parenthesis. I.e.
% (\alph*) = (a)
% Label: \alph, \Alph, \arabic, \roman and \Roman


%---------------------------------------------------------------------------
% url, hyperref, bookmark, etc...
%---------------------------------------------------------------------------

\WarningFilter{hyperref}{You have enabled option `breaklinks'.}

% url is loaded by hyperref, which is loaded by bookmark.

% The command \PassOptionsToPackage (used below) tells LaTeX to behave as 
% if its options were passed, when it finally loads a package. As you would 
% expect from its name, \PassOptionsToPackage can deal with a list of 
% options, just as you would have in the the options brackets of 
% \usepackage.

% Ordinarily, breaks are not allowed after “-” characters because this leads
% to confusion. (Is the “-” part of the address or just a hyphen?) The 
% package option “[hyphens]” allows breaks after explicit hyphen characters.
% The \url command will never ever hyphenate words.
% Also can use \path{}
\PassOptionsToPackage{hyphens,obeyspaces,spaces}{url}

% Allows link text to break across lines; since this cannot be accommodated 
% in PDF, it is only set true by default if the pdftex driver is used. This
% makes links on multiple lines into different PDF links to the same target.
\PassOptionsToPackage{breaklinks=true}{hyperref}

% http://tex.stackexchange.com/questions/167948/package-rerunfilecheck-warning-file-out-has-changed
% Or add \usepackage{bookmark}. Then a more modern implementation of the 
% bookmarks managing is used without .out file. The bookmarks are updated 
% earlier, thus in most cases only one LaTeX run is needed.
\usepackage{bookmark}

%% breakurl needs hyperref, but hyperref is loaded by bookmark!
\usepackage[anythingbreaks]{breakurl}

%% Internal links
\hypersetup{
colorlinks=true,
linkcolor=blue,
filecolor=magenta,      
urlcolor=cyan,
citecolor=cyan,
%pdfborder={0 0 0},      % No borders around internal hyperlinks
pdftitle={Raymon White Resume},
pdfauthor={Raymon White} % author
}
\urlstyle{same}



%---------------------------------------------------------------------------
% graphic, graphicx
%---------------------------------------------------------------------------

%% This is for includegraphics{}
% \usepackage{graphics} % Already loaded by graphicsx
% pdftex (default if compiling with pdflatex), if you are compiling with 
% pdftex to get a PDF that you will see with any PDF viewer.

% This gets an error, reason is below:
%\usepackage[pdftex]{graphicx}

% PDFLaTeX does not support EPS files. However, some modern LaTeX 
% distributions will try to automatically convert it to a PDF image, which 
% maybe this doesn't work in your case. It would be also possible that the 
% size information, namely the bounding box, in that EPS file is missing.
%
%If you are using latex, i.e. the DVI mode of LaTeX, which supports EPS then
%you should remove the incorrect pdftex option from the graphicx package. 
%The packages should be able to figure out the used driver by themselves, so
%you should avoid such options anyway. If you are using pdflatex then it 
%would be better to convert the EPS to a PDF file manually, e.g. using the 
%epstopdf tool (I'm not sure if it comes with the Windows version, but I 
%think so). You need to change the .eps extension to .pdf then, of course. 
%You can also drop the extension and LaTeX will look for files with the 
%given base name with all supported extensions, i.e. \includegraphics{file} 
%will use file.eps for latex and file.pdf (or file.png, file.jpg) for 
%pdflatex.

% TL;DR: Take out pdftex, and convert eps to pdf manually (although the
% engine should do it for us, if required)
\usepackage{graphicx}




\usepackage{titlesec}

% For lorem ipsum
\usepackage[english]{babel}
\usepackage{blindtext}
% usage:
%\blindtext
%\Blindtext



%\usepackage{setspace}
% Below are the spacing options.
%\doublespacing
%\singlespacing
%\onehalfspacing
%\setstretch{1.8}

%\newtheorem*{definition}{Definition}
%\newtheorem*{example}{Example}
%\newtheorem*{theorem}{Theorem}
%\newtheorem*{theorem*}{Theorem}

% This is the standardone inch all around border.
%\usepackage{fullpage}

% This is what I've made myself - max space used, for Tisseur cw.
% For most stand alone work, use this one.
%\setlength{\topmargin}{-1in}
%\setlength{\textheight}{9.75in}
%\setlength{\oddsidemargin}{-0.25in}
%\setlength{\textwidth}{7in}

% This is the geometry package... apparently good!
\usepackage[left=2cm,top=4cm,right=2cm,a4paper]{geometry}
\geometry{top=0.7in, bottom=0.7in, left=0.7in, right=0.7in}

\pagenumbering{gobble}

%\lstset{numbers=left, 
%        stepnumber=2, 
%        frame=single,
%        basicstyle=\footnotesize,
%        showstringspaces=false,
%        language=C++}

\raggedbottom
\raggedright
\setlength{\tabcolsep}{0in}

% Sections formatting
\titleformat{\section}{
  \vspace{-4pt}\Large\bfseries\raggedright
}{}{0em}{}[\color{black}\titlerule \vspace{-5pt}]

%Add horizontal line under each section heading
%\titleformat{\section}
%  {\normalfont\Large\bfseries}{}{1em}{}[{\titlerule[0.8pt]}]


%-------------------------
% Custom commands
\newcommand{\resumeItem}[2]{
  \item\small{
    \textbf{#1}{: #2 \vspace{-2pt}}
  }
}

\newcommand{\resumeSubheading}[4]{
  \vspace{-1pt}\item
    \begin{tabular*}{0.97\textwidth}{l@{\extracolsep{\fill}}r}
      \textbf{#1} & #2 \\
      \textit{\small#3} & \textit{\small #4} \\
    \end{tabular*}\vspace{-5pt}
}

\newcommand{\resumeSubItem}[2]{\resumeItem{#1}{#2}\vspace{-4pt}}

\renewcommand{\labelitemii}{$\circ$}

\newcommand{\resumeSubHeadingListStart}{\begin{itemize}[leftmargin=*]}
\newcommand{\resumeSubHeadingListEnd}{\end{itemize}}
\newcommand{\resumeItemListStart}{\begin{itemize}}
\newcommand{\resumeItemListEnd}{\end{itemize}\vspace{-5pt}}

\setlength{\parindent}{1em}
\setlength{\parskip}{1em}
%\renewcommand{\baselinestretch}{2.0}



% Underlined and bold and newlined
% the \phantomsection is for when we use label{rh:foo}
\DeclareRobustCommand{\rrheaderunderline}[1]{\par\medskip\phantomsection\noindent\textbf{\underline{#1}}\medskip}
\DeclareRobustCommand{\rrheader}[1]{\par\medskip\phantomsection\noindent{\normalsize\textbf{#1}}\medskip}
\DeclareRobustCommand{\rrheaderlarge}[1]{\par\medskip\phantomsection\noindent{\large\textbf{#1}}\medskip}
\DeclareRobustCommand{\rrheaderLarge}[1]{\par\medskip\phantomsection\noindent{\Large\textbf{#1}}\medskip}




%%%%%%%%%%%%%%%%%%%%%%%%%%%%%%%%%%%%%%%%%%%%%%%%%%%%%%%%%%%%%%%%%%%%%%%%%%%%
%%%%%%%%%%%%%%%%%%%%%%%%%%%%%%%%%%%%%%%%%%%%%%%%%%%%%%%%%%%%%%%%%%%%%%%%%%%%
%%Document begin
%%%%%%%%%%%%%%%%%%%%%%%%%%%%%%%%%%%%%%%%%%%%%%%%%%%%%%%%%%%%%%%%%%%%%%%%%%%%
%%%%%%%%%%%%%%%%%%%%%%%%%%%%%%%%%%%%%%%%%%%%%%%%%%%%%%%%%%%%%%%%%%%%%%%%%%%%
 
\begin{document}

%-----------HEADING---------------------------------------------------------
%\begin{tabular*}{\textwidth}{l@{\extracolsep{\fill}}l}
%\textbf{\Large Raymon White} & Email: raymonwhite42@gmail.com\\
%44 Canterbury Court, Gorefield Place, London, NW6 5SX & 
%Mobile: +4478-5149-9485\\
%\end{tabular*}


%%%%%%%%%%%%%%%%%%%%%%%%%%%%%%%%%%%%%%%%%%%%%%%%%%%%%%%%%%%%%%%%%%%%%%%%%%%%
%-----------Summary---------------------------------------------------------
%%%%%%%%%%%%%%%%%%%%%%%%%%%%%%%%%%%%%%%%%%%%%%%%%%%%%%%%%%%%%%%%%%%%%%%%%%%%
%\section{Summary}
%
%Numerical computation PhD graduate with +3 years experience writing
%numerical modelling software and proficient with C++98/11 and 14.

%\textbf{Notes/Questions:} 
%\textbf{(1) Note:} I'll write more here targeted
%to specific job specifications. 
%%%%
% FB: Yes, good plan. If you're including a personal profile, it should be
% very targetted. Personal profiles are not neccessary to include,
% especially when writing a covering letter as it is the covering letter
% that is the extended profile.

%\textbf{(2) Question:} Should I take this
%part our entirely? If I keep it in, should I mentioned that I took a year
%out to go travelling? I feel that I shouldn't (I've been reading examples
%online.)
%%%%FB If you have been travelling, you could say that on a covering letter
% or include it in an interests section on your CV. What did you do for the
% year? Were you just travelling or did you do any voluntary work? If you
% did the latter, include some details on your CV too.

%%%%%%%%%%%%%%%%%%%%%%%%%%%%%%%%%%%%%%%%%%%%%%%%%%%%%%%%%%%%%%%%%%%%%%%%%%%%
%-----------EDUCATION-------------------------------------------------------
%%%%%%%%%%%%%%%%%%%%%%%%%%%%%%%%%%%%%%%%%%%%%%%%%%%%%%%%%%%%%%%%%%%%%%%%%%%%
\section{Wikipedia Article}

\url{https://en.wikipedia.org/wiki/Schlumberger}

Schlumberger Limited is the world's largest oilfield services company.
Schlumberger employs approximately \textbf{100,000 people} representing more
than 140 nationalities working in more than 85 countries. Schlumberger has
four principal executive offices located in \textbf{Paris},
\textbf{Houston}, \textbf{London}, and \textbf{the Hague} (a city on the
western coast of the Netherlands and the capital of the province of South
Holland.).

Oilfield service companies--companies which provide services to the
petroleum exploration and production industry but do not typically produce
petroleum themselves. 

\textbf{Schlumberger subidiaries:}
\begin{itemize}[noitemsep,topsep=0pt]
\item Cameron International: a global provider of pressure control,
  processing, flow control and compression systems as well as project
  management and aftermarket services for the \textbf{oil and gas and
    process industries}.
\item Geophysical Service (GSI): purpose of using refraction and reflection
  seismology to explore for petroleum deposits.
\item Geoservices: an \textbf{upstream} (see below for explanation) oilfield
  service company founded in 1958 by Gaston Rebilly. It provides a range of
  skills that help evaluate hydrocarbon reservoirs and optimize field
  exploration, development and production. Geoservices provides in 3
  business segments: \textbf{(1)} Mud Logging (World No. 1), \textbf{(2)}
  Field Surveillance and \textbf{(2)} Well Intervention (World No. 2).
\item Smith International: Smith International was a Fortune 500 company
  headquartered in the Greenspoint district and in unincorporated Harris
  County, Texas. Smith International ceased to exist as an independent
  company following \textbf{the merger with Schlumberger}. This company
  supplies products to gas and oil production and exploration companies. The
  company used to be easily identified by its red \textbf{Sii} logo. The
  company had recently changed its logo to consist of the word "SMITH" in
  black capital letters with a green globe.
\item WesternGeco: WesternGeco is a geophysical services company. It is
  headquartered in the Schlumberger House on the property of London Gatwick
  Airport in Crawley, West Sussex, in Greater London. The company provides
  reservoir imaging, monitoring, and development services. The company, a
  business segment of Schlumberger, offers 3D and time-lapse seismic
  surveys, electromagnetic surveys, and multicomponent surveys for
  delineating prospects and reservoir management. It also provides
  geophysical, land and transition-zone acquisition, marine acquisition,
  electromagnetics, and data processing and reservoir seismic services. The
  company, formerly known as Western Geophysical Company, was founded in
  1933.
\end{itemize}

\textbf{Upstream/midstream and downstream:}

The oil and gas industry is usually divided into three major sectors:
upstream (or \textbf{exploration and production--E\&P}), midstream and
downstream.
\begin{itemize}[noitemsep,topsep=0pt]
\item Upstream: The upstream sector includes searching for potential
  underground or underwater crude oil and natural gas fields, drilling
  exploratory wells, and subsequently drilling and operating the wells that
  recover and bring the crude oil or raw natural gas to the surface.
  There has been a significant shift toward including unconventional gas as
  a part of the upstream sector, and corresponding developments in liquefied
  natural gas (LNG) processing and transport.

  \textbf{Business:} This categorization comes from value chain concepts,
  even before formal development Value Chain Management: \textbf{Oil Service
    Company:} A company that provides products and/or services to the oil
  and gas industry.  Usually a combination of labor, equipment, and/or other
  support services.  Examples include Baker Hughes, Haliburton, and
  Schlumberger.
\item The midstream sector involves the transportation (by pipeline, rail,
  barge, oil tanker or truck), storage, and wholesale marketing of crude or
  refined petroleum products. Pipelines and other transport systems can be
  used to move crude oil from production sites to refineries and deliver the
  various refined products to \textbf{downstream distributors}. Natural gas
  pipeline networks aggregate gas from natural gas purification plants and
  deliver it to \textbf{downstream customers}, such as local utilities.

  The midstream operations are often taken to include some elements of the
  upstream and downstream sectors. For example, the midstream sector may
  include natural gas processing plants that purify the raw natural gas as
  well as removing and producing elemental sulfur and natural gas liquids
  (NGL) as finished end-products.
\item The downstream sector is the refining of petroleum crude oil and the
  processing and purifying of raw natural gas, as well as the marketing
  and distribution of products derived from crude oil and natural gas. The
  downstream sector reaches consumers through products such as gasoline or
  petrol, kerosene, jet fuel, diesel oil, heating oil, fuel oils,
  lubricants, waxes, asphalt, natural gas, and liquefied petroleum gas (LPG)
  as well as hundreds of petrochemicals.

  Midstream operations are often included in the downstream category and are
  considered to be a part of the downstream sector.
\end{itemize}

Schlumberger was \textbf{founded in 1926} by brothers Conrad and Marcel
Schlumberger from the Alsace region in France as the Soci\'et\'e de
prospection \'electrique (French: Electric Prospecting Company). The company
recorded the first-ever electrical resistivity well log in
Merkwiller-Pechelbronn, France in 1927. Today Schlumberger supplies the
petroleum industry with services such as \textbf{seismic acquisition} and
\textbf{processing}, \textbf{formation evaluation}, \textbf{well testing}
and \textbf{directional drilling}, \textbf{well cementing} and
\textbf{stimulation}, \textbf{artificial lift}, \textbf{well completions},
\textbf{flow assurance} and \textbf{consulting}, and \textbf{software and
  information management}. The company is also involved in the groundwater
extraction and carbon capture and storage industries.

The brothers had experience conducting geophysical surveys in countries such
as Romania, Canada, Serbia, South Africa, the Democratic Republic of the
Congo and the United States. The new company sold electrical-measurement
mapping services, and recorded the first-ever electrical resistivity well
log in Merkwiller-Pechelbronn, France in 1927. The company quickly expanded,
logging its first well in the U.S. in 1929, in Kern County, California. In
1935, the Schlumberger Well Surveying Corporation was founded in Houston,
later evolving into Schlumberger Well Services, and finally Schlumberger
Wireline \& Testing. Schlumberger invested heavily in research, inaugurating
the Schlumberger-Doll Research Center in Ridgefield, Connecticut in 1948,
contributing to the development of a number of new logging tools. In 1956,
Schlumberger Limited was incorporated as a holding company for all
Schlumberger businesses, which by now included American testing and
production company Johnston Testers.

Over the years, Schlumberger continued to expand its operations and
acquisitions. In \textbf{1960}, Dowell Schlumberger (50\% Schlumberger, 50\%
Dow Chemical), which specialized in pumping services for the oil industry,
was formed. In \textbf{1962}, Schlumberger Limited became listed on the New
York Stock Exchange. That same year, \textbf{Schlumberger purchased
  Daystrom}, an electronic instruments manufacturer in South Boston,
Virginia which was \textbf{making furniture by the time the division was
  sold to Sperry \& Hutchinson in 1971}. Schlumberger purchased 50\% of
Forex in 1964 and merged it with 50\% of Languedocienne to create the
Neptune Drilling Company. The first computerized reservoir analysis,
SARABAND, was introduced in 1970. The remaining 50\% of Forex was acquired
the following year; Neptune was renamed Forex Neptune Drilling Company. In
1979, Fairchild Camera and Instrument (including Fairchild Semiconductor)
became a subsidiary of Schlumberger Limited.

In 1981, Schlumberger established the first international data links with
e-mail. \textbf{In 1983, Schlumberger opened their Cambridge Research Center
  in Cambridge, England and in 2012 it was renamed the Schlumberger Gould
  Research Center after the company's former CEO Andrew Gould.}

The SEDCO drilling rig company and half of Dowell of North America were
acquired in 1984, resulting in the creation of the Anadrill drilling
segment, a combination of Dowell and The Analysts' drilling segments. Forex
Neptune was merged with SEDCO to create the Sedco Forex Drilling Company the
following year, when Schlumberger purchased Merlin and 50\% of GECO.

In 1987, Schlumberger completed their purchases of Neptune (North America),
Bosco and Cori (Italy), and Allmess (Germany). That same year, National
Semiconductor acquired Fairchild Semiconductor from Schlumberger for \$122
million. In 1991, Schlumberger acquired PRAKLA-SEISMOS, and pioneered the
use of geosteering to plan the drill path in horizontal wells.

\textbf{Schlumberger acquired software company GeoQuest Systems in 1992.}
With the purchase came the conversion of SINet to TCP/IP and www capability.
In the 1990s Schlumberger bought out the petroleum division, AEG meter, and
ECLIPSE reservoir study team Intera Technologies Corp. A joint venture
between Schlumberger and Cable \& Wireless resulted with the creation of
Omnes, which then handled all of Schlumberger's internal IT business.
Oilphase and Camco International were also purchased.

\textbf{In 1999, Schlumberger and Smith International created a joint venture, M-I
L.L.C., the world's largest drilling fluids (or mud) company.} The company
consists of 60\% Smith International, and 40\% Schlumberger. Since the joint
venture was prohibited by a 1994 antitrust consent decree barring Smith from
selling or combining their fluids business with certain other companies,
including Schlumberger, the U.S. District Court in Washington, D.C. found
Smith International Inc. and Schlumberger Ltd. guilty of criminal contempt
and fined each company \$750,000 and placed each company on five years
probation. Both companies also agreed to pay a total of \$13.1 million,
representing a full disgorgement of all of the joint venture's profits
during the time the companies were in contempt.[17]

In 2000, the Geco-Prakla division was merged with Western Geophysical to
create the seismic contracting company WesternGeco, of which Schlumberger
held a 70\% stake, the remaining 30\% belonging to competitor Baker Hughes.
Sedco Forex was spun off, and merged with Transocean Drilling company in
2000.

In 2001, Schlumberger acquired the IT consultancy company Sema plc for \$5.2
billion. The company was an Athens 2004 Summer Olympics partner, but
Schlumberger's venture into IT consultancy did not pay off, and
divestiture\footnote{the action or process of selling off subsidiary
  business interests or investments.} of Sema to Atos Origin was completed
that year for \$1.5 billion. The cards division was divested through an IPO
to form Axalto, which later merged with Gemplus to form Gemalto, and the
Messaging Solutions unit was spun off and merged with Taral Networks to form
Airwide Solutions. In 2003, the Automated Test Equipment group, part of the
1979 Fairchild Semiconductor acquisition, was spun off to NPTest Holding,
which later sold it to Credence.

In 2004, Schlumberger Business Consulting was launched. Based in Paris, it
is the company's management consultancy arm.

In 2005, Schlumberger purchased Waterloo Hydrogeologic, which was followed
by several other groundwater industry related companies, such as Westbay
Instruments, and Van Essen Instruments. Also that year, Schlumberger
relocated its U.S. corporate offices from New York to Houston.

In 2006, Schlumberger purchased the remaining 30\% of WesternGeco from Baker
Hughes for US\$2.4 billion. Also that year, the Schlumberger-Doll
Research Center was relocated to a newly built research facility in
Cambridge, Massachusetts to replace the Ridgefield, Connecticut research
center. The facility joins the other research centers operated by the
company in Cambridge, England; Moscow, Russia; Stavanger, Norway; and
Dhahran, Saudi Arabia.

In 2010, the acquisition of Smith International in an all-stock deal valued
at \$11.3 billion was announced. The sale price is 45.84-a-share price was
37.5 percent higher than Smith closing price on 18 February 2010. The deal
is the biggest acquisition in Schlumberger history. The merger was
completed on August 27, 2010. Also announced in 2010 were Schlumberger
plans to acquire Geoservices, a French-based company specializing in energy
services, in a deal valued at \$1.1 billion, including debt.

In 2014, Schlumberger announced the purchase of the remaining shares of SES
Holdings Limited (``Saxon''), a Calgary-based provider of international land
drilling services, from First Reserve and certain members of Saxon
management. The transaction is subject to customary closing conditions,
including the receipt of regulatory approvals. Schulmberger had a minority
share in Saxon previously.

In 2015, Schlumberger was indicted by the US Department of Justice for
sanction violations of conducting business in Iran and Sudan; the company
was fined \$233 million, amounting to the largest fine for sanctions to
date.

Due to a downturn in the global oil \& gas industry in 2015, Schlumberger
announced 21,000 layoffs accounting for 15\% of their total workforce.

In August 2015, Schlumberger agreed to acquire oilfield equipment
manufacturer Cameron International for \$14.8 billion.



%%%%%%%%%%%%%%%%%%%%%%%%%%%%%%%%%%%%%%%%%%%%%%%%%%%%%%%%%%%%%%%%%%%%%%%%%%%%
%-----------EDUCATION-------------------------------------------------------
%%%%%%%%%%%%%%%%%%%%%%%%%%%%%%%%%%%%%%%%%%%%%%%%%%%%%%%%%%%%%%%%%%%%%%%%%%%%
\section{Schlumberger Technology}


\subsection{From job spec:}

\textbf{Company description:}

Schlumberger (\url{www.slb.com}) is the world's leading supplier of
technology, integrated project management and information solutions to
customers working in the oil and gas industry worldwide.  

Schlumberger is leading a digital transformation of the oil and gas industry
to enhance real-time global collaboration, operational efficiency, and the
integration of data, expertise, and technology information. Our technology
services and solutions translate acquired data into useful information that
improves decision-making--anytime, anywhere.

Our team of software domain experts \textbf{invents}, \textbf{designs}, and
applies disruptive cutting-edge technologies to enable our customers to
increase \textbf{reliability}, \textbf{efficiency}, and
\textbf{integration}. Throughout, we maintain \textbf{agile methodology} and
\textbf{skilled development processes}.

As a Schlumberger Software Engineer you'll have responsibility
\textbf{throughout the full development cycle}. You'll apply your skills in
computer science, engineering, and mathematics to \textbf{design},
\textbf{develop}, and \textbf{test software}. Whether you're creating new
products in our transformational programs or enhancing existing ones, you'll
have the opportunity to listen to what each customer needs and bring all the
elements together to provide solutions.

Currently, we have the \textbf{fifth largest super computer infrastructure
  in the world}, comprising more than \textbf{65 petaflops of processing
  power}. You will join a team of \textbf{software experts} with a variety
of core competencies such as Internet of things (IoT), user experience (UX),
data analytics, and web and mobile front-end development. Keeping pace with
and applying the \textbf{latest digital technology trends} in the oil and
gas industry will be your main focus.

\textbf{Role Specification:}

A software developer is required to strengthen the \textbf{commercial
  software development group}. This team works on the modeling of
\textbf{hydrocarbon reservoirs} and on the \textbf{development of end-user
  workflows} involving the simulation of production from an oilfield asset.
The position involves work spanning the whole product development life
cycle, \textbf{enhancing and extending the functionality of our commercial
  products}. These products are market leaders with an established
user-base, and the \textbf{development team has been steadily growing over
  the past few years}.

The successful applicant will be required to collaborate with other groups,
both internal and external to the organization; hence a good team worker is
required for this role. Strong software skills are required to develop high
quality technical applications.

\textbf{Person Specification:}
\begin{itemize}%[noitemsep,topsep=0pt]
\item A proven ability to solve complex mathematical/scientific problems
\item Mathematical modeling
\item C++ or related programming language 
\item Object Oriented Design
\item Good communication skills
\end{itemize}

\textbf{Exposure to any of the following would be advantageous:}
\begin{itemize}%[noitemsep,topsep=0pt]
\item Commercial software development 
\item HPC \& Parallel programming
\item Cross-platform development (Linux and Windows)
\end{itemize}

Qualifications:
\begin{itemize}%[noitemsep,topsep=0pt]
\item BSc/MSc or PhD in Mathematics, Science or Engineering 
\item Fresh-out or with industry/academic experience
\end{itemize}

\subsection{From their website: Schlumberger Abingdon Technology Center}

\url{http://www.slb.com/about/rd/technology/abtc.aspx}

Abingdon Technology Center (AbTC) is the Schlumberger \textbf{Center of
  Excellence for Reservoir Engineering}. It develops software that enables
oil and gas companies to \textbf{make better decisions for reservoir
  development}.  \textbf{Simulation} helps \emph{quantify} \textbf{reserves}
and \textbf{distribution} within the formation rocks and to predict changes
over the life of the reservoir. \textbf{Production decisions} involve where
and how to drill and operate wells and how to most safely and efficiently
bring oil and gas to the surface. Related projects include CO2
capture-and-storage technology, water management, and reservoir
geomechanics. Petroleum and reservoir specialists are involved in product
commercialization, user training as part of Schlumberger Technical Services,
and worldwide product support. Schlumberger Information Solutions is ISO
9001:2000-certified.

AbTC works closely with other Schlumberger research and technology centers
and segments. Its specialists directly support internal and external users
worldwide and communicate with other support engineers through the
Schlumberger InTouchSupport.com online support and knowledge management
system. This global connection can help customers analyze problems
encountered in the field, identify solutions, and improve future operations.

\rrheader{Employee collaboration and global knowledge sharing}

Through collaboration with universities in the UK, USA, and France, as well
as an internship program, AbTC promotes relationships with universities
around the globe. Employees at all levels benefit from a continuous training
program, tailored career management, extensive electronic library
facilities, and a global structure that supports knowledge sharing and
teamwork. Management encourages patent applications, conference
participation, and journal publications.


\subsection{EPMAG: New Technology, M\&A Steer Investment For Schlumberger}


\url{https://www.epmag.com/new-technology-ma-steer-investment-schlumberger-846536#p=full}

By Velda Addison, Hart Energy Monday, April 25, 2016--1:47pm

Despite double-digit revenue declines in just about all parts of the world,
Schlumberger (NYSE: SLB) is \textbf{still adding to its technological
  arsenal} with new drilling and hydraulic fracturing systems in the works
while investing in \textbf{niche technologies and multiclient seismic
  surveys}.

Hopes are that its offerings coupled with a broader global reach and other
strategic moves will create the leverage needed to grow revenue market share
among other goals, according to Schlumberger executives speaking during an
April 22 conference call on the company's first-quarter 2016 results.

``So far in this downturn, we have closed our largest ever acquisition, made
a series of smaller but still significant investments in \textbf{specific
  technology niches}, and invested in \textbf{integrated services and
  projects} that will boost our financial performance going forward,''
Schlumberger CEO Paal Kibsgaard said April during the conference call.

But if all goes as planned, in 2017 Schlumberger will have also produced a
new hydraulic fracturing system, which Kibsgaard said will ``span the
complete range of surface components, such as \textbf{Cameron's CAMSHALE
  pressure control and wellhead systems} together with our own perforating,
fracturing, cleanup, and flowback services, as well as our latest downhole
completion technology and fracturing fluids.'' Plus, with newly acquired
Cameron International Corp.---following the merger's closure April
1--Schlumberger plans to have its ``land drilling system of the future''
available commercially.

The system will integrate purpose-built surface and downhole hardware into a
drilling system managed by \textbf{optimized software}, targeting
operational efficiency gains. The system, Kibsgaard said, will draw on the
rig design acquisition and rig manufacturing joint venture made in 2015.
Plans include having five engineering prototypes of the new system ready for
field testing in Ecuador this year.

Schlumberger's continued drive comes as the oil and gas market endures tough
times---low commodity prices brought on by a supply-demand imbalance that
has, in turn, sucked dollars from profit margins and hundreds of thousands
of jobs off payrolls and driven most, \textbf{if not all, to seek the
  benefits of technology}.

First-quarter 2016 revenue plummeted 36\% year-on-year for Schlumberger to
\$6.5 billion. The figure was down 16\% from the previous quarter. In North
America, operating expenses came in higher than the cost of goods sold,
sending the company's pretax operating income into negative territory as the
its overall headcount shrunk by 8,000{} in the first quarter.

Schlumberger also shaved \$400 million off its initial capex guidance. The 
budget is now about \$2 billion, which includes funds for the two new 
systems.

``Our ability to invest through this unprecedented industry downturn on the 
back of our strong cash flow and balance sheet enables us to capitalize on 
the current market conditions to gain significant relative strength compared 
to our surroundings,'' Kibsgaard said.

Other investment areas highlighted by Kibsgaard include:

\begin{itemize}%[noitemsep,topsep=0pt]
\item Multiclient seismic surveys. Focus is currently on Mexico's Campeche
  Basin and offshore Mozambique and South Africa. Schlumberger currently has
  eight 3-D vessels active;
\item Production management. Schlumberger Production Management (SPM)
  continues work to rejuvenate the Shushufindi oil field in Ecuador and has
  started a new SPM project at the Auca Field, also in Ecuador, and
\item M\&A investment. In addition to completing its nearly \$15 billion
  Cameron deal, Schlumberger acquired two U.K.-based companies in
  first-quarter 2016---the Meta Downhole Ltd. engineering and service
  company that offers downhole metal-to-metal isolation technology in well
  integrity applications and Asset Development \& Improvement Ltd., an oil
  and gas consultancy. The acquisitions were announced in March.
\end{itemize}

But attention on investment does not mean Schlumberger as changed its view
of the oil markets.

The company expects market conditions to worsen in the second quarter,
Kibsgaard said. However, for what it's worth, he added that E\&P investment
cuts are already so severe that they can ``only accelerate production
decline and the consequent upward movement in oil price.''

Kibsgaard added, ``We remain convinced that the tightening of the supply
(and) demand balance is well underway. And while the operating environment
remains tough, the market presents a range of opportunities which we will
continue to actively pursue.''


\subsection{Fortune: Schlumberger heats up Big Oil's tech war}

CYRUS SANATI September 8, 2015

\url{http://fortune.com/2015/09/08/big-oil-schlumberger/}

The technology war bubbling inside the energy industry is starting to heat
up.

Schlumberger's recent acquisition of oilfield equipment manufacturer Cameron 
International for \$15 billion is the latest in a string of high-profile 
mergers in the energy space, such as Halliburton's pending purchase of Baker
Hughes, in which technology, not just the oil price, was one of the major 
motivating factors behind the deal.

The Cameron deal centers on \textbf{Schlumberger's desire to create a
  proprietary technology} platform to standardize the disparate services and
equipment along the energy value chain. The company hopes its new technology
ecosystem will help it win new customers by offering turnkey drilling
solutions and retaining the old ones by locking them into a closed system.

Much has been said about how advances in fracking ``technology'' paved the
way for the shale oil boom in the U.S. of the last few years. But it isn't
as if there was anything new or particularly novel about fracking; indeed,
the practice (which basically consists of shooting liquid underground to
break up rock beds) has been around for decades. No, it was high oil prices
that allowed fracked wells to profitably produce oil and gas in places where
it was once cost-prohibitive.

During that time, producers and the wildcatters drilling the wells got rich,
but so did the big oilfield services companies, such as Schlumberger (SLB,
+0.08\%), Halliburton (HAL, +1.28\%), and Baker Hughes (BHI, +0.00\%), along
with equipment manufacturers and drillers including Cameron (CAM, +0.00\%),
Wood Group, FMC Technologies (FTI, +1.09\%), National-Oilwell Varco (NOV,
+0.18\%), and Dril-Quip (DRQ, +0.74\%).

But as oil prices weakened over the last year, so too has demand for
oilfield services and equipment. It is now cost-prohibitive to drill in many
locations, especially for wells operating offshore. When oil was well over
\$100 a barrel, oilfield services, equipment, and rigs were in short supply
as producers bid up their prices. Now service providers are fighting to win
contracts, and that translates to \textbf{lower prices, smaller margins, and
  greater uncertainty for their sector}.

But while a few companies have gone under, most are still up and pumping
away. That's because producers hedged or sold their production forward at
higher prices, thus allowing them to preserve fat margins in a low oil price
environment. Oilfield service providers have also lowered their prices to
match falling oil. Remember last fall when industry watchers were claiming
that U.S. producers needed oil to be around \$80 to break even? Well, that
was only true because the cost to drill was so ridiculously high. But with
service providers and equipment manufacturers lowering their prices,
producers were able to keep drilling right-side up.

This won't last forever. Producers will eventually start to see their hedges
roll off, leaving them totally exposed to the weak oil price. To continue
drilling, \textbf{they'll need even greater discounts from their buddies in
  oilfield services}. That means lower margins and less profit, and in many
cases, it means operating at a loss.

Oilfield service providers have been prepping for this scenario for the
better part of a year. Hiring has stopped, bonuses and salaries have been
cut, jobs have been eliminated, and fancy perks, like trips on private
planes, have been discouraged. \textbf{\emph{But being the cheapest operator
    only goes so far---you need to give your clients a little something
    extra---something they can't get from the other guy}}.

Schlumberger believes that little something extra will take the form of what
the company is calling ``\textbf{\emph{pore-to-pipeline}}'' coverage for
operators. This is where Cameron comes in. As a provider of oilfield
equipment, Cameron will provide the parts needed for Schlumberger to offer
this new comprehensive coverage. The goal is to lower the cost and
complications associated with drilling, providing clients with greater value
and less headache.

But it's not just about eliminating a middleman in the oil services value
chain. Sure, there are cost savings to be had here, but that's not enough to
move the needle. \textbf{\emph{The real value is in the technology behind
    the integration}}.

\textbf{\emph{Computer engineers inside Schlumberger, who did not wish to be
    named as they are not authorized to speak to the media, tell Fortune
    that the company has been busy designing a proprietary operating system
    that will bring all the disparate parts of the oil and gas value chain
    onto a single technology platform.}} (DELFI?) This new operating system
will be able to analyze everything happening on a rig in real time, boosting
both efficiency and safety. It also serves to lock customers into a single
ecosystem that the company hopes will act as an incentive to use
Schlumberger services and (now) equipment for all their drilling needs.

``It is similar to an operating system used on your mobile phone,'' Doug
Sheridan, the founder of EnergyPoint Research, which conducts satisfaction
surveys, ratings, and reports for the energy industry, tells Fortune. ``The
question is whether Schlumberger will make this system open source, like
Google's Android, or closed, like Apple's iOS''.

A lack of technology standardization was one of the reasons behind the
Deepwater Horizon disaster in 2010. The U.S. Government concluded that the
explosion on the rig, which killed 11 workers and caused millions of barrels
of oil to spew into the Gulf of Mexico, was due to a combination of six
operations, tests, and equipment failures. Such a catastrophic failure
across the board may have been noticed by engineers well in advance if all
the disparate components on the rig were actually able to talk to each
other.

So can Schlumberger sell customers on the benefits of its new ecosystem, or
is its single platform a bad bet that customers will change the way they do
business? Old-school operators in the energy space are all about going with
``the best in breed'' equipment and service providers. They have deep
relationships with salesmen in certain companies and tend to be very loyal
to them, so it will be hard to sell many of them on the idea of buying from
just one source.

But the benefits of having all (or most) of your equipment operate on a
single (relatively) easy-to-use platform may outweigh the marginal benefit
of going with the best in breed for every piece of equipment on a rig.
National oil companies, which usually don't have a clue about what they're
doing, could benefit greatly from such a package deal and might actually
even pay a premium for it.

Schlumberger hopes to win a great deal of business by offering bundled
service packages, according to people with knowledge of the company's plans.
This will allow it to charge a premium for some services and offer discounts
on others. The question then is, \textbf{\emph{How good will this platform
    be}}, and can it live up to the safety and efficiency expectations of
the market? We probably won't have the answer in the next few months or even
the next year. It is clearly a long-term plan, which will take time to
execute. But once the system goes live, it may be unstoppable.

\subsection{DELFI}

\url{https://www.software.slb.com/delfi}

\rrheader{A new E\&P environment for the evolving world}

Market dynamics have challenged the oil and gas industry to evolve. Digital
technology now stands to bring its own disruption to our industry. It is
fundamentally changing how we work---to enhance performance and enable
unmatched value creation along the way.

Schlumberger is delivering the \textbf{DELFI cognitive E\&P environment} for
this evolving world. It is a multidimensional environment that unites
planning and operations. Bringing together advances in technical disciplines
such as \textbf{artificial intelligence}, \textbf{data analytics}, and
\textbf{automation}---underpinned by decades of unrivaled domain
knowledge---the result is an E\&P experience like no other.

\rrheader{Everything working together, continuously}

The DELFI environment defines a new standard in cross-discipline
collaboration. For the first time in one place, users can see and understand
the connections and dynamic effects from exploration to abandonment. More
than individuals collaborating, the DELFI environment is everything working
together continuously: teams, systems, software, legacy data, and live
inputs---all feeding into a unified environment that grows to become greater
than the sum of its parts.

The environment amplifies the individual capabilities and expertise of
people working at each stage of the hydrocarbon life cycle. Whether
explorationist, driller, or production engineer, the DELFI environment
augments every connection and interaction, enabling you to arrive at better
outcomes sooner.

\rrheader{Built on deep science to automate and accelerate complex
  functions}

At its heart, the DELFI environment utilizes powerful software built on deep
science to automate and accelerate complex functions such as modeling,
simulation, analysis, and forecasting. These advanced computational
capabilities have been developed using insights from lab and field
measurements, drawing on datasets across a range of diverse sources from
flow assurance studies to fiscal libraries. It is this scientific dedication
that ensures the engines driving the environment will always be
fit-for-purpose and best-in-class.

Trusted to solve the industry's most complex challenges, the DELFI cognitive
E\&P environment is supported by the collective knowledge of the world's
largest body of petrotechnical experts. Now the DELFI environment's
cognitive approach amplifies the capabilities of every expert; enabling you
to automate tasks, use learning systems, interrogate richer data sources,
and deliver better and faster decisions.

\rrheader{Why the DELFI environment?}

``If you want your people to work in one space, if you want limitless
collaboration, if you want to use your own IP, if you want to benefit from
all the best-in-class technology in the industry---then go to the DELFI
environment.''



\section{Why I want to work at Schlumberger}


%%% My own:
Being the world's largest oilfield service company, Schlumberger will offer
lots of challengers and therefore opportunities for me to develop
professionally. But what attracted me to Schlumberger the most is the fact
that even during a downturn in the global oil \& gas industry, Schlumberger
continued to invest in technology.

This shows foresight and this is what will give Schlumberger a competitive
edge, as exploiting new technology is what will shorten the project time and
lower cost for our costumers. For example, like the DELFI cognitive E\&P
environment which provides a unified and simplistic technology solution
cross exploration and production teams.

This technology opens up opportunities to work on big data analytics,
machine learning and high performance computing, possibly combining all
three. These are all areas which I am interested in and I hope to have a
chance to work with cutting edge technology and interesting real-world
engineering problems.

The ability to innovate through technology is what sets Schlumberger apart
from the competition, and this is what I want to be a part of, this is what
I want to contribute to, this is why I want to work at Schlumberger.


\section{What can I offer Schlumberger}
 
\begin{itemize}%[noitemsep,topsep=0pt]
\item I have a proven ability to solve complex mathematical/scientific 
  problems, as evident during my PhD project.
\item I have exposure to mathematical modeling
\item I am proficient with C++ 98/11 and 14.
\item I am aware of the principles of Object Oriented Design: SOLID
  \begin{itemize}%[noitemsep,topsep=0pt]
  \item \textbf{Single responsibility principle:} a class should have only a
    single responsibility (i.e. changes to only one part of the software's
    specification should be able to affect the specification of the class).
  \item \textbf{Open/closed principle:} ``software entities ... should be
    open for extension, but closed for modification.''
  \item \textbf{Liskov substitution principle:} ``objects in a program
    should be replaceable with instances of their subtypes without altering
    the correctness of that program.'' See also design by contract.
  \item \textbf{Interface segregation principle:} ``many client-specific
    interfaces are better than one general-purpose interface.''
  \item \textbf{Dpendency inverse principle:} one should “depend upon
    abstractions, not concretions.
  \end{itemize}
\item Good communication skills
\end{itemize}





 
 \end{document}
