\documentclass[12pt,a4paper]{article}

% New versions of latex has deprecated \it and \tt with \itshape and
% \ttfamily. But some packages (such as lstlisting)  still use it. So we
% alias for compatibility
\renewcommand{\it}{\itshape}
\renewcommand{\tt}{\ttfamily}

%%%%%%%%%%%%%%%%%%%%%%%%%%%%%%%%%%%%%%%%%%%%%%%%%%%%%%%%%%%%%%%%%%%%%%%%%%%%
%%%%%%%%%%%%%%%%%%%%%%%%%%%%%%%%%%%%%%%%%%%%%%%%%%%%%%%%%%%%%%%%%%%%%%%%%%%%
%%% PACKAGES
%%%%%%%%%%%%%%%%%%%%%%%%%%%%%%%%%%%%%%%%%%%%%%%%%%%%%%%%%%%%%%%%%%%%%%%%%%%%
%%%%%%%%%%%%%%%%%%%%%%%%%%%%%%%%%%%%%%%%%%%%%%%%%%%%%%%%%%%%%%%%%%%%%%%%%%%%

%---------------------------------------------------------------------------
% Apply patches to other packages
%---------------------------------------------------------------------------
\usepackage{xpatch} % Currently not used.


%---------------------------------------------------------------------------
% Silence warnings
%---------------------------------------------------------------------------
% Silence warnings by hyper link
\usepackage{silence}


%---------------------------------------------------------------------------
% Colours
%---------------------------------------------------------------------------
%\usepackage{color} % loaded by xcolor

% If you are using tikz or pstricks package you must declare the xcolor 
% package before that, otherwise it will not work.
% http://ctan.org/pkg/xcolor
\usepackage[usenames,dvipsnames,svgnames,x11names]{xcolor}

%---------------------------------------------------------------------------
% Maths
%---------------------------------------------------------------------------

%http://tex.stackexchange.com/questions/32100/what-does-each-ams-package-do
%Most of the answer was extracted from the Introduction sections of the 
%documentation of amsmath and amsthm:
%\usepackage{amsmath}%AMS mathematical facilities for LATEX
\usepackage{mathtools} % mathtools loads/extends amsmath.
\usepackage{amsthm} % ams theorem
\usepackage{amssymb} % loads amsfonts

%https://tex.stackexchange.com/questions/232922/a-hack-for-getting-a-capital-weierstrass-p-in-order-to-represent-the-power-set
\usepackage{mathrsfs}
\usepackage[mathscr]{euscript}
\let\euscr\mathscr \let\mathscr\relax% just so we can load this and rsfs
\usepackage[scr]{rsfso}
\newcommand{\powerset}{\raisebox{.15\baselineskip}{\Large\ensuremath{\wp}}}
%Usage:
%$\mathcal{P}(X)$ 
%$\euscr{P}(X)$
%$\mathscr{P}(X)$
%$\powerset(X)$

% For boxes around equations
% https://tex.stackexchange.com/questions/20575/attractive-boxed-equations
% https://tex.stackexchange.com/questions/193242/how-to-make-a-box-around-an-equation-in-align-environment
\usepackage{empheq}
\usepackage[most]{tcolorbox}
%\begin{empheq}[box=\tcbhighmath]{equation*}
%    c_i = \langle\psi|\phi\rangle
%\end{empheq}



%1)amsmath provides miscellaneous enhancements for improving the information 
%structure and printed output of documents containing mathematical formulas.
%Some of the features provided by this package are:
%  The \DeclareMathOperator command (through the auxiliary package amsopn) 
%  to define new "operator name" commands analogous to \sin and \lim, 
%  including proper side spacing and automatic selection of the correct font
%  style and size (even when used in sub- or superscripts).
%
%  Multiple substitutes for the eqnarray environment to make various kinds 
%  of equation arrangements easier to write.
%
%  Equation numbers automatically adjust up or down to avoid overprinting on
%  the equation contents (unlike eqnarray).
%
%  Spacing around equals signs matches the normal spacing in the equation 
%  environment (unlike eqnarray).
%
%  A way to produce multiline subscripts as are often used with summation or
%  product symbols.
%
%  An easy way to substitute a variant equation number for a given equation
%  instead of the automatically supplied number.
%
%  An easy way to produce subordinate equation numbers of the form (1.3a) 
%  (1.3b) (1.3c) for selected groups of equations.
%
%  The \text command (through the auxiliary package amstext) for typesetting
%  a fragment of text inside a display.

%2)amsthm helps to define theorem-like structures; the introduction to the 
%documentation gives a nice concise description of the package:
%  The amsthm package provides an enhanced version of LaTeX's  \newtheorem 
%  command for defining theorem-like environments. The enhanced \newtheorem
%  recognizes a \theoremstyle specification (as in Mittelbach's theorem 
%  package) and has a * form for defining unnumbered environments. The
%  amsthm package also defines a proof environment that automatically adds a
%  QED symbol at the end. AMS document classes incorporate the amsthm 
%  package, so everything described here applies to them as well.
%
%  If the amsthm package is used with a non-AMS document class and with the
%  amsmath package, amsthm must be loaded after amsmath, not before.

%3)amssymb provides an extended symbol collection. For example, after 
%loading amssymb you have the following additional binary relation symbols:
%\barwedge, \boxdot, \boxminus, \boxplus, \boxtimes, \Cap, \Cup (and many 
%more), the arrow \leadsto, and some other symbols such as \Box and 
%\Diamond. Another useful feature is the \mathbb command to produce 
%blackboard bold characters.

% Since amssymb internally loads amsfonts, it's enough to load the former.




%%%%%%%%%%%%%%%%%%%%%%%%%%%%%%%%%%%%%%%%%%%%%%%%%%%%%%%%%%%%%%%%%%%%%%%%%%%%
%%%%%%%%%%%%%%%%%%%%%%%%%%%%%%%%%%%%%%%%%%%%%%%%%%%%%%%%%%%%%%%%%%%%%%%%%%%%


%---------------------------------------------------------------------------
% enumerate number per section
%---------------------------------------------------------------------------

\usepackage{enumitem}
% Note: This also allows us to do something like this:
%\begin{enumerate}[label=\FOO*]
% \alph* = a, \Alph* = A, \roman* = i, ii, etc...
% Put in parenthesis for parenthesis. I.e.
% (\alph*) = (a)
% Label: \alph, \Alph, \arabic, \roman and \Roman


%---------------------------------------------------------------------------
% url, hyperref, bookmark, etc...
%---------------------------------------------------------------------------

\WarningFilter{hyperref}{You have enabled option `breaklinks'.}

% url is loaded by hyperref, which is loaded by bookmark.

% The command \PassOptionsToPackage (used below) tells LaTeX to behave as 
% if its options were passed, when it finally loads a package. As you would 
% expect from its name, \PassOptionsToPackage can deal with a list of 
% options, just as you would have in the the options brackets of 
% \usepackage.

% Ordinarily, breaks are not allowed after “-” characters because this leads
% to confusion. (Is the “-” part of the address or just a hyphen?) The 
% package option “[hyphens]” allows breaks after explicit hyphen characters.
% The \url command will never ever hyphenate words.
% Also can use \path{}
\PassOptionsToPackage{hyphens,obeyspaces,spaces}{url}

% Allows link text to break across lines; since this cannot be accommodated 
% in PDF, it is only set true by default if the pdftex driver is used. This
% makes links on multiple lines into different PDF links to the same target.
\PassOptionsToPackage{breaklinks=true}{hyperref}

% http://tex.stackexchange.com/questions/167948/package-rerunfilecheck-warning-file-out-has-changed
% Or add \usepackage{bookmark}. Then a more modern implementation of the 
% bookmarks managing is used without .out file. The bookmarks are updated 
% earlier, thus in most cases only one LaTeX run is needed.
\usepackage{bookmark}

%% breakurl needs hyperref, but hyperref is loaded by bookmark!
\usepackage[anythingbreaks]{breakurl}

%% Internal links
\hypersetup{
colorlinks=true,
linkcolor=blue,
filecolor=magenta,      
urlcolor=cyan,
citecolor=cyan,
%pdfborder={0 0 0},      % No borders around internal hyperlinks
pdftitle={Raymon White Resume},
pdfauthor={Raymon White} % author
}
\urlstyle{same}



%---------------------------------------------------------------------------
% graphic, graphicx
%---------------------------------------------------------------------------

%% This is for includegraphics{}
% \usepackage{graphics} % Already loaded by graphicsx
% pdftex (default if compiling with pdflatex), if you are compiling with 
% pdftex to get a PDF that you will see with any PDF viewer.

% This gets an error, reason is below:
%\usepackage[pdftex]{graphicx}

% PDFLaTeX does not support EPS files. However, some modern LaTeX 
% distributions will try to automatically convert it to a PDF image, which 
% maybe this doesn't work in your case. It would be also possible that the 
% size information, namely the bounding box, in that EPS file is missing.
%
%If you are using latex, i.e. the DVI mode of LaTeX, which supports EPS then
%you should remove the incorrect pdftex option from the graphicx package. 
%The packages should be able to figure out the used driver by themselves, so
%you should avoid such options anyway. If you are using pdflatex then it 
%would be better to convert the EPS to a PDF file manually, e.g. using the 
%epstopdf tool (I'm not sure if it comes with the Windows version, but I 
%think so). You need to change the .eps extension to .pdf then, of course. 
%You can also drop the extension and LaTeX will look for files with the 
%given base name with all supported extensions, i.e. \includegraphics{file} 
%will use file.eps for latex and file.pdf (or file.png, file.jpg) for 
%pdflatex.

% TL;DR: Take out pdftex, and convert eps to pdf manually (although the
% engine should do it for us, if required)
\usepackage{graphicx}




\usepackage{titlesec}

% For lorem ipsum
\usepackage[english]{babel}
\usepackage{blindtext}
% usage:
%\blindtext
%\Blindtext



%\usepackage{setspace}
% Below are the spacing options.
%\doublespacing
%\singlespacing
%\onehalfspacing
%\setstretch{1.8}

%\newtheorem*{definition}{Definition}
%\newtheorem*{example}{Example}
%\newtheorem*{theorem}{Theorem}
%\newtheorem*{theorem*}{Theorem}

% This is the standardone inch all around border.
%\usepackage{fullpage}

% This is what I've made myself - max space used, for Tisseur cw.
% For most stand alone work, use this one.
%\setlength{\topmargin}{-1in}
%\setlength{\textheight}{9.75in}
%\setlength{\oddsidemargin}{-0.25in}
%\setlength{\textwidth}{7in}

% This is the geometry package... apparently good!
\usepackage[left=2cm,top=4cm,right=2cm,a4paper]{geometry}
\geometry{top=0.7in, bottom=0.7in, left=0.7in, right=0.7in}

\pagenumbering{gobble}

%\lstset{numbers=left, 
%        stepnumber=2, 
%        frame=single,
%        basicstyle=\footnotesize,
%        showstringspaces=false,
%        language=C++}

\raggedbottom
\raggedright
\setlength{\tabcolsep}{0in}

% Sections formatting
\titleformat{\section}{
  \vspace{-4pt}\Large\bfseries\raggedright
}{}{0em}{}[\color{black}\titlerule \vspace{-5pt}]

%Add horizontal line under each section heading
%\titleformat{\section}
%  {\normalfont\Large\bfseries}{}{1em}{}[{\titlerule[0.8pt]}]


%-------------------------
% Custom commands
\newcommand{\resumeItem}[2]{
  \item\small{
    \textbf{#1}{: #2 \vspace{-2pt}}
  }
}

\newcommand{\resumeSubheading}[4]{
  \vspace{-1pt}\item
    \begin{tabular*}{0.97\textwidth}{l@{\extracolsep{\fill}}r}
      \textbf{#1} & #2 \\
      \textit{\small#3} & \textit{\small #4} \\
    \end{tabular*}\vspace{-5pt}
}

\newcommand{\resumeSubItem}[2]{\resumeItem{#1}{#2}\vspace{-4pt}}

\renewcommand{\labelitemii}{$\circ$}

\newcommand{\resumeSubHeadingListStart}{\begin{itemize}[leftmargin=*]}
\newcommand{\resumeSubHeadingListEnd}{\end{itemize}}
\newcommand{\resumeItemListStart}{\begin{itemize}}
\newcommand{\resumeItemListEnd}{\end{itemize}\vspace{-5pt}}

\setlength{\parindent}{1em}
\setlength{\parskip}{1em}
%\renewcommand{\baselinestretch}{2.0}



% Underlined and bold and newlined
% the \phantomsection is for when we use label{rh:foo}
\DeclareRobustCommand{\rrheaderunderline}[1]{\par\medskip\phantomsection\noindent\textbf{\underline{#1}}\medskip}
\DeclareRobustCommand{\rrheader}[1]{\par\medskip\phantomsection\noindent{\normalsize\textbf{#1}}\medskip}
\DeclareRobustCommand{\rrheaderlarge}[1]{\par\medskip\phantomsection\noindent{\large\textbf{#1}}\medskip}
\DeclareRobustCommand{\rrheaderLarge}[1]{\par\medskip\phantomsection\noindent{\Large\textbf{#1}}\medskip}




%%%%%%%%%%%%%%%%%%%%%%%%%%%%%%%%%%%%%%%%%%%%%%%%%%%%%%%%%%%%%%%%%%%%%%%%%%%%
%%%%%%%%%%%%%%%%%%%%%%%%%%%%%%%%%%%%%%%%%%%%%%%%%%%%%%%%%%%%%%%%%%%%%%%%%%%%
%%Document begin
%%%%%%%%%%%%%%%%%%%%%%%%%%%%%%%%%%%%%%%%%%%%%%%%%%%%%%%%%%%%%%%%%%%%%%%%%%%%
%%%%%%%%%%%%%%%%%%%%%%%%%%%%%%%%%%%%%%%%%%%%%%%%%%%%%%%%%%%%%%%%%%%%%%%%%%%%
 
\begin{document}

%%%%%%%%%%%%%%%%%%%%%%%%%%%%%%%%%%%%%%%%%%%%%%%%%%%%%%%%%%%%%%%%%%%%%%%%%%%%
%-----------Begin Doc-------------------------------------------------------
%%%%%%%%%%%%%%%%%%%%%%%%%%%%%%%%%%%%%%%%%%%%%%%%%%%%%%%%%%%%%%%%%%%%%%%%%%%%
\section{Wikipedia Article}

\url{https://en.wikipedia.org/wiki/Schlumberger}

Schlumberger Limited is the world's largest oilfield services company.
Schlumberger employs approximately \textbf{100,000 people} representing more
than 140 nationalities working in more than 85 countries. Schlumberger has
four principal executive offices located in \textbf{Paris},
\textbf{Houston}, \textbf{London}, and \textbf{the Hague} (a city on the
western coast of the Netherlands and the capital of the province of South
Holland.).

Oilfield service companies--companies which provide services to the
petroleum exploration and production industry but do not typically produce
petroleum themselves. 

\textbf{Schlumberger subidiaries:}
\begin{itemize}[noitemsep,topsep=0pt]
\item Cameron International: a global provider of pressure control,
  processing, flow control and compression systems as well as project
  management and aftermarket services for the \textbf{oil and gas and
    process industries}.
\item Geophysical Service (GSI): purpose of using refraction and reflection
  seismology to explore for petroleum deposits.
\item Geoservices: an \textbf{upstream} (see below for explanation) oilfield
  service company founded in 1958 by Gaston Rebilly. It provides a range of
  skills that help evaluate hydrocarbon reservoirs and optimize field
  exploration, development and production. Geoservices provides in 3
  business segments: \textbf{(1)} Mud Logging (World No. 1), \textbf{(2)}
  Field Surveillance and \textbf{(2)} Well Intervention (World No. 2).
\item Smith International: Smith International was a Fortune 500 company
  headquartered in the Greenspoint district and in unincorporated Harris
  County, Texas. Smith International ceased to exist as an independent
  company following \textbf{the merger with Schlumberger}. This company
  supplies products to gas and oil production and exploration companies. The
  company used to be easily identified by its red \textbf{Sii} logo. The
  company had recently changed its logo to consist of the word "SMITH" in
  black capital letters with a green globe.
\item WesternGeco: WesternGeco is a geophysical services company. It is
  headquartered in the Schlumberger House on the property of London Gatwick
  Airport in Crawley, West Sussex, in Greater London. The company provides
  reservoir imaging, monitoring, and development services. The company, a
  business segment of Schlumberger, offers 3D and time-lapse seismic
  surveys, electromagnetic surveys, and multicomponent surveys for
  delineating prospects and reservoir management. It also provides
  geophysical, land and transition-zone acquisition, marine acquisition,
  electromagnetics, and data processing and reservoir seismic services. The
  company, formerly known as Western Geophysical Company, was founded in
  1933.
\end{itemize}

\textbf{Upstream/midstream and downstream:}

The oil and gas industry is usually divided into three major sectors:
upstream (or \textbf{exploration and production--E\&P}), midstream and
downstream.
\begin{itemize}[noitemsep,topsep=0pt]
\item Upstream: The upstream sector includes searching for potential
  underground or underwater crude oil and natural gas fields, drilling
  exploratory wells, and subsequently drilling and operating the wells that
  recover and bring the crude oil or raw natural gas to the surface.
  There has been a significant shift toward including unconventional gas as
  a part of the upstream sector, and corresponding developments in liquefied
  natural gas (LNG) processing and transport.

  \textbf{Business:} This categorization comes from value chain concepts,
  even before formal development Value Chain Management: \textbf{Oil Service
    Company:} A company that provides products and/or services to the oil
  and gas industry.  Usually a combination of labor, equipment, and/or other
  support services.  Examples include Baker Hughes, Haliburton, and
  Schlumberger.
\item The midstream sector involves the transportation (by pipeline, rail,
  barge, oil tanker or truck), storage, and wholesale marketing of crude or
  refined petroleum products. Pipelines and other transport systems can be
  used to move crude oil from production sites to refineries and deliver the
  various refined products to \textbf{downstream distributors}. Natural gas
  pipeline networks aggregate gas from natural gas purification plants and
  deliver it to \textbf{downstream customers}, such as local utilities.

  The midstream operations are often taken to include some elements of the
  upstream and downstream sectors. For example, the midstream sector may
  include natural gas processing plants that purify the raw natural gas as
  well as removing and producing elemental sulfur and natural gas liquids
  (NGL) as finished end-products.
\item The downstream sector is the refining of petroleum crude oil and the
  processing and purifying of raw natural gas, as well as the marketing
  and distribution of products derived from crude oil and natural gas. The
  downstream sector reaches consumers through products such as gasoline or
  petrol, kerosene, jet fuel, diesel oil, heating oil, fuel oils,
  lubricants, waxes, asphalt, natural gas, and liquefied petroleum gas (LPG)
  as well as hundreds of petrochemicals.

  Midstream operations are often included in the downstream category and are
  considered to be a part of the downstream sector.
\end{itemize}

Schlumberger was \textbf{founded in 1926} by brothers Conrad and Marcel
Schlumberger from the Alsace region in France as the Soci\'et\'e de
prospection \'electrique (French: Electric Prospecting Company). The company
recorded the first-ever electrical resistivity well log in
Merkwiller-Pechelbronn, France in 1927. Today Schlumberger supplies the
petroleum industry with services such as \textbf{seismic acquisition} and
\textbf{processing}, \textbf{formation evaluation}, \textbf{well testing}
and \textbf{directional drilling}, \textbf{well cementing} and
\textbf{stimulation}, \textbf{artificial lift}, \textbf{well completions},
\textbf{flow assurance} and \textbf{consulting}, and \textbf{software and
  information management}. The company is also involved in the groundwater
extraction and carbon capture and storage industries.

The brothers had experience conducting geophysical surveys in countries such
as Romania, Canada, Serbia, South Africa, the Democratic Republic of the
Congo and the United States. The new company sold electrical-measurement
mapping services, and recorded the first-ever electrical resistivity well
log in Merkwiller-Pechelbronn, France in 1927. The company quickly expanded,
logging its first well in the U.S. in 1929, in Kern County, California. In
1935, the Schlumberger Well Surveying Corporation was founded in Houston,
later evolving into Schlumberger Well Services, and finally Schlumberger
Wireline \& Testing. Schlumberger invested heavily in research, inaugurating
the Schlumberger-Doll Research Center in Ridgefield, Connecticut in 1948,
contributing to the development of a number of new logging tools. In 1956,
Schlumberger Limited was incorporated as a holding company for all
Schlumberger businesses, which by now included American testing and
production company Johnston Testers.

Over the years, Schlumberger continued to expand its operations and
acquisitions. In \textbf{1960}, Dowell Schlumberger (50\% Schlumberger, 50\%
Dow Chemical), which specialized in pumping services for the oil industry,
was formed. In \textbf{1962}, Schlumberger Limited became listed on the New
York Stock Exchange. That same year, \textbf{Schlumberger purchased
  Daystrom}, an electronic instruments manufacturer in South Boston,
Virginia which was \textbf{making furniture by the time the division was
  sold to Sperry \& Hutchinson in 1971}. Schlumberger purchased 50\% of
Forex in 1964 and merged it with 50\% of Languedocienne to create the
Neptune Drilling Company. The first computerized reservoir analysis,
SARABAND, was introduced in 1970. The remaining 50\% of Forex was acquired
the following year; Neptune was renamed Forex Neptune Drilling Company. In
1979, Fairchild Camera and Instrument (including Fairchild Semiconductor)
became a subsidiary of Schlumberger Limited.

In 1981, Schlumberger established the first international data links with
e-mail. \textbf{In 1983, Schlumberger opened their Cambridge Research Center
  in Cambridge, England and in 2012 it was renamed the Schlumberger Gould
  Research Center after the company's former CEO Andrew Gould.}

The SEDCO drilling rig company and half of Dowell of North America were
acquired in 1984, resulting in the creation of the Anadrill drilling
segment, a combination of Dowell and The Analysts' drilling segments. Forex
Neptune was merged with SEDCO to create the Sedco Forex Drilling Company the
following year, when Schlumberger purchased Merlin and 50\% of GECO.

In 1987, Schlumberger completed their purchases of Neptune (North America),
Bosco and Cori (Italy), and Allmess (Germany). That same year, National
Semiconductor acquired Fairchild Semiconductor from Schlumberger for \$122
million. In 1991, Schlumberger acquired PRAKLA-SEISMOS, and pioneered the
use of geosteering to plan the drill path in horizontal wells.

\textbf{Schlumberger acquired software company GeoQuest Systems in 1992.}
With the purchase came the conversion of SINet to TCP/IP and www capability.
In the 1990s Schlumberger bought out the petroleum division, AEG meter, and
ECLIPSE reservoir study team Intera Technologies Corp. A joint venture
between Schlumberger and Cable \& Wireless resulted with the creation of
Omnes, which then handled all of Schlumberger's internal IT business.
Oilphase and Camco International were also purchased.

\textbf{In 1999, Schlumberger and Smith International created a joint venture, M-I
L.L.C., the world's largest drilling fluids (or mud) company.} The company
consists of 60\% Smith International, and 40\% Schlumberger. Since the joint
venture was prohibited by a 1994 antitrust consent decree barring Smith from
selling or combining their fluids business with certain other companies,
including Schlumberger, the U.S. District Court in Washington, D.C. found
Smith International Inc. and Schlumberger Ltd. guilty of criminal contempt
and fined each company \$750,000 and placed each company on five years
probation. Both companies also agreed to pay a total of \$13.1 million,
representing a full disgorgement of all of the joint venture's profits
during the time the companies were in contempt.[17]

In 2000, the Geco-Prakla division was merged with Western Geophysical to
create the seismic contracting company WesternGeco, of which Schlumberger
held a 70\% stake, the remaining 30\% belonging to competitor Baker Hughes.
Sedco Forex was spun off, and merged with Transocean Drilling company in
2000.

In 2001, Schlumberger acquired the IT consultancy company Sema plc for \$5.2
billion. The company was an Athens 2004 Summer Olympics partner, but
Schlumberger's venture into IT consultancy did not pay off, and
divestiture\footnote{the action or process of selling off subsidiary
  business interests or investments.} of Sema to Atos Origin was completed
that year for \$1.5 billion. The cards division was divested through an IPO
to form Axalto, which later merged with Gemplus to form Gemalto, and the
Messaging Solutions unit was spun off and merged with Taral Networks to form
Airwide Solutions. In 2003, the Automated Test Equipment group, part of the
1979 Fairchild Semiconductor acquisition, was spun off to NPTest Holding,
which later sold it to Credence.

In 2004, Schlumberger Business Consulting was launched. Based in Paris, it
is the company's management consultancy arm.

In 2005, Schlumberger purchased Waterloo Hydrogeologic, which was followed
by several other groundwater industry related companies, such as Westbay
Instruments, and Van Essen Instruments. Also that year, Schlumberger
relocated its U.S. corporate offices from New York to Houston.

In 2006, Schlumberger purchased the remaining 30\% of WesternGeco from Baker
Hughes for US\$2.4 billion. Also that year, the Schlumberger-Doll
Research Center was relocated to a newly built research facility in
Cambridge, Massachusetts to replace the Ridgefield, Connecticut research
center. The facility joins the other research centers operated by the
company in Cambridge, England; Moscow, Russia; Stavanger, Norway; and
Dhahran, Saudi Arabia.

In 2010, the acquisition of Smith International in an all-stock deal valued
at \$11.3 billion was announced. The sale price is 45.84-a-share price was
37.5 percent higher than Smith closing price on 18 February 2010. The deal
is the biggest acquisition in Schlumberger history. The merger was
completed on August 27, 2010. Also announced in 2010 were Schlumberger
plans to acquire Geoservices, a French-based company specializing in energy
services, in a deal valued at \$1.1 billion, including debt.

In 2014, Schlumberger announced the purchase of the remaining shares of SES
Holdings Limited (``Saxon''), a Calgary-based provider of international land
drilling services, from First Reserve and certain members of Saxon
management. The transaction is subject to customary closing conditions,
including the receipt of regulatory approvals. Schulmberger had a minority
share in Saxon previously.

In 2015, Schlumberger was indicted by the US Department of Justice for
sanction violations of conducting business in Iran and Sudan; the company
was fined \$233 million, amounting to the largest fine for sanctions to
date.

Due to a downturn in the global oil \& gas industry in 2015, Schlumberger
announced 21,000 layoffs accounting for 15\% of their total workforce.

In August 2015, Schlumberger agreed to acquire oilfield equipment
manufacturer Cameron International for \$14.8 billion.

 
 \end{document}
