

% List of long classes and function names to use in texttt
\newcommand{\NSSCP}{Navi\-er\-Stokes\-Schur\-Com\-ple\-ment\-Pre\-con\-di\-tion\-er}
\newcommand{\HYPREPREC}{Hy\-pre\-Pre\-con\-di\-tion\-er}
\newcommand{\HYPRESETCOARSE}{HYPRE\-\_\-Boom\-er\-AMG\-Set\-Coars\-en\-Type(...)}
\newcommand{\HYPRESETRELAX}{HYPRE\-\_\-Boom\-er\-AMG\-Set\-Re\-lax\-Type(...)}
\newcommand{\BLOCKPREC}{Block\-Pre\-con\-di\-tion\-er}
\newcommand{\PREC}{Pre\-con\-di\-tion\-er}
\newcommand{\GENELEMENT}{Gen\-er\-al\-ised\-El\-e\-ment}
\newcommand{\QTAYLORTWO}{Q\-Tay\-lor\-Hood\-El\-e\-ment\-<2>}
\newcommand{\QTAYLORTHREE}{Q\-Tay\-lor\-Hood\-El\-e\-ment\-<3>}
\newcommand{\FACEELEMENT}{Face\-El\-e\-ment}
\newcommand{\IMPOSEPELEMENT}{Im\-pose\-Par\-al\-lel\-Out\-flow\-El\-e\-ment}
\newcommand{\BLOCKTRIANGULARPREC}{Block\-Tri\-an\-gu\-lar\-Pre\-con\-di\-tion\-er}
\newcommand{\TTPROBLEM}{Prob\-lem}
\newcommand{\TTMESH}{Mesh}
\newcommand{\TTMETIS}{ME\-TIS}
\newcommand{\TTELEMENT}{El\-e\-ment}
\newcommand{\TTNODE}{Node}
\newcommand{\TTCRDOUBLEMATRIX}{C\-R\-Dou\-ble\-Ma\-trix}
\newcommand{\TTDOUBLEVECTOR}{Dou\-ble\-Vec\-tor}
\newcommand{\TTSUPERLUPRECONDITIONER}{Su\-per\-LU\-Pre\-con\-di\-tion\-er}


%%%%%%%%%%%%%%%%%%%%%%%%%%%%%%%%%%%%%%%%%%%%%%%%%%%%%%%%%%%%%%%%%%%%%%%%%%%%

% namespaces
\newcommand{\NAMESPACE}{name\-space}

% CRDoubleMatrixHelpers namespace
\newcommand{\CRDOUBLEMATRIXHELPERS}{CR\-Dou\-ble\-Ma\-trix\-Help\-ers}
% Here are the functions in the above namespace
\newcommand{\CATWOCOMM}{con\-cat\-e\-nate\-\_\-with\-out\-\_\-com\-mu\-ni\-ca\-tion}


\newcommand{\LINALGDIST}{Lin\-e\-ar\-Al\-ge\-bra\-Dis\-tri\-bu\-tion}

% Functions, these all end with (...)
\newcommand{\TURNINTO}{turn\-\_\-in\-to\-\_\-sub\-sid\-i\-ar\-y\-\_\-block\-\_\-pre\-con\-di\-tion\-er(...)}
\newcommand{\BLOCKSETUP}{block\-\_\-set\-up(...)}
\newcommand{\SETUP}{set\-up(...)}
\newcommand{\PRECONDITIONERSOLVE}{pre\-con\-di\-tion\-er\-\_\-solve(...)}
\newcommand{\NDOFTYPESFUN}{n\-dof\-\_\-types()}
\newcommand{\NBLOCKTYPESFUN}{n\-block\-\_\-types()}
\newcommand{\GETDOFNUM}{get\-\_\-dof\-\_\-num\-bers\-\_\-for\-\_\-un\-knowns(...)}
\newcommand{\SETNMESH}{set\-\_\-n\-mesh(...)}
\newcommand{\SETMESH}{set\-\_\-mesh(...)}
\newcommand{\SETFPREC}{set\-\_\-f\-\_\-pre\-con\-di\-tion\-er(...)}
\newcommand{\MATRIXPTFUN}{ma\-trix\-\_\-pt()}
\newcommand{\GETBLOCK}{get\-\_\-block(...)}
\newcommand{\SETREPLACEMENTDOFBLOCK}{set\-\_\-re\-place\-ment\-\_\-dof\-\_\-block(...)}
\newcommand{\GETBLOCKVECTOR}{get\-\_\-block\-\_\-vec\-tor(...)}
\newcommand{\RETURNBLOCKVECTOR}{re\-turn\-\_\-block\-\_\-vec\-tor(...)}
% For the following command, we must put in (i,j) ourselves.
\newcommand{\GETBLOCKIJ}{get\-\_\-block}



% Variables
\newcommand{\BULKMESHPT}{Bulk\_mesh\_pt}
\newcommand{\BOUNDARYMESHPT}{Bound\-a\-ry\_mesh\_pt}
\newcommand{\NDOFTYPESVAR}{n\-dof\-\_\-types}
\newcommand{\DOFTOBLOCKMAP}{dof\_to\_block\_map}
\newcommand{\INTERNALDOFTOBLOCKMAP}{in\-ter\-nal\_dof\_to\_block\_map}
\newcommand{\DOFTYPECOARSENMAP}{dof\-type\-\_\-coars\-en\-\_\-map}
\newcommand{\FPRECPT}{F\-\_\-pre\-con\-di\-tion\-er\_pt}
\newcommand{\NVELOCITYDOFTYPES}{{n\-ve\-loc\-i\-ty\-\_\-dof\-\_\-types}}
\newcommand{\DOFMAP}{dof\-\_\-map\-}
\newcommand{\NAVIERSTOKESBLOCKPRECONDITIONERPT}{Navi\-er\_stokes\_block\_pre\-con\-di\-tion\-er\_pt}
\newcommand{\DOFTYPECOARSENMAPFINE}{Dof\-type\-\_\-coars\-en\-\_\-map\-\_\-fine}
\newcommand{\WPRECONDITIONERPT}{W\-\_\-pre\-con\-di\-tion\-er\-\_\-pt}
\newcommand{\WPT}{w\-\_\-pt}
\newcommand{\MATRIXVECTORPRODUCT}{Ma\-trix\-Vec\-tor\-Prod\-uct}


% Identifiers?
\newcommand{\VECTORVECTORUNSIGNED}{Vec\-tor<\-Vec\-tor<un\-signed> >}
\newcommand{\VECTORUNSIGNED}{Vec\-tor<un\-signed\->}
\newcommand{\VECTOR}{Vec\-tor}
\newcommand{\UNSIGNED}{un\-signed}
\newcommand{\INT}{int}
\newcommand{\DOUBLE}{dou\-ble}
\newcommand{\STDVECTOR}{std\-::\-vec\-tor}

% Other
\newcommand{\TTFOR}{for}

%% these must be in mathmode
\newcommand{\nsubprecdoftypes}{N_{dof}^{S}}
\newcommand{\nmasprecdoftypes}{N_{dof}^{M}}

\newcommand{\nloc}{N_{p}}
%\newcommand{\nproc}{n_{p}}


\newcommand{\pde}{partial differential equation}
\newcommand{\Pde}{Partial differential equation}
\newcommand{\pdes}{partial differential equations}
\newcommand{\Pdes}{Partial differential equations}
%\newcommand{\ul}[1]{\underline{#1}}
\newcommand{\degree}{^{\circ}}

% Registered trademark sign
\newcommand{\supertextregistered}{\textsuperscript{\textregistered}}

% Target problem size for matrix concatenation
\newcommand{\tarN}{\mathcal{N}}
\newcommand{\Nrowsub}{\widehat{\mathcal{R}}}

%%%%%%%%%%%%%%%%%%%%%%%%%%%%%%%%%%%%%%%%%%%%%%%%%%%%%%%%%%%%%%%%%%%%%%%%%%%%
%%%%%%%%%%%%%%%%%%%%%%%%%%%%%%%%%%%%%%%%%%%%%%%%%%%%%%%%%%%%%%%%%%%%%%%%%%%%
%%%%%%%%%%%%%%%%%%%%%%%%%%%%%%%%%%%%%%%%%%%%%%%%%%%%%%%%%%%%%%%%%%%%%%%%%%%%
%%%%%%%%%% Words
\newcommand{\oomphlib}{\texttt{oomph-lib}}
\newcommand{\Oomphlib}{\texttt{Oomph-lib}}
\newcommand{\SuperLU}{\texttt{SuperLU}}
\newcommand{\SuperLUDIST}{\texttt{SuperLU\_DIST}}
\newcommand{\SuperLUMT}{\texttt{SuperLU\_MT}}
\newcommand{\faceelement}{\texttt{FaceElement}}
\newcommand{\infsup}{\emph{inf-sup}}
\newcommand{\HYPRE}{\texttt{HYPRE}}
\newcommand{\BoomerAMG}{\texttt{BoomerAMG}}
\newcommand{\TRILINOS}{\texttt{Trilinos}}
\newcommand{\AZTECOO}{\texttt{AztecOO}}
\newcommand{\Belos}{\texttt{Belos}}
\newcommand{\EPETRA}{\texttt{Epetra}}
\newcommand{\EPETRAEXT}{\texttt{EpetraExt}}
\newcommand{\EPETRACRSMATRIX}{\texttt{Epetra\_CrsMatrix}}
\newcommand{\EPETRAMAP}{\texttt{Epetra\_Map}}
\newcommand{\MATLAB}{\texttt{MATLAB}}
\newcommand{\polyfit}{\texttt{polyfit}}
\newcommand{\MAXVMEM}{\texttt{maxvmem}}
\newcommand{\QACCT}{\texttt{qacct}}
\newcommand{\TetGen}{\texttt{Tet\-Gen}}
\newcommand{\SVTNSE}{SVT-NSE}
\newcommand{\SDVTNSE}{SDVT-NSE}

%%%%%%%%%%%%%%%%%%%%%%%%%%%%%%%%%%%%%%%%%%%%%%%%%%%%%%%%%%%%%%%%%%%%%%%%%%%%
%%%%%%%%%%%%%%%%%%%%%%%%%%%%%%%%%%%%%%%%%%%%%%%%%%%%%%%%%%%%%%%%%%%%%%%%%%%%
%%%%%%%%%%%%%%%%%%%%%%%%%%%%%%%%%%%%%%%%%%%%%%%%%%%%%%%%%%%%%%%%%%%%%%%%%%%%

\newcommand{\diag}{\mathop{\mathrm{diag}}}

\newcommand{\suchthat}{\mathrel{}\middle|\mathrel{}}

% Ceiling and floor.
\DeclarePairedDelimiter\ceil{\lceil}{\rceil}
\DeclarePairedDelimiter\floor{\lfloor}{\rfloor}

% Operators
% abs | . |
\DeclarePairedDelimiter\abs{\lvert}{\rvert}% need mathtools
% norm || . ||
\DeclarePairedDelimiter\norm{\lVert}{\rVert}% need mathtools
% parenthesis ( . )
\DeclarePairedDelimiter\paren{(}{)}% need mathtools
% bracket parenthesis [ . ]
\DeclarePairedDelimiter\bparen{[}{]}% need mathtools
% curly bracket parenthesis { . }
\DeclarePairedDelimiter\Bparen{\{}{\}}% need mathtools

% Swap the definition of \abs* and \norm*, so that \abs
% and \norm resizes the size of the brackets, and the 
% starred version does not.
\makeatletter
%
\let\oldceil\ceil
\def\ceil{\@ifstar{\oldceil}{\oldceil*}}
%
\let\oldfloor\floor
\def\floor{\@ifstar{\oldfloor}{\oldfloor*}}
%
\let\oldabs\abs
\def\abs{\@ifstar{\oldabs}{\oldabs*}}
%
\let\oldnorm\norm
\def\norm{\@ifstar{\oldnorm}{\oldnorm*}}
%
\let\oldparen\paren
\def\paren{\@ifstar{\oldparen}{\oldparen*}}
%
\let\oldbparen\bparen
\def\bparen{\@ifstar{\oldbparen}{\oldbparen*}}
%
\let\oldBparen\Bparen
\def\Bparen{\@ifstar{\oldBparen}{\oldBparen*}}
\makeatother

%%%%%%%%%%%%%%%%%%%%%%%%%%%%%%%%%%%%%%%%%%%%%%%%%%%%%%%%%%%%%%%%%%%%%%%%%%%%
%%%%%%%%%%%%%%%%%%%%%%%%%%%%%%%%%%%%%%%%%%%%%%%%%%%%%%%%%%%%%%%%%%%%%%%%%%%%
%%%%%%%%%%%%%%%%%%%%%%%%%%%%%%%%%%%%%%%%%%%%%%%%%%%%%%%%%%%%%%%%%%%%%%%%%%%%

% bold face mathematics for greeks only
% for letters, we use mathbf, they produce very different results.
\newcommand{\veccalc}[1]{\bm{#1}}

% new sign for transpose, this is good, again must be in mathmode.
\newcommand{\raytran}{T}
% sign for the inverse
\newcommand{\rayinv}{-1}


%%%%%%%%%%%%%%%%%%%%%%%%%%%%%%%%%%%%%%%%%%%%%%%%%%%%%%%%%%%%%%%%%%%%%%%%%%%%
%%%%%%%%%%%%%%%%%%%%%%%%%%%%%%%%%%%%%%%%%%%%%%%%%%%%%%%%%%%%%%%%%%%%%%%%%%%%
%%%%%%%%%%%%%%%%%%%%%%%%%%%%%%%%%%%%%%%%%%%%%%%%%%%%%%%%%%%%%%%%%%%%%%%%%%%%


% Navier-Stokes / Fluid stuff. This must go into math mode.

% Dimensionless numbers
% Note: We use \mathrm here (as oppose to \text) because \text takes on the
% current text format (so it may be italic, but these are mathematical
% symbols.)
\newcommand{\dlReynolds}{\mathrm{Re}} % Reynolds number
\newcommand{\dlStrouhal}{\mathrm{St}} % Strouhal number
\newcommand{\dlFroude}{\mathrm{Fr}} % Froude number
\newcommand{\dlKnudsen}{\mathrm{Kn}} % Knudsen number, used in introduction
\newcommand{\dlMach}{\mathrm{Ma}} % Mach number, used in NS section.

% Others like kinematic viscosity, mean free math, density, etc...
\newcommand{\meanfreepath}{\lambda} % mean free path, used in introduction

\newcommand{\charlength}{\mathcal{L}} % Characteristic length scale.
\newcommand{\charvelo}{\mathcal{U}} % Characteristic velocity.
\newcommand{\chartime}{\mathcal{T}} % Characteristic time.

\newcommand{\nstemperature}{\theta} % Fluid's temperature
\newcommand{\nsdensity}{\rho} % Fluid's density
\newcommand{\nsviscosity}{\mu} % Fluid's dynamic viscosity
\newcommand{\nsdensityref}{\nsdensity_{ref}} % Fluid's reference density
\newcommand{\nsviscosityref}{\nsviscosity_{ref}} % Fluid's reference viscosity
\newcommand{\nsratiodensity}{\mathrm{R}_{\nsdensity}}
\newcommand{\nsratioviscosity}{\mathrm{R}_{\nsviscosity}}
\newcommand{\nskinematicviscosity}{\nu}

\newcommand{\nsintensive}{\varphi}
\newcommand{\nscontrolvolume}{\Omega}


% Gresho p424: We use Greek indices to denote spatial vectors and directions
% because this will help us later when we introduce nodes and finite element
% basis functions.
\newcommand{\nsi}{i} % For the i-th direction
\newcommand{\nsj}{j} % for the j-th direction.
\newcommand{\nsvelo}{u} % velocity
\newcommand{\nsvelovec}{\mathbf{\nsvelo}}
\newcommand{\nsvelow}{w} % velocity
\newcommand{\nsvelowvec}{\mathbf{\nsvelow}}
\newcommand{\nsnormvec}{\mathbf{n}}
\newcommand{\nspres}{p} % pressure
\newcommand{\nstime}{t} % time
\newcommand{\nsx}{x} % x - Cartesian coordinate
\newcommand{\nsbody}{B} % variable body force
\newcommand{\nsgravity}{G} % constant body force
\newcommand{\nsvolsource}{Q} % volumetric source term
\newcommand{\strainratetensor}{\bm{\epsilon}}

% Solution of linear systems LSC derivation:
\newcommand{\CDOpVec}{\mathcal{L}}
\newcommand{\CDOpPres}{\mathcal{L}_p}

% Navier-Stokes generic body force.
\newcommand{\nsgenbody}{\mathbf{f}}

\newcommand{\normal}{n}
\newcommand{\unitnormal}{\hat{\normal}}
% Outward UNIT normal vector
\newcommand{\unitnormalvec}{\mathbf{\unitnormal}}

\newcommand{\unittangent}{\hat{t}}
\newcommand{\unittangents}{\hat{t}_s}
\newcommand{\unittangentvec}{\mathbf{\unittangent}}
\newcommand{\unittangentsvec}{\mathbf{\hat{t}}_s}

% Traction force
\newcommand{\tractionforce}{T}
% Traction force vector
\newcommand{\tractionforcevec}{\mathbf{\tractionforce}}

% Stress tensor
\newcommand{\stresstensor}{\veccalc{\sigma}}



%%%%%%%%%%%%%%%%%%%%%%%%%%%%%%%%%%%%%%%%%%%%%%%%%%%%%%%%%%%%%%%%%%%%%%%%%%%%
%%%%%%%%%%%%%%%%%%%%%%%%%%%%%%%%%%%%%%%%%%%%%%%%%%%%%%%%%%%%%%%%%%%%%%%%%%%%
%%%%%%%%%%%%%%%%%%%%%%%%%%%%%%%%%%%%%%%%%%%%%%%%%%%%%%%%%%%%%%%%%%%%%%%%%%%%
%% Maths stuff, these should always go in math mode.

% Generic
\newcommand{\realnumbers}{\mathbb{R}} % Real numbers
\newcommand{\complexnumbers}{\mathbb{C}} % Complex
\newcommand{\imaginarynumbers}{\mathbb{I}} % Complex
\newcommand{\realvectors}[1]{\mathbb{R}^{#1}} % Real numbers
\newcommand{\realmatrices}[2]{\mathbb{R}^{#1 \times #2}} % Real numbers
\newcommand{\tolgeneral}{\varepsilon} % General tol for iterative methods
\newcommand{\tolnewton}{\varepsilon_N} % Newton tolerance
\newcommand{\tolkrylov}{\varepsilon_K} % Krylov/GMRES tolerance
\newcommand{\toltime}{\varepsilon_T} % adaptive time stepping tolerance
\newcommand{\tolnumnewton}{10^{-6}}
\newcommand{\tolnumkrylov}{10^{-6}}
\newcommand{\tolnumtime}{10^{-4}}

% AMG parameters
% Strength of dependence
\newcommand{\strn}{\theta}
% damping for Jacobi
\newcommand{\jdamp}{\omega}


% Big-O notation
% Keywords (for searching):
% bigo, big-o, asymptotic, complexity
\newcommand{\bigo}[1]{\mathcal{O}\paren{#1}} 
\newcommand{\pow}[1]{^{#1}}
\newcommand{\inv}{\pow{-1}}
\newcommand{\tran}{^{T}}



% Stuff about domains%%%%%%%%%%%%%%%%%%%%%%%%%%%%%%%%%%%%%%%%%%%%%%%%%%%%%%%
\newcommand{\domain}{\Omega}
\newcommand{\tdomain}{T}

% discretisation parameter
\newcommand{\meshsizeh}{h}

% Stuff used for multigrid.
% Mesh discretisation parameter
\newcommand{\finemeshh}{h}  % fine
\newcommand{\coarsemeshH}{H} % coarse
\newcommand{\coarsemeshhh}{2\finemeshh}

% Grid transfer operators
\newcommand{\MGprolongH}{I^{\finemeshh}_{\coarsemeshH}}
\newcommand{\MGprolonghh}{I^{\finemeshh}_{\coarsemeshhh}}
\newcommand{\MGrestrictH}{I^{\coarsemeshH}_{\finemeshh}}
\newcommand{\MGrestricthh}{I^{\coarsemeshhh}_{\finemeshh}}
\newcommand{\MGprogmat}{P}
\newcommand{\MGrestgmat}{R}
\newcommand{\MGP}{P}
\newcommand{\MGR}{R}

\newcommand{\MGcoeffAh}{A_{\finemeshh}}
\newcommand{\MGcoeffAhh}{A_{\coarsemeshhh}}
\newcommand{\MGbh}{\bar{b}_{\finemeshh}}
\newcommand{\MGbhh}{\bar{b}_{\coarsemeshhh}}
\newcommand{\MGxh}{\bar{x}_{\finemeshh}}
\newcommand{\MGxhh}{\bar{x}_{\coarsemeshhh}}
\newcommand{\MGyh}{\bar{y}_{\finemeshh}}
\newcommand{\MGyhh}{\bar{y}_{\coarsemeshhh}}
\newcommand{\MGeh}{\bar{e}_{\finemeshh}}
\newcommand{\MGehh}{\bar{e}_{\coarsemeshhh}}
\newcommand{\MGrh}{\bar{r}_{\finemeshh}}
\newcommand{\MGrhh}{\bar{r}_{\coarsemeshhh}}
\newcommand{\MGAh}{A_{\finemeshh}}
\newcommand{\MGAhh}{A_{\coarsemeshhh}}
\newcommand{\MGIh}{I_{\finemeshh}}
\newcommand{\MGIhh}{I_{\coarsemeshhh}}
\newcommand{\MGMh}{\tilde{M}_{\finemeshh}}
\newcommand{\MGMhh}{\tild{M}_{\coarsemeshhh}}
\newcommand{\MGNh}{\tilde{N}_{\finemeshh}}
\newcommand{\MGNhh}{\tild{N}_{\coarsemeshhh}}


% Domain/mesh
\newcommand{\domainh}{\Omega_{\finemeshh}}
\newcommand{\domainH}{\Omega_{\coarsemeshH}}
\newcommand{\domainhh}{\Omega_{\coarsemeshhh}}

% Basis sets
\newcommand{\xbasish}{X^{\finemeshh}_{0}}
\newcommand{\xbasisH}{X^{\coarsemeshH}_{0}}
\newcommand{\xbasishh}{X^{\coarsemeshhh}_{0}}

% Dimensions of basis sets
\newcommand{\xbasisdimh}{N_{\finemeshh}}
\newcommand{\xbasisdimH}{N_{\coarsemeshH}}
\newcommand{\xbasisdimhh}{N_{\coarsemeshhh}}



\newcommand{\vsolh}{\psi_{\finemeshh}}
\newcommand{\vsolhh}{\psi_{\coarsemeshhh}}





\newcommand{\ffh}{^{\finemeshh}} % Fine grid, length h between nodes
\newcommand{\cch}{^{\coarsemeshH}} % Course grid, length 2h between nodes
\newcommand{\ok}{^{[k]}}
\newcommand{\okk}{^{[k+1]}}

\newcommand{\ncmeshh}{2\finemeshh}
\newcommand{\ncch}{^{2\finemeshh}}





% Angle of rotation
\newcommand{\angleofrot}{\alpha}

% Domain with alpha (angle of rotation) as superscript.
\newcommand{\rotdomain}{\Omega^{[\angleofrot]}}

% Domain with param (angle of rotation) as superscript.
\newcommand{\rotdomainang}[1]{\Omega^{[#1]}}

% Boundary labels%%%%%%%%%%%%%%%%%%%%%%%%%%%%%%%%%%%%%%%%%%%%%%%%
\newcommand{\inflowlabel}{I} % Inflow
\newcommand{\characteristiclabel}{C} % Characteristic
\newcommand{\outflowlabel}{O} % Outflow
\newcommand{\naturallabel}{N} % Neumann natural outflow boundary
\newcommand{\dirichletlabel}{D} % Dirichlet boundary

\newcommand{\paralleloutflowlabel}{P} % Parallel outflow
\newcommand{\nopenetrationlabel}{T} % no penetration
\newcommand{\genconstrainedlabel}{\lambda}

% Boundaries
\newcommand{\bound}{\partial\domain} % Generic boundary

\newcommand{\inbound}{\bound_\inflowlabel} % Inflow - I
\newcommand{\charbound}{\bound_\characteristiclabel} % Characteristic - C
\newcommand{\outbound}{\bound_\outflowlabel} % Outflow - O
\newcommand{\natbound}{\bound_\naturallabel} % Neumann/natural - N
\newcommand{\dirbound}{\bound_\dirichletlabel} % Dirichlet - D

\newcommand{\pobound}{\bound_\paralleloutflowlabel} % See labels ^
\newcommand{\tfbound}{\bound_\nopenetrationlabel} % See labels ^
\newcommand{\constrainedbound}{\bound_\genconstrainedlabel} % lambda
\newcommand{\genconstrainedbound}{\bound_\genconstrainedlabel}

% Indexed boundary with i
%\newcommand{dirichletii}{\bound_\dirichletlabel} % D_i
\newcommand{\neumanni}{\bound_{\naturallabel_i}} % Neumann/natural - N_i
\newcommand{\dirichleti}{\bound_{\dirichletlabel_i}} % Dirichlet - D_i
\newcommand{\constrainedi}{\bound_{\genconstrainedlabel_i}} % Weakly - lambda_i


%\newcommand{\tractvec}{T}



% Finite element discretisation stuff
\newcommand{\eleq}[1]{\mathbf{Q}_{#1}} % Remove this after fixing Chap 2
\newcommand{\elep}[1]{\mathbf{P}_{#1}} % Remove this after fixing Chap 2
\newcommand{\triele}[1]{\mathbf{P}_{#1}}
\newcommand{\quadele}[1]{\mathbf{Q}_{#1}}
\newcommand{\taylorhood}{\quadele{2}-\quadele{1}}
\newcommand{\ijdim}{\text{for } i,j = 1,2\,[,3]}
\newcommand{\idim}{\text{for } i = 1,2\,[,3]}
\newcommand{\jdim}{\text{for } j = 1,2\,[,3]}
\newcommand{\vsol}{\psi}
\newcommand{\vsolvec}{\veccalc{\vsol}}
\newcommand{\lgrconstrainttest}{\psi^{c}}
\newcommand{\psol}{\varphi}
\newcommand{\vtest}{\Phi}
\newcommand{\ptest}{\chi}
%\newcommand{\dirichleti}{\Omega_{D_i}}
%\newcommand{\neumanni}{\Omega_{N_i}}
\newcommand{\pfrac}[2]{\frac{\partial #1}{\partial #2}}

% Element jacobian mapping between global and local coordinates
\newcommand{\elejacmap}{\hat{\mathcal{J}}}
\newcommand{\genbasis}{\phi} % General basis
\newcommand{\gbasis}{\phi_{j}} % Global basis
\newcommand{\lbasis}{\phi_{j_{local}}^{(e)}} % Local basis



% Systems of linear equations
\newcommand{\approxchi}{x}
\newcommand{\approxx}[1]{\bar{x}^{[#1]}} % For the approximate solution
\newcommand{\comapproxx}[1]{x^{[#1]}}
% A general approximate solution
% Not associated with any steps.
\newcommand{\genapproxx}{\bar{x}^{[k]}} 
\newcommand{\mapproxx}{\bar{x}^{[m]}} 
\newcommand{\Mapproxx}{\bar{x}^{[M]}} 
\newcommand{\vvv}[1]{\bar{v}^{[#1]}} % For the approximate solution
\newcommand{\cgp}[1]{\bar{p}^{[#1]}} % CG search direction
%  For CG Conjugate Gradient
\newcommand{\cgalpha}{\alpha}

%\newcommand{\cgsearchdir}{\alpha^{\[ #1 \]}}

% initial residual vector
\newcommand{\initialr}{\bar{r}^{[0]}}
% used in projection method, for the next projection:
\newcommand{\lindelta}{\bar{\delta}}

\newcommand{\iteratex}{\bar{x}}
\newcommand{\exactx}{\bar{x}^{*}}
\newcommand{\initialx}{\iteratex^{[0]}}
\newcommand{\kthx}{\iteratex^{[k]}}
\newcommand{\INUpdate}{\bar{s}}

%% Error at an iterate, for example:
% Let e_k = x^* - x_k denote the error at the kth step of an inexact Newton
% iteration.
\newcommand{\linerror}{\bar{e}}
\newcommand{\linktherror}{\bar{e}^{[k]}}
\newcommand{\lingenerror}[1]{\bar{e}^{[#1]}}


% Ax = b %%%%%%%%%%%%%%%%%%%%%%%%%%%%
\newcommand{\linA}{A} % Matrix A
\newcommand{\linB}{B} % Matrix A
\newcommand{\linC}{C} % Matrix A
\newcommand{\linD}{D} % Matrix A
\newcommand{\linE}{E} % Matrix A
\newcommand{\linF}{F} % Matrix A
\newcommand{\linG}{G} % Matrix A
\newcommand{\linH}{H} % Matrix A
\newcommand{\linI}{I} % Matrix A
\newcommand{\linJ}{J} % Matrix A
\newcommand{\linK}{K} % Matrix A
\newcommand{\linL}{L} % Matrix A
\newcommand{\linM}{M} % Matrix A
\newcommand{\linN}{N} % Matrix A
\newcommand{\linO}{O} % Matrix A
\newcommand{\linP}{P} % Matrix A
\newcommand{\linQ}{Q} % Matrix A
\newcommand{\linR}{R} % Matrix A
\newcommand{\linS}{S} % Matrix A
\newcommand{\linT}{T} % Matrix A
\newcommand{\linU}{U} % Matrix A
\newcommand{\linV}{V} % Matrix A
\newcommand{\linW}{W} % Matrix A
\newcommand{\linX}{X} % Matrix A
\newcommand{\linY}{Y} % Matrix A
\newcommand{\linZ}{Z} % Matrix A
\newcommand{\linzero}{\bar{0}} % Matrix A
\newcommand{\linzeromat}{\mathit{O}}
\newcommand{\zeromat}{\linzeromat}
\newcommand{\zm}{\linzeromat}


\newcommand{\itermat}{\linG{}}
\newcommand{\itermatwj}{\linG{}_{J_{\omega}}}
\newcommand{\eye}{\linI{}}

\newcommand{\bxx}{11}
\newcommand{\bxxc}{1\hat{1}}
\newcommand{\bxy}{12}
\newcommand{\bxyc}{1\hat{2}}

\newcommand{\bxcx}{\hat{1}1}
\newcommand{\bxcxc}{\hat{1}\hat{1}}
\newcommand{\bxcy}{\hat{1}2}
\newcommand{\bxcyc}{\hat{1}\hat{2}}

\newcommand{\byx}{21}
\newcommand{\byxc}{2\hat{1}}
\newcommand{\byy}{22}
\newcommand{\byyc}{2\hat{2}}

\newcommand{\bycx}{\hat{2}1}
\newcommand{\bycxc}{\hat{2}\hat{1}}
\newcommand{\bycy}{\hat{2}2}
\newcommand{\bycyc}{\hat{2}\hat{2}}

\newcommand{\bx}{1}
\newcommand{\by}{2}
\newcommand{\bxc}{\hat{1}}
\newcommand{\byc}{\hat{2}}

%\newcommand{\xcx}{\hat{1}1}
%\newcommand{\xcxc}{\hat{1}\hat{1}}
%\newcommand{\yy}{11}
%\newcommand{\ycy}{\hat{1}1}
%\newcommand{\yyc}{1\hat{1}}
%\newcommand{\ycyc}{\hat{1}\hat{1}}



% vectors in linear algebra has a bar on top.
\newcommand{\linvec}[1]{\bar{#1}}
\newcommand{\lina}{\linvec{a}} % vector a
\newcommand{\linb}{\linvec{b}} % vector b
\newcommand{\linc}{\linvec{c}} % c
\newcommand{\lind}{\linvec{d}} % d
\newcommand{\linee}{\linvec{e}} % e \line already defined
\newcommand{\linf}{\linvec{f}} % f
\newcommand{\ling}{\linvec{g}} % g
\newcommand{\linh}{\linvec{h}} % h
\newcommand{\lini}{\linvec{i}} % i
\newcommand{\linj}{\linvec{j}} % j
\newcommand{\link}{\linvec{k}} % k
\newcommand{\linl}{\linvec{l}} % l
\newcommand{\linm}{\linvec{m}} % m
\newcommand{\linn}{\linvec{n}} % n
\newcommand{\lino}{\linvec{o}} % o
\newcommand{\linp}{\linvec{p}} % p
\newcommand{\linq}{\linvec{q}} % q
\newcommand{\linx}{\linvec{x}} % 
\newcommand{\liny}{\linvec{y}} % 
\newcommand{\linz}{\linvec{z}} % 
\newcommand{\linr}{\linvec{r}} % r - normally used for residuals
\newcommand{\lins}{\linvec{s}} % r - normally used for residuals
\newcommand{\lint}{\linvec{t}} % r - normally used for residuals
\newcommand{\linu}{\linvec{u}} % r - normally used for residuals
\newcommand{\linv}{\linvec{v}} % r - normally used for residuals
\newcommand{\linw}{\linvec{w}} % r - normally used for residuals
\newcommand{\linzerovec}{\linvec{0}}

\newcommand{\tangvec}{\hat{t}^{\,i}}
\newcommand{\normvec}{\hat{n}}

% Used for the constraint vector component in Chapter 4
\newcommand{\constraintc}{\hat{c}}
% Used for the constraint vector in Chapter 4
\newcommand{\constraintcvec}{\mathbf{\constraintc}}
% Used for the Lagrange multiplier variable in Chapter 4
\newcommand{\lgrVar}{\lambda} % the actual lgr multiplier variable
\newcommand{\lgrNewtCor}{\Lambda} % Used for the Newton corrections
% Lgr subscript Used for anything else related, such as ndof
\newcommand{\lgrSubscr}{\ell} 
\newcommand{\nsSubscr}{ns}
\newcommand{\veloSubscr}{u}


\newcommand{\eigFull}{\nu} % eigenvalue for full jac
\newcommand{\eigMo}{\mu} % eigenvalue for momentum sized jac
\newcommand{\eigMoL}{\lambda} % = - mu/(mu+1)
\newcommand{\eigMotL}{\tilde{\lambda}} % = sigma/lambda



\newcommand{\kspace}[1]{\mathcal{K}_#1}
\newcommand{\knspace}{\kspace{n}}
\newcommand{\fwj}{\linf{}_{J_{\omega}}}
\newcommand{\numitererror}[1]{\lind{}^{[#1]}}
\newcommand{\kthitererror}{\numitererror{k}}
\newcommand{\initialitererror}{\numitererror{0}}

\newcommand{\tenpower}[1]{\times10^{#1}}


\newcommand{\coma}{a} % vector a
\newcommand{\comb}{b} % vector b
\newcommand{\comc}{c} % c
\newcommand{\comd}{d} % d
\newcommand{\comee}{e} % \line is defined already
\newcommand{\comf}{f} % d
\newcommand{\comg}{g} % d
\newcommand{\comh}{h} % d
\newcommand{\comi}{i} % d
\newcommand{\comj}{j} % d
\newcommand{\comk}{k} % d
\newcommand{\coml}{l} % d
\newcommand{\comm}{m} % d
\newcommand{\comn}{n} % d
\newcommand{\como}{o} % d
\newcommand{\comp}{p} % d
\newcommand{\comx}{x} % 
\newcommand{\comy}{y} % 
\newcommand{\comz}{z} % 
\newcommand{\comr}{r} % r - normally used for residuals
\newcommand{\comu}{u} % r - normally used for residuals
\newcommand{\comv}{v} % r - normally used for residuals
\newcommand{\comw}{w} % r - normally used for residuals


% Term to augment the momentum, used in Chap 4, discretisation
\newcommand{\constraintterm}{\Pi_{\lambda}}
% First variation - used in Chapter 4, discretisation
\newcommand{\variation}{\delta}


\newcommand{\linAx}{\linA{}\linx{}}  %Ax = b
\newcommand{\linAexactx}{\linA{}\exactx{}} %A \tild{x}^*
\newcommand{\linAappoxx}[1]{\linA{}\approxx{#1}}%A \tilde{x}^#1

% This is quite specific to my thesis, this is the dimensions of my
% generic linear system Ax = b
% Dimensions of A is n x n, x is n, b is n
\newcommand{\genrowdim}{N}
\newcommand{\gencoldim}{M}
\newcommand{\dimN}{N}
\newcommand{\dimM}{M}
\newcommand{\dimlinA}{\realmatrices{N}{N}}
\newcommand{\dimlinx}{\realvectors{N}}
\newcommand{\sizeofmat}{N}
\newcommand{\forioneton}{i = 1,\ldots,\genrowdim{}}
\newcommand{\NdofLgr}{N_{\ell}} 
\newcommand{\NdofLgrI}{N_{\ell,i}} 
\newcommand{\NdofP}{N_{p}}
\newcommand{\NdofNS}{N_{ns}}

\newcommand{\NdofU}[1][]{%
\ifthenelse{\isempty{#1}}{N_{u}}{N_{u_{#1}}}%
}

\newcommand{\NdofHatU}[1][]{%
\ifthenelse{\isempty{#1}}{N_{\hat{u}}}{N_{\hat{u}_{#1}}}%
}

\newcommand{\NdofEle}{N_{e}} % Ndof within an element
%\newcommand{\NInt}{N_{int}} % number of integration points.

\newcommand{\NInt}[1][]{%
\ifthenelse{\isempty{#1}}{N_{int}}{N_{{#1}int}}%
}


\newcommand{\numboundvdof}{N_{\bound u}}

% Key for the below variable:
% size of linear system
% number of unknowns
% problem size
\newcommand{\probsizeN}{N} % This is used in my results chapter
\newcommand{\timestepsize}{\Delta}

\newcommand{\wscaling}{\sigma}
% Function spaces
\newcommand{\Lpspace}[1]{L^{#1}}
\newcommand{\Ltwo}{\Lpspace{2}}
\newcommand{\Hspace}[1]{\mathcal{H}^{#1}}
\newcommand{\Hone}{\Hspace{1}}
\newcommand{\Honevecsoln}{\mathbf{H}^{1}_E}
\newcommand{\Honevectest}{\mathbf{H}^{1}_0}
\newcommand{\zspace}{Z}

\newcommand{\genjac}{\mathcal{F}}
\newcommand{\genjacJ}{\mathcal{J}}

% Matrix for inexact newton method demo.
\newcommand{\INMat}{\mathcal{F}} 

% Original NS Jacobian
\newcommand{\nsjac}{\mathcal{F}_{ns}}

% Used to denote F__ns +L^T W^-1 L
\newcommand{\modnsjac}{\tilde{\mathcal{F}}_{ns}}

% Used to denote F_ns + L^T \widehat{D}_M L
% Where widehat{D}_M is the sum of the squared diagonal of the mass matrices
% See the results chapter (serial results)
\newcommand{\diagwmodnsjac}{\widehat{\mathcal{F}}_{ns}}

\newcommand{\genprec}{\mathcal{P}}
\newcommand{\precAL}{\mathcal{P}_{AL}}
\newcommand{\precmAL}{\tilde{\mathcal{P}}_{AL}}
\newcommand{\precNS}{\mathcal{P}_{NS}}
\newcommand{\precE}{\mathcal{P}_E}
\newcommand{\precLSC}{\mathcal{P}_{LSC}}
\newcommand{\preciLSC}{\widehat{\mathcal{P}}_{LSC}}
\newcommand{\preciLSCLIST}{\hat{\mathcal{P}}_{LSC}}
\newcommand{\precNSLSC}{\mathcal{P}_{LSC}^{NS}}
\newcommand{\precNSiLSC}{\widehat{\mathcal{P}}_{LSC}^{NS}}
\newcommand{\precNSiLSCLIST}{\hat{\mathcal{P}}_{LSC}^{NS}}
%\newcommand{\bL}{L} % block L
%\newcommand{\bLt}{L^{t}}
\newcommand{\ExactW}{W}
\newcommand{\gendiagM}{\widehat{W}}
\newcommand{\diagW}{\widehat{W}}

\newcommand{\pmm}{M_{p}} % Pressure mass matrix
\newcommand{\diagpmm}{\widehat{M}_{p}}
\newcommand{\vmm}{M_{u}} % velocity mass matrix
\newcommand{\diagvmm}{\widehat{M}_{u}}
\newcommand{\cmm}{M_{l}} % constrained mass matrix
\newcommand{\diagcmm}{\widehat{M}_{l}}
\newcommand{\gmm}{M} % general mass matrix
\newcommand{\diaggmm}{\widehat{M}}



\newcommand{\schur}{S}
\newcommand{\schurapprox}{\tilde{S}}
\newcommand{\schurLSC}{S_{LSC}}
\newcommand{\schurPCD}{S_{PCD}}

% This is from the full preconditioner, the augmentation one.
\newcommand{\fullL}{L}
% This one can be added to the NS momentum F block.
\newcommand{\subLMomentumSized}{\widehat{L}} % 
\newcommand{\subL}{\widehat{L}} % 

% F block from NSE
\newcommand{\momentumF}{F}
% Modified F block.
\newcommand{\modMomentumF}{\widehat{\momentumF}}
\newcommand{\modMomentumFLIST}{\hat{\momentumF}}

% Discrete divergence block B
\newcommand{\divergenceBMat}{B}
\newcommand{\nsB}{B}
\newcommand{\nsF}{F}
\newcommand{\nsFa}{F_{a}}
\newcommand{\nsBt}{B^{T}}
\newcommand{\nsP}{P} % pressure poission block
\newcommand{\nsBQBt}{\nsB\diagvmm^{-1}\nsBt}
\newcommand{\nsL}{L}
\newcommand{\nsLt}{L^{T}}
\newcommand{\nsW}{W}
\newcommand{\nsInvW}{W^{-1}}
\newcommand{\nsLMo}{\widehat{L}}
\newcommand{\nsLMot}{\widehat{L}^{T}}


\newcommand{\nsFbb}{F_{bb}}
\newcommand{\nsFbl}{F_{b\ell}}
\newcommand{\nsFlb}{F_{\ell b}}
\newcommand{\nsFll}{F_{\ell \ell}}

\newcommand{\nsFllSchur}{S_{\ell \ell}}

% Stuff in analysis:
% New F which contains L, but not B.
\newcommand{\nstF}{\tilde{F}}
% Like nstF but augmented.
\newcommand{\nstFa}{\tilde{F}_a}

% Expanded B to include o block of size of constaints.
\newcommand{\nstB}{\tilde{B}}
\newcommand{\nstBt}{\tilde{B}^{T}}



% Consistent pressure-Poisson matrix P = B \hat{Q} B^T
\newcommand{\consistentPressurePoissonMat}{P}

\newcommand{\reyparam}{Re}
\newcommand{\subprobeig}{\tilde{\lambda}}
\newcommand{\seigi}{\tilde{\lambda}_i}
\newcommand{\seigr}{\tilde{\lambda}_r}

%eigenvectors for analysis
\newcommand{\evns}{\linv_{ns}}
\newcommand{\evlgr}{\linv_{\ell}}
\newcommand{\evp}{\linv_{p}}
\newcommand{\evu}{\linv_{u}}
\newcommand{\evb}{\linv_{b}}
\newcommand{\evec}[1]{\linv_{#1}}

\newcommand{\FSizedLTWL}{\wscaling\subLMomentumSized^\raytran\gendiagM^{\rayinv}\subLMomentumSized}

% Preconditioner splitting for relaxation methods, M and N
\newcommand{\precsplitM}{\tilde{M}}
\newcommand{\precsplitN}{\tilde{N}}

% ILU preconditioner
\newcommand{\ILUL}{\tilde{L}}
\newcommand{\ILUU}{\tilde{U}}


% For graphs
% Type
\newcommand{\RRLoglogplot}{Log-log plot of}
\newcommand{\RRSemilogplot}{Semi-log plot of}
\newcommand{\RRExecution}{execution times (s)}
\newcommand{\RRIterasion}{iteration counts}
\newcommand{\RRFuncOfDis}{as a function of the problem size}
\newcommand{\RRPE}{the GMRES$+\precE$ solver}
\newcommand{\RRPELSC}{the GMRES$+\precLSC$ solver}
\newcommand{\RRPALSC}{the GMRES$+\preciLSC$ solver}

% For tables
\newcommand{\RRGMRESIts}{The average number of iterations taken by}
\newcommand{\RRGMRESExec}{The execution times (s) taken by}
\newcommand{\LLGMRESIts}{the average number of iterations taken by}
\newcommand{\LLGMRESExec}{the execution times (s) taken by}

\newcommand{\RRBlockApproxF}{using different block approximations of
  $\modMomentumF$ in the LSC preconditioner}
\newcommand{\RRSmoother}{with different smoothers and different $V-$cycles used for the AMG approximation of the $\modMomentumF$ block}
\newcommand{\RRStrnStudy}{for varying strength of dependence parameter
  $\strn$ in the AMG approximation of the $\modMomentumF$ block within the LSC
  preconditioner}
\newcommand{\RRJdampStudy}{for varying Jacobi damping parameter
  $\jdamp$ for the AMG approximation of the $\modMomentumF$ block within the LSC
  preconditioner}


% For Both?
\newcommand{\RRRESareFOR}{The results are presented for}



