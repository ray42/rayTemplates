\begin{figure}[!h]
\centering
\begin{tikzpicture}[scale=0.5,node distance=3cm]
% node size
\newcommand\nodesize{0.45}
%\draw[help lines] (0,0) grid (17,1);
%%%%%%%%%%%%%%%%%%%%%%%%%%%%%%%%%%%%%%%%%%%%%%%%%%%%%%%%%%%%%%%%%%%%%%%%%%%%
%%%%%%%%%%%%%%%%%%%%%%%%%%%%%%%%%%%%%%%%%%%%%%%%%%%%%%%%%%%%%%%%%%%%%%%%%%%%
\begin{scope}[shift={(0,0)}]
% The box
%\tikzstyle{tikzraybox} = [rectangle, rounded corners, minimum width=3cm, minimum height=1cm,text centered, draw=black, fill=red!30]
\tikzstyle{tikzraybox} = [rectangle, rounded corners, minimum width=2cm, minimum height=0.5cm,text centered, draw=black]
% The arrow
\tikzstyle{tikzrayarrow} = [very thick,->,>=stealth]
%%%%%%%%%%%%%%%%%%%%%%%%%%%%%%%%%%%%%%%%%%%%%%%%%%%%%%%%%%%%%%%%%%%%%%%%%%%%
%%%%%%%%%%%%%%%%%%%%%%%%%%%%%%%%%%%%%%%%%%%%%%%%%%%%%%%%%%%%%%%%%%%%%%%%%%%%
% Now start drawing
%\node(ppi) [tikzraybox, label=above:{\small \ctt{ppi}}] {};
%\node(pi) [tikzraybox, label=above:{\small \ctt{pi}}, right of=ppi] {};
%\node(ival) [tikzraybox, label=above:{\small \ctt{ival}},right of=pi] {1024};

%\draw [tikzrayarrow] (ppi.center) -- (pi);
%\draw [tikzrayarrow] (pi.center) -- (ival);

%%%%%%%%%%%%%%%%%%%%%%%%%%%%%%%%%%%%%%%%%%%%%%%%%%%%%%%%%%%%%%%%%%%%%%%%%%%%
% First draw the main box
\draw (0,0) rectangle (17,1);
% Draw the segments
\draw (1,0)--(1,1);
\draw (2,0)--(2,1);
\draw (3,0)--(3,1);
\draw (4,0)--(4,1);
\draw (5,0)--(5,1);

\pgfmathsetmacro{\StartX}{0.5}
\pgfmathsetmacro{\StartY}{0.5}
\node at (\StartX,\StartY){\ctt{0}};
\node at (\StartX+1,\StartY){\ctt{1}};
\node at (\StartX+2,\StartY){\ctt{2}};
\node at (\StartX+3,\StartY){\ctt{3}};
\node at (\StartX+4,\StartY){\ctt{4}};
\node at (\StartX+10.5,\StartY){\textit{unconstructed elements}};

% Bot arrows going up
\pgfmathsetmacro{\BotArrowYStart}{-1}
\pgfmathsetmacro{\BotArrowYEnd}{-0.1}
\draw[tikzrayarrow](\StartX,\BotArrowYStart)--(\StartX,\BotArrowYEnd);
\draw[tikzrayarrow](\StartX+5,\BotArrowYStart)--(\StartX+5,\BotArrowYEnd);
\draw[tikzrayarrow](\StartX+17,\BotArrowYStart)--(\StartX+17,\BotArrowYEnd);

% Bot words
\pgfmathsetmacro{\BotWordY}{-1.7}
\node at (\StartX,\BotWordY){\ctt{element}};
\node at (\StartX+5,\BotWordY){\ctt{first\_free}};
\node at (\StartX+17,\BotWordY){\ctt{cap}};

\end{scope}
\end{tikzpicture}
\caption[StrVec pointers]
 {\ctt{StrVec} pointers.
  \label{figC13N2StrVecPointers}}
\end{figure}
